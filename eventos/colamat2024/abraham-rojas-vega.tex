% VI CONGRESO LATINOAMERICA DE MATEM?TICA - COLAMAT 2024
\documentclass[12pt,a4paper]{article}
\usepackage[latin1]{inputenc}
\usepackage[spanish]{babel}
\usepackage{amsmath}
\usepackage{amsfonts}
\usepackage{amssymb}
\usepackage{graphicx}
\usepackage{lmodern}
\usepackage{xcolor}
\usepackage{fancyhdr}
\usepackage[left=2cm,right=2cm,top=3cm,bottom=3cm]{geometry}
\lhead{V COLAMAT 2024 : Congreso Latinoamericano de Matem�tica}
\rhead{UNJBG}
\lfoot{FACI\,-\,DAMS\,-\,ESMA}
\cfoot{\thepage}
\rfoot{Tacna, Per�, 2024}
\renewcommand{\headrulewidth}{0.15pt}
\renewcommand{\footrulewidth}{0.15pt}
\begin{document}
\pagestyle{fancy}
% PARTE 01: SECCI?N TITULO
% Titulo en Espa?ol o Portugues
\begin{center}
{\large  \textbf{Conjuntos simpliciales y aplicaciones}}
\end{center}
% Titulo en Ingles
\begin{center}
{\bf \textcolor{blue}{Simplicial sets and applications}}
\end{center}

% PARTE 02: SECCI?N AUTORES
% Autor 01: Colocar Nombres y Apellidos
\begin{center}
{\underline{MSc. Abraham Rojas Vega}\\
{\it {abraham.rojas@usp.br}\\
\small{Instituto de Ci�ncias Matem�ticas e de Computa��o, Universidade de S�o Paulo, Brasil}}}\\
\end{center}

% PARTE 03: SECCI?N ABSTRACT
\begin{abstract}
    Los conjuntos simpliciales son un concepto fundamental en la topolog�a algebraica, proporcionando un marco combinatorio para estudiar espacios topol�gicos. Generalizan el concepto de complejo simplicial, enfoc�ndose en la estructura combinatoria de los s�mplices y sus caras. Desempe�an un papel crucial en la formulaci�n de la teor�a de homotop�a, proporcionando un marco para entender categor�as derivadas y categor�as modelo, que son vitales en la geometr�a algebraica moderna y la f�sica matem�tica. En esta presentaci�n, se discutir�n los conceptos b�sicos de los conjuntos simpliciales y se explorar�n algunas de las aplicaciones mencionadas.\\

\noindent\textbf{Palabras Claves:} Conjuntos simpliciales, teor�a de homotop�a, topolog�a algebraica.
\end{abstract}

% PARTE 04: SECCI?N REFERENCIAS BIBLIOGR?FICAS
\begin{thebibliography}{10}
\bibitem{ref_book1}
May P. A Concise Course in Algebraic Topology. University of Chicago Press, 1999.

\bibitem{ref_book2}
Weibel C. An Introduction to Homological Algebra. Cambridge University Press. 1994.


\bibitem{ref_book3}
Richter B. From Categories to Homotopy Theory. Cambridge University Press, 2020.

\end{thebibliography}
\end{document}