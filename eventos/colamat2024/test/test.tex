\documentclass{beamer}
\usetheme{Madrid}
\usecolortheme{whale}

\usepackage{amsmath}
\usepackage{amssymb}
\usepackage{tikz}
\usepackage{tikz-cd}
\usepackage{mathrsfs}
\usepackage{graphicx}

\title{A Comprehensive Introduction to Simplicial Sets}
\subtitle{From Foundations to Applications}
\author{Based on Work by Emily Riehl}
\date{\today}

\begin{document}

\frame{\titlepage}

\begin{frame}{Outline}
  \tableofcontents
\end{frame}

% SECTION 1: INTRODUCTION
\section{Introduction}

\begin{frame}{What Are Simplicial Sets?}
  \begin{itemize}
    \item Simplicial sets generalize simplicial complexes.
    \item Provide a combinatorial framework for:
      \begin{itemize}
        \item Algebraic topology
        \item Higher category theory
      \end{itemize}
    \item Key features:
      \begin{itemize}
        \item Flexibility in modeling homotopy types
        \item Strong categorical properties
        \item Applications in modern mathematics
      \end{itemize}
  \end{itemize}
\end{frame}

\begin{frame}{Historical Context}
  \begin{itemize}
    \item Originated in algebraic topology (1940s-50s)
    \item Developed to model homotopy theory
    \item Key milestones:
      \begin{itemize}
        \item Simplicial objects (Eilenberg-Zilber)
        \item Kan complexes (1950s)
        \item Quillen model categories (1960s)
        \item Higher category theory (2000s)
      \end{itemize}
    \item Modern applications in derived geometry and higher topos theory
  \end{itemize}
\end{frame}

\section{Category Theory Prerequisites}

\begin{frame}{Category Theory Basics}
  \begin{itemize}
    \item A category \(\mathcal{C}\) consists of:
      \begin{itemize}
        \item Objects \(\text{ob }\mathcal{C}\)
        \item Morphisms \(\text{mor }\mathcal{C}\)
        \item Composition \((f \circ g)\) and identities
      \end{itemize}
    \item Functors: \(F: \mathcal{C} \to \mathcal{D}\)
    \item Natural transformations: \(\alpha: F \Rightarrow G\)
  \end{itemize}
\end{frame}

\begin{frame}{Yoneda Lemma}
  \begin{theorem}[Yoneda Lemma]
    For any small category \(\mathcal{C}\), object \(c \in \mathcal{C}\), and functor \(F: \mathcal{C}^{op} \to \mathbf{Set}\):
    \[
    \text{Nat}(\mathcal{C}(-, c), F) \cong Fc
    \]
  \end{theorem}
  \begin{itemize}
    \item Yoneda embedding:
      \[
      y: \mathcal{C} \to \mathbf{Set}^{\mathcal{C}^{op}}, \quad c \mapsto \mathcal{C}(-, c)
      \]
    \item Full and faithful functor
    \item Foundation for simplicial sets
  \end{itemize}
\end{frame}

\section{The Simplex Category}

\begin{frame}{The Category \(\Delta\)}
  \begin{definition}
    \(\Delta\) is the category whose:
    \begin{itemize}
      \item Objects are finite, non-empty, totally ordered sets
        \[
        [n] = \{0, 1, \ldots, n\}
        \]
      \item Morphisms are order-preserving maps
    \end{itemize}
  \end{definition}
\end{frame}

\begin{frame}{Structure of \(\Delta\)}
  \begin{itemize}
    \item Generating morphisms:
      \begin{itemize}
        \item Coface maps \(d_i: [n-1] \to [n]\)
        \item Codegeneracy maps \(s_i: [n+1] \to [n]\)
      \end{itemize}
    \item Relations among morphisms:
      \[
      d_jd_i = d_id_{j-1}, \quad i < j
      \]
      \[
      s_js_i = s_is_{j+1}, \quad i \leq j
      \]
      \[
      s_jd_i = \begin{cases} 
        1 & i = j, j+1 \\
        d_is_{j-1} & i < j \\
        d_{i-1}s_j & i > j+1
      \end{cases}
      \]
  \end{itemize}
\end{frame}

\section{Simplicial Sets}

\begin{frame}{Definition of Simplicial Sets}
  \begin{definition}
    A simplicial set is a functor:
    \[
    X: \Delta^{op} \to \mathbf{Set}
    \]
  \end{definition}
  \begin{itemize}
    \item Equivalent to:
      \begin{itemize}
        \item Sets \(X_n\) for \(n \geq 0\)
        \item Face maps \(d_i: X_n \to X_{n-1}\)
        \item Degeneracy maps \(s_i: X_n \to X_{n+1}\)
        \item Satisfying dual relations to \(\Delta\)
      \end{itemize}
  \end{itemize}
\end{frame}

\begin{frame}{Key Properties of Simplicial Sets}
  \begin{itemize}
    \item \(X_n\): Set of \(n\)-simplices
    \item Face maps \(d_i\):
      \[
      d_i(x) = \text{remove the } i\text{-th vertex of } x
      \]
    \item Degeneracy maps \(s_i\):
      \[
      s_i(x) = \text{collapse edge between } i \text{ and } i+1
      \]
    \item Degenerate simplices: images of degeneracy maps
    \item Non-degenerate simplices: not degenerate
  \end{itemize}
\end{frame}

\section{Examples of Simplicial Sets}

\begin{frame}{Standard Simplex \(\Delta^n\)}
  \begin{itemize}
    \item Represented functor:
      \[
      \Delta^n = \Delta(-, [n])
      \]
    \item \(k\)-simplices:
      \[
      (\Delta^n)_k = \Delta([k], [n])
      \]
    \item Unique non-degenerate \(n\)-simplex: identity map
    \item Non-degenerate \(k\)-simplices: injective maps \([k] \to [n]\)
  \end{itemize}
\end{frame}

\begin{frame}{Nerve of a Category}
  \begin{itemize}
    \item For category \(\mathcal{C}\), the nerve \(N\mathcal{C}\):
      \begin{itemize}
        \item \(N\mathcal{C}_0 = \text{ob }\mathcal{C}\)
        \item \(N\mathcal{C}_1 = \text{mor }\mathcal{C}\)
        \item \(N\mathcal{C}_2 = \{f, g \mid \text{cod}(f) = \text{dom}(g)\}\)
      \end{itemize}
    \item Face maps:
      \begin{itemize}
        \item \(d_0, d_2\): drop outer arrows
        \item \(d_1\): compose inner arrows
      \end{itemize}
    \item Degeneracy maps: insert identity morphisms
  \end{itemize}
\end{frame}

\begin{frame}{Singular Complex}
  \begin{itemize}
    \item For topological space \(Y\), the singular complex \(SY\):
      \[
      (SY)_n = \text{Top}(\Delta^n, Y)
      \]
    \item \(n\)-simplices are continuous maps from \(\Delta^n\) to \(Y\)
    \item Face and degeneracy maps defined by precomposition
  \end{itemize}
\end{frame}

\section{Kan Complexes and Quasi-Categories}

\begin{frame}{Kan Complexes}
  \begin{definition}
    A Kan complex is a simplicial set \(X\) where every horn has a filler:
    \[
    \Lambda^n_k \to X \quad \text{extends to} \quad \Delta^n \to X
    \]
  \end{definition}
  \begin{itemize}
    \item Models for homotopy types
    \item Example: Singular complex \(SY\) of a space \(Y\)
  \end{itemize}
\end{frame}

\begin{frame}{Quasi-Categories}
  \begin{definition}
    A quasi-category is a simplicial set where every inner horn has a filler.
  \end{definition}
  \begin{itemize}
    \item Models for (\(\infty, 1\))-categories
    \item Example: Nerve of any ordinary category
  \end{itemize}
\end{frame}

\section{Applications}

\begin{frame}{Applications in Algebraic Topology}
  \begin{itemize}
    \item Singular homology:
      \[
      H_n(X) = H_n(S_*X)
      \]
    \item Classifying spaces via nerve functor
    \item Homotopy theory: Kan complexes as models
  \end{itemize}
\end{frame}

\begin{frame}{Applications in Higher Category Theory}
  \begin{itemize}
    \item (\(\infty, 1\))-categories as quasi-categories
    \item Higher categorical limits and colimits
    \item Derived functors and higher topos theory
  \end{itemize}
\end{frame}

\section{Further Reading}

\begin{frame}{Recommended Texts}
  \begin{itemize}
    \item Classical References:
      \begin{itemize}
        \item May: "Simplicial Objects in Algebraic Topology"
        \item Gabriel-Zisman: "Calculus of Fractions and Homotopy Theory"
      \end{itemize}
    \item Modern Developments:
      \begin{itemize}
        \item Lurie: "Higher Topos Theory"
        \item Goerss-Jardine: "Simplicial Homotopy Theory"
      \end{itemize}
  \end{itemize}
\end{frame}

\end{document}