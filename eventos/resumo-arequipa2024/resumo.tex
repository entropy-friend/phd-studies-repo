\documentclass[12pt]{article}
\usepackage[utf8]{inputenc}
\usepackage[spanish]{babel}

\title{Introducción a la K-Teoría Algebraica y Aplicaciones}
\author{Abraham Rojas Vega}
\date{V SEMINARIO INTERNACIONAL DE MATEMÁTICAS PERÚ-BRASIL 2024}

\begin{document}

\maketitle

\begin{abstract}
La K-teoría algebraica es una rama de las matemáticas que estudia los invariantes algebraicos asociados a anillos y módulos. En esta charla, se presentará una introducción a los conceptos fundamentales de la K-teoría algebraica, incluyendo los grupos K0, K1 y K2. Además, se discutirán diversas aplicaciones de la K-teoría en topología algebraica (fibrados vectoriales y su clasificación mediante invariantes topológicos), la teoría de números (cuerpos numéricos y sus extensiones) y la geometría algebraica(clasificación de variedades). Finalizaremos con comentarios sobre la K-Teoría en áreas externas a la Matemática.

\end{abstract}

\begin{thebibliography}{9}
\bibitem{reference1}
Weibel, Charles A.
\textit{The K-book: An Introduction to Algebraic K-theory}.
American Mathematical Society, 2013.

\bibitem{reference2}
Vasershtein, L.
\textit{Título del artículo}.
Russian Mathematical Survey, 31(4), 89-156, 1976.
\end{thebibliography}
\end{document}

