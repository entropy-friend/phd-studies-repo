%------------------------------------------------------------------------
%  Template    
%------------------------------------------------------------------------

% This file will be compiled by a computer program.
% Please do not change preamble (before II. COMMANDS ENTERED BY THE AUTHOR  ).

% Este archivo será procesado automaticamente por un programa.
% Por favor no altere nada en el preamble (hasta II. COMMANDS ENTERED BY THE AUTHOR  ).


%------------------------
%  I. PREAMBLE        
%-----------------------

% I.1 Class of document
\documentclass[11pt,a4paper]{article}

% I.2 Language
\usepackage[spanish, english]{babel}

% I.3 Encodings
\usepackage[utf8]{inputenc} % UTF8 encodings 
\usepackage[T1]{fontenc} 
\usepackage{newtxtext} 

% I.4 AMS-LaTeX      
\usepackage{amsmath} % it's include amsfonts amstext, amsgen, amsbsy, amsopn too.
\usepackage{amssymb} % it's include amsfonts too
\usepackage{amsthm}
\usepackage{amscd} 

%
\usepackage{lipsum}

% I.5 Page layout
\usepackage[margin=3cm]{geometry}

% I.6 Title and Abstract customization
\renewenvironment{abstract}{%
	\section*{\abstractname}}
	{} 
\usepackage[affil-it]{authblk}

% I.7 Title and author information
\title{Una invitación a la Estabilidad Homológica}
\author[1]{Abraham Rojas Vega \thanks{abraham.rojas@usp.br} }
%\author[2]{Autor }
%\author[3]{Autor }
\affil[1]{Instituto de Ciências Matemáticas e de Computação, Universidade de São Paulo, Brasil}
%\affil[2]{Institución del autor }
%\affil[3]{Institución del autor }
%

\date{} %Please, do not change -  POR FAVOR, NO MODIFICAR - .

% Others

\usepackage[backref=page]{hyperref}
%--------------------------------------------------------------------------
%  II. COMMANDS ENTERED BY THE AUTHOR        
%--------------------------------------------------------------------------

% Please, include ONLY the COMMANDS that will be used in the abstract, if necessary.

% Por favor, incluya SOLO los COMANDOS que se utilizarán en el resumen, en caso sea necesario. 

\usepackage{url}

%--------------------------------------------------------------------------
%  III. DOCUMENT       
%--------------------------------------------------------------------------                                     
\pagenumbering{gobble}

\begin{document}

%The main language of the text is English. If the abstract will be written in Spanish, please uncomment the following command:

% El idioma principal del texto es English. Si el resumen será escrito en español, por favor, descomentar el siguiente comando:


%---------
\selectlanguage{spanish}
%---------

\begin{flushleft}
\bfseries\footnotesize VI International Meeting in Mathematical Sciences\\
Ayacucho, Perú\\
December 10--12, 2024
\end{flushleft}

{\let\newpage\relax\maketitle}

\begin{abstract}

	La estabilidad homológica ha demostrado ser una herramienta poderosa para el cálculo de la homología de grupos, como en los grupos lineales generales, los grupos de clases de mapeo o los automorfismos de grupos libres. La estabilidad de $\operatorname{GL}(n)$ es fundamental para el cálculo de los $K-$grupos de orden superior de anillos. La estabilidad homológica tiene aplicaciones en teoría de representación, geometría algebraica y otras áreas. En esta charla, daremos una introducción a la estabilidad homológica y discutiremos algunas de sus aplicaciones, enfocándonos en la $K-$teoría algebraica.

\end{abstract}

\begin{thebibliography}{99}

\bibitem{label1}{Suslin, A. A.}, {\em Stability in algebraic K-theory}. Algebraic K-theory 966, 304-333, 1982.

\bibitem{label2}{Kupers, A.}, {\em Homological Stability Minicourse}. Disponible en \url{https://www.utsc.utoronto.ca/people/kupers/wp-content/uploads/sites/50/homstab.pdf}.



\end{thebibliography}



\end{document}




