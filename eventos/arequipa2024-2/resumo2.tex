\documentclass[12pt]{article}
\usepackage[utf8]{inputenc}
\usepackage[spanish]{babel}
\usepackage{authblk}
\usepackage{url}

\title{Generalizaciones de la Dualidad de Poincaré}
\author{Abraham Rojas Vega}
\affil{Instituto de Ciências Matemáticas e de Computação, Universidade de São Paulo, Brasil}
\date{}

\begin{document}

\maketitle 

\begin{abstract}

La dualidad de Poincaré (DP) es un resultado fundamental en topología algebraica que relaciona la homología y la cohomología de un espacio topológico. Originalmente, la DP estaba formulada para variedades compactas y trianguladas. En esta charla mostraremos diversas generalizaciones de la DP, considerando espacios más generales y otras formas de expresar la DP. También presentaremos dos teorías de homología especiales para variedades singulares: la homología de intersección y la homología de espacios de intersección, que consiguen generalizar varias propiedades de la homología de variedades suaves, entre ellas la DP.

\end{abstract}

\begin{scriptsize}
    
    \begin{thebibliography}{9}
        \bibitem{reference1}
        Hatcher, A.
        \textit{Algebraic Topology}. 2021. Disponible en: \url{http://www.math.cornell.edu/~hatcher/AT/ATpage.html}.
        
        \bibitem{reference2}
        Maxim, L.
        \textit{Intersection Homology \& Perverse Sheaves}.
        Springer, 2019.
        
        \bibitem{reference3}
        Banagl, M.
        \textit{Intersection Spaces, Spatial Homology Truncation, and String Theory}
        Springer, 2010.
    \end{thebibliography}
\end{scriptsize}
\end{document}
