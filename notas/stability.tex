
\part{Homological stability}
\chapter{Motivation}


\cite{kupersHomologicalStabilityMinicourse2021}

The symmetric group $\Sigma_n$ is the group of bijections of the finite set $\underline{n}=\{1, \ldots, n\}$, under composition. The classifying space $B G$ of a discrete group $G$, such as $\Sigma_n$, is the connected space determined uniquely up to weak homotopy equivalence by the property
$$
\pi_*(B G)= \begin{cases}G & \text { if } *=1, \\ 0 & \text { otherwise }\end{cases}
$$

It can be constructed by extracting from $G$ the groupoid $* / / G$ given by:
- a single object *,
- morphisms given by $* \xrightarrow{g} *$ for $g \in G$, and
- composition given by multiplication.

We then take its nerve to obtain a simplicial set, and take the geometric realisation to get a topological space $|N(* / / G)|$; this is a model for $B G$. Exercise 1.3.1 proves it indeed has the desired property.


\begin{prop}
    $H_*\left(B \Sigma_n ; \mathbb{Z}\right)$  is the same as computing the group homology of $\Sigma_n$ with coefficients in $\mathbb{Z}$.
\end{prop}
Let us compute these groups and the homology of their classifying spaces for the first few values of $n$.

\begin{example}
    \begin{enumerate}
        \item For $n=0,1$, the group $\Sigma_n$ is trivial so its classifying space is weakly contractible and hence has trivial homology.
        \item Example 1.1.4. For $n=2, \Sigma_2$ is isomorphic to the cyclic abelian group $\mathbb{Z} / 2$. Then $B \mathbb{Z} / 2$, as constructed above, is homotopy equivalent to $\mathbb{R} P^{\infty}$. We conclude that
        $$
        H_*(B \mathbb{Z} / 2 ; \mathbb{Z})=H_*\left(\mathbb{R} P^{\infty} ; \mathbb{Z}\right)= \begin{cases}\mathbb{Z} & \text { if } *=0 \\ \mathbb{Z} / 2 & \text { if } *>0 \text { is odd, } \\ 0 & \text { if } *>0 \text { is even. }\end{cases}
        $$
    \item
        Example 1.1.5. For $n=3$, the group $\Sigma_3$ is the dihedral group $D_3$ with 6 elements (i.e. the symmetries of a triangle). A more complicated computation given in Exercise 1.3.5 yields the homology of $D_3$ :
$$
H_*\left(B D_3 ; \mathbb{Z}\right)=\left\{\begin{array}{lll}
\mathbb{Z} & \text { if } *=0 \\
\mathbb{Z} / 2 & \text { if } *>0 \text { and } * \equiv 1 \quad(\bmod 4) \\
\mathbb{Z} / 6 & \text { if } *>0 \text { and } * \equiv 3 \quad(\bmod 4), \\
0 & \text { otherwise }
\end{array}\right.
$$
    \end{enumerate}
\end{example}

\paragraph*{Conjectures}

\begin{enumerate}
    \item $\text { Each reduced homology group } \widetilde{H}_d\left(B \Sigma_n ; \mathbb{Z}\right) \text { is finite and has small exponent. }$
    \item The homology in fixed degree $*=d$ becomes independent of $n$ as $n \rightarrow \infty$.
    \item Before becoming independent of $n$, the homology only increases in size.
    \item The $p$-power torsion only changes when $p \mid n$.
\end{enumerate}


If we want to attempt to prove (2)-(4), we need a better way to compare the homology groups for different $n$ than just as abstract abelian groups. This is done by observing that the inclusion $\underline{n} \hookrightarrow \underline{n+1}$ of finite sets gives a homomorphism
$$
\sigma: \Sigma_n \longrightarrow \Sigma_{n+1},
$$
by extending a permutation of $\underline{n}$ by the identity on $n+1 \in \underline{n+1}$ to a permutation of $n+1$. Our construction of $B G$ is natural in groups and homomorphisms, so this homomorphism induces a map
$$
\sigma: B \Sigma_n \longrightarrow B \Sigma_{n+1},
$$
which in turn induces a map $\sigma_*: H_*\left(B \Sigma_n ; \mathbb{Z}\right) \rightarrow H_*\left(B \Sigma_{n+1} ; \mathbb{Z}\right)$ on homology. We can then give sharper formulations of (2)-(4) in terms of these stabilisation maps:
(2') The maps $\sigma_*$ are isomorphisms in a range increasing with $n$.\\
(3') The maps $\sigma_*$ are injective.\\
(4') The maps $\sigma_*$ are isomorphisms on $p$-power torsion unless $p \mid n+1$.\\
Property (1) holds for all finite groups, and the result which proves it also implies (4'):

\begin{prop}
For a finite group $G, \widetilde{H}_*(B G ; \mathbb{Z}[1 /|G|])=0$. More generally, for $H \subset G$ the map $\iota_*: H_*(B H ; \mathbb{Z}[1 /[G: H]]) \rightarrow H_*(B G ; \mathbb{Z}[1 /[G: H]])$ admits a right inverse $\tau$ (i.e. $\iota_* \circ \tau=\mathrm{id}$ ).
\end{prop}

To deduce (4') from Proposition 1.1.6, note that $\left[\Sigma_{n+1}: \Sigma_n\right]=n+1$ so by the long exact sequence on homology groups so that $H_*\left(B \Sigma_n ; \mathbb{Z}\right) \rightarrow H_*\left(B \Sigma_{n+1} ; \mathbb{Z}\right)$ is surjective after inverting $n+1$. Now set $n+1$ equal to $p$ and invoke (3').
It is phenomenon indicated by (2') that is the subject of this minicourse:

A sequence $X_0 \xrightarrow{\sigma} X_1 \xrightarrow{\sigma} X_2 \xrightarrow{\sigma} \cdots$ exhibits \textbf{homological stability} if the maps $\sigma_*: H_*\left(X_n ; \mathbb{Z}\right) \rightarrow H_*\left(X_{n+1} ; \mathbb{Z}\right)$ are isomorphisms in a range of degrees * increasing with $n$.

In the next two lectures we will prove the following result, due to Nakaoka [Nak60] (though he proved much more):
\begin{theo}
The sequence $B \Sigma_0 \xrightarrow{\sigma} B \Sigma_1 \xrightarrow{\sigma} B \Sigma_2 \xrightarrow{\sigma} \cdots$ exhibits homological stability. More precisely, the induced map
    $$
    \sigma_*: H_*\left(B \Sigma_n ; \mathbb{Z}\right) \longrightarrow H_*\left(B \Sigma_{n+1} ; \mathbb{Z}\right)
    $$
    is surjective if $* \leq \frac{n}{2}$ and an isomorphism if $* \leq \frac{n-1}{2}$.
\end{theo}

Remark 1.1.9. Of course, if we know property (3') holds then the range in the previous theorem in which $\sigma_*$ is an isomorphism improves to $* \leq \frac{n}{2}$. However, property ( 3 ') is rather special—related to the existence of transfer maps-and you should not expect it to hold for general sequences of classifying spaces of groups. We will not comment on it again, but see Exercise 1.3.6.\\
Remark 1.1.10. The ranges in the previous remark are optimal among those of the form $* \leq a n+b$ with $a, b \in \mathbb{Q}$.

\section{Applications}

Homological stability is a structural property of a sequence of groups, or more
generally topological spaces, but it is also useful tool. In fact, many homological stability theorems are proven in service of obtaining other mathematical results. To illustrate this, I now want to explain some straightforward applications of Theorem 1.1.8. These concern the transfer of information from low n to high n and vice-versa. They can be
obtained by other methods as well, but their generalisations to other sequences of groups
often can not.

\subsection{Altenating groups}

Recall that for path-connected $X$, the Hurewicz map $\pi_1(X) \rightarrow H_1(X ; \mathbb{Z})$ coincides with abelianisation (we are suppressing the basepoint). In particular, the map $G \rightarrow$ $H_1(B G ; \mathbb{Z})$ induces an isomorphism $G^{\text {ab }} \rightarrow H_1(B G ; \mathbb{Z})$ naturally in $G$. Thus we can understand the abelianisation of $\Sigma_n$ by computing its first homology group.
The sign homomorphism sign: $\Sigma_n \rightarrow \mathbb{Z} / 2$ yields a map
$$
\text { sign: } B \Sigma_n \longrightarrow B \mathbb{Z} / 2,
$$
which induces a map on homology. This is compatible with stabilisation, in the sense that sign $\circ \sigma=$ sign, so we get a commutative squares
$$
\begin{array}{cc}
H_1\left(B \Sigma_{n-1} ; \mathbb{Z}\right) \xrightarrow{\sigma_*} H_1\left(B \Sigma_n ; \mathbb{Z}\right) \\
\downarrow_{\text {sign }} & \mid \text { sign } \\

\mathbb{Z} / 2 \xlongequal{Z} / 2 .
\end{array}
$$

The map $H_1\left(B \Sigma_2 ; \mathbb{Z}\right) \rightarrow \mathbb{Z} / 2$ is an isomorphism because sign: $\Sigma_2 \rightarrow \mathbb{Z} / 2$ is. By Theorem 1.1.8, in the commutative diagram
the right-most top horizontal map is surjective and the other top horizontal maps are isomorphisms. A single diagram chase then deduces from the fact that the left-most vertical map is an isomorphism that all other vertical maps are.

Thus we have used homological stability to prove that
$$
\text { sign: } \Sigma_n \longrightarrow \mathbb{Z} / 2
$$
is the abelianisation for $n \geq 2$, or equivalently that the kernel of the sign homomorphism is exactly the subgroup $\left[\Sigma_n, \Sigma_n\right]$ generated by commutators. Recalling that this kernel is exactly the alternating group $A_n$, we conclude that:

\begin{theo}
$\left[\Sigma_n, \Sigma_n\right]=A_n$.
\end{theo} 

Remark 1.2.2. This is a fact you likely knew already, and elementary group-theoretic arguments exist. We could have used this fact instead to give an elementary proof of Theorem 1.1.8 in degree $*=1$.

\section{Group Completion}

Homological stability implies that for in fixed degree $*$, for $n$ sufficienty large the canonical map
$$
H_*\left(B \Sigma_n ; \mathbb{Z}\right) \longrightarrow \underset{n \rightarrow \infty}{\operatorname{colim}} H_*\left(B \Sigma_n ; \mathbb{Z}\right)
$$
is an isomorphism; the right hand side is known as the stable homology. This has two somewhat tautological consequences:
1. We can compute the right side from the left side.
2. We can compute the left side from the right side.

This is particularly interesting because the stable homology on the right side has a more familiar description.

When we constructed the stabilisation map, we used that inclusion $\underline{n} \rightarrow \underline{n+1}$ yields a homomorphism $\Sigma_n \rightarrow \Sigma_{n+1}$. More generally, disjoint union induces a homomorphism $\Sigma_n \times \Sigma_m \rightarrow \Sigma_{n+m}$, which yields "multiplication" maps
$$
B \Sigma_n \times B \Sigma_m \longrightarrow B \Sigma_{n+m},
$$
making the space $\bigsqcup_{n \geq 0} B \Sigma_n$ into a unital topological monoid (these are associative but not commutative, and it is probably better to say $E_1$-space since that is a homotopy-invariant notion).

\begin{theo}[McDuff-Segal]
     If $M$ is a homotopy-commutative unital associative topological monoid, then $H_*(M ; \mathbb{Z})\left[\pi_0^{-1}\right] \cong H_*(\Omega B M ; \mathbb{Z})$.    
\end{theo}

%entender mejor esta parte

\section{Serre's finiteness theorem and variations}

Let us now use Corollary 1.2.6. By (1) the groups $H_*\left(B \Sigma_n ; \mathbb{Z}\right)$ are finite for $*>0$. By Theorem 1.1.8 the same is true for the stable homology as long as restrict to degrees $* \leq \frac{n}{2}$. Since $n$ is arbitrary, the stable homology is finite in all positive degrees. This has the following consequence:

\begin{theo}
$\pi_*(\mathbb{S})$ is finite for all $*>0$.
\end{theo}

%we can say something smimilar about torsion....

Exercise 1.3.8 (Using Serre's finiteness theorem). Serre proved that $\pi_*(\mathbb{S})$ is finite for $*>0$. Combine this with Corollary 1.2.6 and Exercise 1.3.6 to prove that the sequence $B \Sigma_0 \xrightarrow{\sigma} B \Sigma_1 \xrightarrow{\sigma} B \Sigma_2 \xrightarrow{\sigma} \cdots$ exhibits homological stability. (Hint: you will not be able to give an explicit range.)

Remark 1.3.9. See [McD75] for a similar qualitative argument for configuration spaces of manifolds.


\chapter{Homological stability for symmetric groups}

\begin{theo}
The sequence $B \Sigma_0 \xrightarrow{\sigma} B \Sigma_1 \xrightarrow{\sigma} B \Sigma_2 \xrightarrow{\sigma} \cdots$ exhibits homological stability. More precisely,
$$
\sigma_*: H_*\left(B \Sigma_n ; \mathbb{Z}\right) \longrightarrow H_*\left(B \Sigma_{n+1} ; \mathbb{Z}\right)
$$
is surjective if $* \leq \frac{n}{2}$ and an isomorphism if $* \leq \frac{n-1}{2}$.
\end{theo}



\chapter*{Stability in algebraic K-theory}

% includes proofs!!!

\section{Previous work}
%Quillen Wagoner, van der Kallen
\begin{theo}
    The canonical map $k_{i, n}^Q(R) \rightarrow k_{i, n+1}^Q(R)$ is surjective for $n \geq 2 i+\max (s . r . R-1,1)-1$ and bijective for $n \geq 2 i+\max (s . r . R-1,1)+1$. 
\end{theo}
A common feature in all these papers is the approach to stability problems for higher K-groups through stability for homology of linear groups.


\section{Suslin's work}
\cite{suslinStabilityAlgebraicKtheory1982} 

\begin{theo}
    Let $R$ be a ring, $r=s . r . R$. The canonical homomorphism $\quad k_{i, n}(R) \rightarrow k_{i, n+1}(R)$ is surjective for $n \geq r+i-1$ and bijective for $n \geq r+i$.    
    \end{theo}

 \begin{prop}
If $q \leq n-r$, then the differentials $d_{p q}^t$ are trivial for $t \geq 2$. Moreover $E_{p, q}^{\infty}=0$ for $0<q \leq n-r$.
 \end{prop}   

 \begin{coro}
    If $n \geq r+i$ then the action of $S t_n(R)$ and of $S_n$ on $K_{i, n}(R)$ is trivial.
 \end{coro}

\subsection{Homotopy fiber of Quillen's plus construction}

Suppose that $G$ is a group, $H$ a perfect normal subgroup and $B G \rightarrow \mathrm{BG}^{+}$Quillen's plus construction relative to $H$. Let $Y$ be the homotopy fiber of $B G \rightarrow B G^{+}$.

\begin{lemm}
    \begin{enumerate}
        \item a) $Y$ has the homotopy type of a CW-complex.
        b) $\quad Y$ is connected, $\pi_1(Y)$ is a universal central extension of the perfect group $H \quad($ see $[15]), \pi_j(Y)$ acts trivially on $\pi_i(Y)(i \geq 2)$.
        c) $\tilde{H}_{\star}(Y)=0$.
        d) $\quad \pi_i(Y)=\pi_{i+1}\left(B G^{+}\right)$for $i \geq 2$.
    \end{enumerate}
    These properties characterize $Y$ up to homotopy equivalence.
\end{lemm}


