
\chapter{Group (Cohomology) Theory} 

A \textbf{semigroup} is a nonempty set G together with a binary operation on G which is associative. A \textbf{monoid} is a semigroup G which contains a (two-sided) identity element. A \textbf{group} is a monoid G such that for every element there exists a (two-sided) inverse element.

\begin{theo}
Let G be a finitely generated abelian group.
    \begin{enumerate}
        \item There is a unique nonnegative integer s such that the number of infinite cyclic summands in any decomposition of G as a direct sum of cyclic groups is precisely s;
        \item either G is free abelian or there is a unique list of (not necessarily distinct) positive integers $\mathrm{m}_1, \ldots, \mathrm{m}_{\mathrm{t}}$ such that $\mathrm{m}_1>1, \mathrm{~m}_1\left|\mathrm{~m}_2\right| \cdots \mid \mathrm{m}_{\mathrm{t}}$ and
        $$
        \mathrm{G} \cong \mathbf{Z}_{m_1} \oplus \cdots \oplus \mathbf{Z}_{m_t} \oplus \mathrm{F}
        $$
        with F free abelian;
        \item either G is free abelian or there is a list of positive integers $\mathrm{p}_1^{s_1}, \ldots, \mathrm{p}_{\mathrm{k}}^{s_ k}$, which is unique except for the order of its members, such that $\mathrm{p}_1, \ldots, \mathrm{p}_{\mathrm{k}}$ are (not necessarily distinct) primes, $\mathrm{s}_1, \ldots, \mathrm{s}_{\mathrm{k}}$ are (not necessarily distinct) positive integers and
        $$
        \mathrm{G} \cong \mathrm{Z}_{\mathrm{p}_1} {\mathrm{s}_1} \oplus \ldots \oplus \mathrm{Z}_{\mathrm{p}_k} \mathrm{s}_k \oplus \mathrm{F}
        $$
        with F free abelian.
    \end{enumerate}
\end{theo}



\section{Actions}

\cite{brownCohomologyGroups1982} An action of a group G on a set S is a function $\mathrm{G} \times \mathrm{S} \rightarrow \mathrm{S}$ (usually denoted by $(\mathrm{g}, \mathrm{x}) \mapsto \mathrm{gx}$ ) such that for all $\mathrm{x} \in \mathrm{S}$ and $\mathrm{g}_1, \mathrm{~g}_2 \in \mathrm{G}$ :
$$
\mathrm{ex}=\mathrm{x} \quad \text { and } \quad\left(\mathrm{g}_1 \mathrm{~g}_2\right) \mathrm{x}=\mathrm{g}_1\left(\mathrm{~g}_2 \mathrm{x}\right) .
$$

When such an action is given, we say that G acts on the set S.  The {\bfseries orbit of $x\in X$} is $Gx = \{ gx \, | \, g\in G \}$ and its {\bfseries stabilizer} (or isotropy group) is $G_x = \{ g\in G \, | \, gx = x \}$.



\begin{theo}
    \begin{enumerate}
        \item Orbits have cardinality equal to the index of the corresponding stabilizer.
        \item The number of elements in the conjugacy class of $\mathrm{x} \in \mathrm{G}$ is $\left[\mathrm{G}: \mathrm{C}_{\mathrm{G}}(\mathrm{x})\right]$, which divides $|\mathrm{G}|$;
        \item (\textbf{Class equation}) If $\overline{\mathrm{x}}_1, \ldots, \overline{\mathrm{x}}_{\mathrm{n}}\left(\mathrm{x}_{\mathrm{i}} \in \mathrm{G}\right)$ are the distinct conjugacy classes of G , then $$
|\mathrm{G}|=\sum_{i=1}^n\left[\mathrm{G}: \mathrm{C}_{\mathrm{G}}\left(\mathrm{x}_{\mathrm{i}}\right)\right]
$$ In particular, we can take $G$ acting on itself by conjugation, so that the conjugacy classes are the orbits of this action.
\item The number of subgroups of G conjugate to K is $\left[\mathrm{G}: \mathrm{N}_{\mathrm{G}}(\mathrm{K})\right]$, which divides $|\mathrm{G}|$.
\end{enumerate}
\end{theo}

Let $G$ and $H$ be groups and $\theta: H \rightarrow$ Aut $G$ a homomorphism. Let $G \times_\theta H$ be the set $G \times H$ with the following binary operation: $(g, h)\left(g^{\prime}, h^{\prime}\right)=\left(g\left[\theta(h)\left(g^{\prime}\right)\right], h h^{\prime}\right)$. Show that $G \times_\theta H$ is a group with identity element $(e, e)$ and $(g, h)^{-1}=$ $\left(\theta\left(h^{-1}\right)\left(g^{-1}\right), h^{-1}\right) . G \times_\theta H$ is called the {\bfseries } of $G$ and $H$.

\paragraph*{Group rings}

Let $G$ be a (multiplicative) group. Let $\mathbb{Z} G$ be the free $\mathbb{Z}$-module generated by the elements of $G$. The multiplication in $G$ extends uniquely to a $\mathbb{Z}$-bilinear product $\mathbb{Z G} \times \mathbb{Z G} \rightarrow \mathbb{Z G}$; this makes $\mathbb{Z G}$ a ring, called the \textbf{(integral) group ring} of $G$.

Note that $G$ is a subgroup of the group $(\mathbb{Z} G)^*$ of units of $\mathbb{Z G}$ 

\begin{theo}[Universal property]
Given a ring $R$ and a group homomorphism $f: G \rightarrow R^*$, there is a unique extension of $f$ to a ring homomorphism $\mathbb{Z G} \rightarrow R$. Thus we have the "adjunction formula"
    $$
    \operatorname{Hom}_{\text {(rings) }}(\mathbb{Z} G, R) \approx \operatorname{Hom}_{\text {(groups) }}\left(G, R^*\right) .
    $$
\end{theo}

A \textbf{(left) $\mathbb{Z} G$-module}, or $G$-module, consists of an abelian group $A$ together with a homomorphism from $\mathbb{Z} G$ to the ring of endomorphisms of $A$. By the universal property, $G$-module is simply an abelian group $A$ together with an action of $G$ on $A$. For example, one has for any $A$ the trivial module structure, with $g a=a$ for $g \in G, a \in A$.

One way of constructing $G$-modules is by linearizing permutation representations. More precisely, if $X$ is a $G$-set (i.e., a set with $G$-action), then one forms the free abelian group $\mathbb{Z X}$ (also denoted $\mathbb{Z}[X]$ ) generated by $X$ and one extends the action of $G$ on $X$ to a $\mathbb{Z}$-linear action of $G$ on $\mathbb{Z} X$. The resulting $G$-module is called a permutation module. In particular, one has a permutation module $\mathbb{Z}[G / H]$ for every subgroup $H$ of $G$, where $G / H$ is the set of cosets $g H$ and $G$ acts on $G / H$ by left translation.

\begin{prop}
Let $X$ be a free $G$-set and let $E$ be a set of representatives for the $G$-orbits in $X$. Then $\mathbb{Z}X$ is a free $\mathbb{Z}G$-module with basis $E$.
\end{prop}


\section{Co-invariants}
If $G$ is a group and $M$ is a $G$-module, then the group of co-invariants of $M$, denoted $M_G$, is defined to be the quotient of $M$ by the additive subgroup generated by the elements of the form $g m-m\left(g \in G, m \in M\right.$ ). Thus $M_G$ is obtained from $M$ by "dividing out" by the $G$-action. (The name "co-invariants" comes from the fact that $M_G$ is the largest quotient of $M$ on which $G$ acts trivially, whereas $M^G$, the group of invariants, is the largest submodule of $M$ on which $G$ acts trivially.) In view of exercise $1 \mathrm{a}$ of $\$ I .2$, we can also describe $M_G$ as $M / I M$, where $I$ is the augmentation ideal of $\mathbb{Z} G$ and $I M$ denotes the set of all finite sums $\sum a_i b_i\left(a_i \in I, b_i \in M\right)$.
Still another description of $M_G$ is given by:
$$
M_G \approx \mathbb{Z} \otimes_{\mathbb{Z} G} M .
$$

Here, in order for the tensor product to make sense, we regard $\mathbb{Z}$ as a right $\mathbb{Z} G$-module (with trivial $G$-action). To prove 2.1 , note that in $\mathbb{Z} \otimes_{\mathbb{Z} G} M$ we have the identity $1 \otimes g m=1 \cdot g \otimes m=1 \otimes m ;$ hence there is a map $M_G \rightarrow$ $\mathbb{Z} \otimes_{\mathbb{Z} G} M$ given by $\bar{m} \mapsto 1 \otimes m$, where $\bar{m}$ denotes the image in $M_G$ of an element $m \in M$. On the other hand, using the universal property of the tensor product, we can define a map $\mathbb{Z} \otimes_{\mathbb{Z} G} M \rightarrow M_G$ by $a \otimes m \mapsto a \bar{m}$. These two maps are inverses of one another. $$\text{Also} M^G \simeq \operatorname{Hom}_G(\mathbb{Z},M) $$

In view of 2.1 and standard properties of the tensor product, we immediately obtain the following two properties of the co-invariants functor:

\begin{enumerate}
    \item Right-exactness: Given an exact sequence $M^{\prime} \rightarrow M \rightarrow M^{\prime \prime} \rightarrow 0$ of $G$-modules, the induced sequence $M_G^{\prime} \rightarrow M_G \rightarrow M_G^{\prime \prime} \rightarrow 0$ is exact.
    \item If $F$ is a free $\mathbb{Z} G$-module with basis $\left(e_i\right)$, then $F_G$ is a free $\mathbb{Z}$-module with basis $\left(\bar{e}_i\right)$.
\end{enumerate}

\begin{prop}
    
    Let $X$ be a free G-complex and let $Y$ be the orbit complex $X / G$. Then $C_*(Y) \approx C_*(X)_G$.
\end{prop}

%incluir más sobre teoria geometrica de grupos





\section{Cohomology}
References \cite{weibelIntroductionHomologicalAlgebra1994}.
Let $A$ be a $G$-module. We write $H_*(G ; A)$ for the left derived functors $L_*\left(-_G\right)(A)$ and call them the homology groups of $G$ with $c O$ efficients in $A ;$ by the lemma above, $$H_*(G ; A) \cong \operatorname{Tor}_*^{\mathbb{Z} G}(\mathbb{Z}, A)$$ By definition, $H_0(G ; A)=A_G$. Similarly, we write $H^*(G ; A)$ for the right derived functors $R^*\left({ }^G\right)(A)$ and call them the cohomology groups of $G$ with coefficients in $A ;$ by the lemma above, $$H^*(G ; A) \cong \operatorname{Ext}_{\mathbb{Z} G}^*(\mathbb{Z}, A)$$ By definition, $H^0(G ; A)=A^G$

\begin{example}
    \begin{enumerate}
        \item If $G=1$ is the trivial group, $A_G=A^G=A$. Since the higher derived functors of an exact functor vanish, $H_*(1 ; A)=H^*(1 ; A)=0$ for * $\neq 0$.
        \item Let $G$ be the infinite cyclic group $T$ with generator $t$. We may identify $\mathbb{Z} T$ with the Laurent polynomial ring $\mathbb{Z}\left[t, t^{-1}\right]$. Since the sequence $$
0 \rightarrow \mathbb{Z} T \xrightarrow{t-1} \mathbb{Z} T \rightarrow \mathbb{Z} \rightarrow 0
$$
is exact,
$$
\begin{aligned}
& H_n(T ; A)=H^n(T ; A)=0 \text { for } n \neq 0,1, \text { and } \\
& H_1(T ; A) \cong H^0(T ; A)=A^T, H^1(T ; A) \cong H_0(T ; A)=A_T
\end{aligned}
$$
In particular, $H_1(T ; \mathbb{Z})=H^1(T ; \mathbb{Z})=\mathbb{Z}$. We will see in the next section that all free groups display similar behavior, because $p d_G(\mathbb{Z})=1$.
    \end{enumerate}
\end{example}

The \textbf{augmentation ideal} of $\mathbb{Z} G$ is the kernel $\mathfrak{I}$ of the ring map $\mathbb{Z} G \xrightarrow{\epsilon} \mathbb{Z}$ which sends $\sum n_g g$ to $\sum n_g$. Because $\{1\} \cup\{g-1: g \in G$, $g \neq 1\}$ is a basis for $\mathbb{Z} G$ as a free $\mathbb{Z}$-module, it follows that $\mathfrak{I}$ is a free $\mathbb{Z}$ module with basis $\{g-1: g \in G, g \neq 1\}$.

\begin{example}
    \begin{enumerate}
        \item Since the trivial $G$-module $\mathbb{Z}$ is $\mathbb{Z} G / \mathfrak{J}, H_0(G ; A)=A_G$ is isomorphic to $\mathbb{Z} \otimes_{\mathbb{Z} G} A=\mathbb{Z} G / \mathcal{I} \otimes_{\mathbb{Z} G} A \cong A / \mathcal{I} A$ for every $G$-module $A$. For example, $H_0(G ; \mathbb{Z})=\mathbb{Z} / \mathfrak{I} \mathbb{Z}=\mathbb{Z}, H_0(G ; \mathbb{Z} G)=\mathbb{Z} G / \mathfrak{I} \cong \mathbb{Z}$, and $H_0(G ; \mathfrak{I})=$ $\mathfrak{I} / \mathfrak{I}^2$

        \item Because $\mathbb{Z} G$ is a projective object in $\mathbb{Z} G$-mod, $H_*(G ; \mathbb{Z} G)=0$ for $* \neq 0$ and $H_0(G ; \mathbb{Z} G)=\mathbb{Z}$. When $G$ is a finite group, Shapiro's Lemma (6.3.2 below) implies that $H^*(G ; \mathbb{Z} G)=0$ for $* \neq 0$. This fails when $G$ is infinite; for example, we saw in 6.1 .4 that $H^1(T ; \mathbb{Z} T) \cong \mathbb{Z}$ for the infinite cyclic group $T$. If $G$ is finite, then $H^0(G ; \mathbb{Z} G) \cong \mathbb{Z}$, but $H^0(G ; \mathbb{Z} G)=0$ if $G$ is infinite.
    \end{enumerate}
\end{example}

% the norm element is used for cohomology of finite gorups (6.1.8 Weibel)

\begin{theo}[$H_1$]
For any group $G, H_1(G ; \mathbb{Z}) \cong \mathfrak{I} / \mathfrak{J}^2 \cong G /[G, G]$.   
\end{theo}

\begin{theo}[Trivial $G-$module]
If $A$ is any trivial $G$-module, $H_0(G ; A) \cong A, H_1(G ; A) \cong$ $G /[G, G] \otimes_{\mathbb{Z}} A$, and for $n \geq 2$ there are (noncanonical) isomorphisms:

$$
H_n(G ; A) \cong H_n(G ; \mathbb{Z}) \otimes_{\mathbb{Z}} A \oplus \operatorname{Tor}_1^{\mathbb{Z}}\left(H_{n-1}(G ; \mathbb{Z}), A\right)
$$
\end{theo}


\subsubsection*{Spectral sequence}

If $A_*$ is a chain complex of $G$-modules, the hyperderived functors $\mathbb{L}_i(-G)\left(A_*\right)$ of 5.7.4 are written as $\mathbb{H}_i\left(G ; A_*\right)$ and called the hyperhomology groups of $G$. Similarly, if $A^*$ is a cochain complex of $G$ modules, the hypercohomology groups $\mathbb{H}^i\left(G ; A^*\right)$ are just the hyper-derived functors $\mathbb{R}^i\left(-{ }^G\right)\left(A^*\right)$. The generalities of Chapter 5 , section 7 become the following facts in this case. The hyperhomology spectral sequences are

$$
\begin{aligned}
& { }^{I I} E_{p q}^2=H_p\left(G ; H_q\left(A_*\right)\right) \Rightarrow \mathbb{H}_{p+q}\left(G ; A_*\right) ; \text { and } \\
& { }^I E_{p q}^2=H_p\left(H_q\left(G ; A_*\right)\right) \Rightarrow \mathbb{H}_{p+q}\left(G ; A_*\right) \text { when } A_* \text { is bounded below, }
\end{aligned}
$$

In particular, suppose that $A$ is bounded below. If each $A_i$ is a flat $\mathbb{Z} G$-module, then $\mathbb{H}_i\left(G ; A_*\right)=H_i\left(\left(A_*\right)_G\right)$; if each $A^i$ is a projective $\mathbb{Z} G$-module, then $H^i\left(G ; A^*\right)=H^i\left(\left(A^*\right)^G\right)$.


\section{Cyclic and Free Groups Cohomology}

\begin{theo}
    If $A$ is a module for the cyclic group $G=C_m$, then

    $$
    \begin{aligned}
    & H_n\left(C_m ; A\right)=\left\{\begin{array}{ll}
    A /(\sigma-1) A & \text { if } n=0 \\
    A^G / N A & \text { if } n=1,3,5,7, \ldots\} ; \\
    \{a \in A: N a=0\} /(\sigma-1) A & \text { if } n=2,4,6,8, \ldots
    \end{array}\right\} ; \\
    & H^n\left(C_m ; A\right)=\left\{\begin{array}{ll}
    A^G & \text { if } n=0 \\
    \{a \in A: N a=0\} /(\sigma-1) A & \text { if } n=1,3,5,7, \ldots \\
    A^G / N A & \text { if } n=2,4,6,8, \ldots
    \end{array}\right\} .
    \end{aligned}
    $$    
\end{theo}

$$\begin{aligned} H_n\left(C_m ; \mathbb{Z}\right) & =\left\{\begin{array}{ll}\mathbb{Z} & \text { if } n=0 \\ \mathbb{Z} / m & \text { if } n=1,3,5,7, \ldots \\ 0 & \text { if } n=2,4,6,8, \ldots\end{array}\right\} ; \\ H^n\left(C_m ; \mathbb{Z}\right) & =\left\{\begin{array}{ll}\mathbb{Z} & \text { if } n=0 \\ 0 & \text { if } n=1,3,5,7, \ldots \\ \mathbb{Z} / m & \text { if } n=2,4,6,8, \ldots\end{array}\right\}\end{aligned}$$

\textbf{Tate Cohomology.} Taking full advantage of this periodicity, we set

$$
\hat{H}^n\left(C_m ; A\right)=\left\{\begin{array}{ll}
A^G / N A & \text { if } n \in \mathbb{Z} \text { is even } \\
\{a \in A: N A=0\} /(\sigma-1) A & \text { if } n \in \mathbb{Z} \text { is odd }
\end{array}\right\}
$$


More generally, if $G$ is a finite group and $A$ is a $G$-module, we define the Tate cohomology groups of $G$ to be the groups

$$
\hat{H}^n(G ; A)=\left\{\begin{array}{ll}
H^n(G ; A) & \text { if } n \geq 1 \\
A^G / N A & \text { if } n=0 \\
\{a \in A: N a=0\} / \mathcal{J} A & \text { if } n=-1 \\
H_{1-n}(G ; A) & \text { if } n \leq-2
\end{array}\right\}
$$

\begin{example}
[Dimension-shifting] Given a $G$-module $A$ ($G$ finite), choose a short exact sequence $0 \rightarrow K \rightarrow P \rightarrow A \rightarrow 0$ with $P$ projective. Shapiro's Lemma (below) implies that $\hat{H}^*(G, P)=0$ for all $* \in \mathbb{Z}$. Therefore $\hat{H}^n(G ; A) \cong$ $\hat{H}^{n+1}(G ; K)$. This shows that every Tate cohomology group $\hat{H}^n(G ; A)$ determines the entire theory.    
\end{example}

\begin{prop}
Let $G$ be the free group on the set $X$. Then the augmentation ideal $\mathfrak{I}$ is a free $\mathbb{Z} G$-module with basis the set $X-1=\{x-1: x \in X\}$. Also, $0 \rightarrow \mathfrak{I} \rightarrow \mathbb{Z} G \rightarrow \mathbb{Z} \rightarrow 0$ is a free resolution of $\mathbb{Z}$. Consequently, $\operatorname{pd}_G(\mathbb{Z})=1$, that is, $H_n(G ; A)=H^n(G ; A)=0$ for $n \neq 0,1$. Moreover, $H_0(G ; \mathbb{Z}) \cong H^0(G ; \mathbb{Z}) \cong \mathbb{Z}$, while

$$
H_1(G ; \mathbb{Z}) \cong \bigoplus_{x \in X} \mathbb{Z} \quad \text { and } \quad H^1(G ; \mathbb{Z}) \cong \prod_{x \in X} \mathbb{Z}
$$
Stallings and Swan proved the converse!
\end{prop}

\section{Calculations with Shapiro's Lemma}

If $H$ is a subgroups of $G$, $\mathbb{Z} G \otimes_{\mathbb{Z} H} A$ is called the \textbf{induced $G$-module} and is written $\operatorname{Ind}_H^G(A)$. Similarly, $\operatorname{Hom}_H(\mathbb{Z} G, A)$ is called the \textbf{coinduced $G$-module} and is written $\operatorname{Coind}_H^G(A)$.

\begin{theo}[Shapiro's Lemma] Let $H$ be a subgroup of $G$ and $A$ an $H$-module. Then

    $$
    H_*\left(G ; \operatorname{Ind}_H^G(A)\right) \cong H_*(H ; A) ; \text { and } H^*\left(G ; \operatorname{Coind}_H^G(A)\right) \cong H^*(H ; A)
    $$       
\end{theo}

\begin{coro}
    \begin{enumerate}
        \item If $A$ is an abelian group, then
        $$
        H_*\left(G ; \mathbb{Z} G \otimes_{\mathbb{Z}} A\right)=H^*\left(G ; \operatorname{Hom}_{\mathrm{Ab}}(\mathbb{Z} G, A)\right)=\left\{\begin{array}{ll}
        A & \text { if } *=0 \\
        0 & \text { if } * \neq 0
        \end{array}\right\}
        $$
    \item If $G$ is a finite group, then $H^*\left(G ; \mathbb{Z} G \otimes_{\mathbb{Z}} A\right)=0$ for $* \neq 0$ and all $A$. 
    \item If $G$ is finite and $P$ is a projective $G$ module,

    $$
    \widehat{H}^*(G ; P)=0 \text { for all } * .
    $$
    
    \end{enumerate}
\end{coro}

\begin{theo}[Hilbert 90, additive version]
Let $K \subset L$ be a finite Galois extension of fields, with Galois group $G$. Then $L$ is a $G$-module, $L^G \cong L_G \cong$ $K$, and
    $$
    H^*(G ; L)=H_*(G ; L)=0 \text { for } * \neq 0
    $$ %review normal basis theorem
\end{theo}

\begin{example}[Cyclic Galois extensions]
Suppose that $G$ is cyclic of order $m$, generated by $\sigma$. The trace $\operatorname{tr}(x)$ of an element $x \in L$ is the element $x+$ $\sigma x+\cdots+\sigma^{m-1} x$ of $K$. In this case, Hilbert's Theorem 90 states that there is an exact sequence

    $$
    0 \rightarrow K \rightarrow L \xrightarrow{\sigma-1} L \xrightarrow{t r} K \rightarrow 0 .
    $$
    
    
    Indeed, we saw in the last section that for $* \neq 0$ every group $H_*(G ; L)$ and $H^*(G ; L)$ is either $K / \operatorname{tr}(L)$ or $\operatorname{ker}(\operatorname{tr}) /(\sigma-1) K$.
    
    As an application, suppose that $\operatorname{char}(K)=p$ and that $[L: K]=p$. Since $\operatorname{tr}(1)=p \cdot 1=0$, there is an $x \in L$ such that $(\sigma-1) x=1$, that is, $\sigma x=$ $x+1$. Hence $L=K(x)$ and $x^p-x \in K$ because
    
    $$
    \sigma\left(x^p-x\right)=(x+1)^p-(x+1)=x^p-x
    $$
% Noether theorem (generalize example when G not cyclic, Brauer Groups) / transfer homomorphism 
\end{example}
%cual es el orgien de las derivaciones?? ver introduccion libro Brown 

%incluir bar resolution en teoria geometrica de grupos. Tambien inclui Weibel 6.10


%antes, podria colocar el teorema de Hopf... pero Brown lo hace mas facil a partir de la interpretacion topologica

\section{Universal Central Extensions}
%comparar con Srinivas

A \textbf{central extension} of $G$ is an extension $0 \rightarrow A \rightarrow X \xrightarrow{\pi} G \rightarrow 1$ such that $A$ is in the center of $X$. (If $\pi$ and $A$ are clear from the context, we will just say that $X$ is a central extension of $G$.) A homomorphism over $G$ from $X$ to another central extension $0 \rightarrow B \rightarrow Y \xrightarrow{\tau} G \rightarrow 1$ of $G$ is a map $f: X \rightarrow$ $Y$ such that $\pi=\tau f .\; X$ is called a universal central extension of $G$ if for every central extension $0 \rightarrow B \rightarrow Y \xrightarrow{\tau} G \rightarrow 1$ of $G$ there exists a unique homomorphism $f$ from $X$ to $Y$ over $G$.

Clearly, a universal central extension is unique up to isomorphism over $G$, provided that it exists.\\
A group $G$ is \textbf{perfect} if it equals its commutator gorup $[G, G]$, or equivalently, if $H_1(G ; \mathbb{Z})=0$.\\



\begin{prop}
\begin{enumerate}
    \item Universal central extensions of perfect groups are perfect.
    \item If $0 \rightarrow A \rightarrow X \rightarrow G \rightarrow 1$ is any central extension in which $G$ and $X$ are perfect groups, show that $H_1(X ; \mathbb{Z})=0$ and that there is an exact sequence
    $$
    \mathrm{H}_2(X ; \mathbb{Z}) \xrightarrow{\text { cor }} H_2(G ; \mathbb{Z}) \rightarrow A \rightarrow 0
    $$
    \item Show that if $G$ is perfect then central extensions $0 \rightarrow A \rightarrow X \rightarrow G \rightarrow 1$ are classified by $\operatorname{Hom}\left(\mathrm{H}_2(G ; \mathbb{Z}), A\right)$. 
\end{enumerate}
\end{prop}

\begin{theo} %utilizes Hopf's theorem
A group $G$ has a universal central extension if and only if $G$ is perfect. In this case, the universal central extension is
 $$0 \rightarrow \mathrm{H}_2(G ; \mathbb{Z}) \rightarrow \frac{[F, F]}{[R, F]} \xrightarrow{\pi} G \rightarrow 1.$$
Here $1 \rightarrow R \rightarrow F \rightarrow G \rightarrow 1$ is any presentation of $G$.
\end{theo}

\begin{prop}[Recognition Criterion] A central extension $0 \rightarrow A \rightarrow X \xrightarrow{\pi} G \rightarrow 1$ is universal if and only if $X$ is perfect and every central extension of $X$ splits as a direct product of $X$ with an abelian group. 
\end{prop}

\begin{coro}
    \begin{enumerate}
        \item If $0 \rightarrow A \rightarrow X \rightarrow G \rightarrow 1$ is a universal central extension, then

        $$
        H_1(X ; \mathbb{Z})=H_2(X ; \mathbb{Z})=0 .
        $$
        \item If $G$ is a perfect group and $H_2(G ; \mathbb{Z})=0$, then every central extension of $G$ is a direct product of $G$ with an abelian group.
    \end{enumerate}
\end{coro}


\begin{example}
    The smallest perfect group is $A_5$. The universal central extension of $A_5$ describes $A_5$ as the quotient $P S L_2\left(\mathbb{F}_5\right)$ of the binary icosahedral group $X=S L_2\left(\mathbb{F}_5\right)$ by the center of order $2, A= \pm\left({ }_{01}^{(0)}\right.$ ) [Suz, 2.9].
    
        $$
        0 \longrightarrow \mathbb{Z} / 2 \xrightarrow{\left(\begin{array}{cc}
        -1 & 0 \\
        0-1
        \end{array}\right)} S L_2\left(\mathbb{F}_5\right) \longrightarrow P S L_2\left(\mathbb{F}_5\right) \longrightarrow 1
        $$     
        Example 6.9.10 (Alternating groups) It is well known that the alternating groups $A_n$ are perfect if $n \geq 5$. From [Suz, 3.2] we see that

        $$
        H_2\left(A_n ; \mathbb{Z}\right) \cong\left\{\begin{array}{ll}
        \mathbb{Z} / 6 & \text { if } n=6,7 \\
        \mathbb{Z} / 2 & \text { if } n=4,5 \text { or } n \geq 8 \\
        0 & \text { if } n=1,2,3
        \end{array}\right\}
        $$
        
        
        We have already mentioned (6.9.1) the universal central extension of $A_5$. In general, the regular representation $A_n \rightarrow S O_{n-1}$ gives rise to a central extension
        
        $$
        0 \rightarrow \mathbb{Z} / 2 \rightarrow \widetilde{A}_n \rightarrow A_n \rightarrow 1
        $$
        
        by restricting the central extension
        
        $$
        0 \rightarrow \mathbb{Z} / 2 \rightarrow \operatorname{Spin}_{n-1}(\mathbb{R}) \rightarrow S O_{n-1} \rightarrow 1
        $$
        
        
        If $n \neq 6,7, \tilde{A}_n$ must be the universal central extension of $A_n$.

    \end{example}

    \begin{example}
It is known [Suz, 1.9] that if $F$ is a field, then the special linear group $S L_n(F)$ is perfect, with the exception of $S L_2\left(\mathbb{F}_2\right) \cong D_6$ and $S L_2\left(\mathbb{F}_3\right)$, which is a group of order 24 . The center of $S L_n(F)$ is the group $\mu_n(F)$ of $n^{t h}$ roots of unity in $F$ (times the identity matrix $I$ ), and the quotient of $S L_n(F)$ by $\mu_n(F)$ is the projective special linear group $P S L_n(F)$.  
When $F=\mathbb{F}_q$ is a finite field, we know that $H_2\left(S L_n\left(\mathbb{F}_q\right) ; \mathbb{Z}\right)=0$ [Suz, 2.9]. It follows, again with two exceptions, that

$$
0 \rightarrow \mu_n\left(\mathbb{F}_q\right) \xrightarrow{I} S L_n\left(\mathbb{F}_q\right) \rightarrow P S L_n\left(\mathbb{F}_q\right) \rightarrow 1
$$

is the universal central extension of the finite group $P S L_n\left(\mathbb{F}_q\right)$.
    \end{example}






\section{An spectral sequence for group cohomology}
% en realidad es mayer vietoris generalizado, podemos aplicar induccion para el caso de productos amalgamado de dos subgrupos, esta en Brown II.7 
%existe una teoria mas general, podemos deducirlo de la secuencia espectral de mayer vietoris??
Suppose that $X$ is a simplicial set and $x_i$ are simplicial subsets such that $X=U X_i$. Then, setting $X_{i j}=X_i \cap X_j$ (etc.) we'11 obviously have for the realisations: $|x|=U\left|x_i\right|,\left|x_i\right| \cap\left|x_j\right|=\left|x_{i j}\right|, \ldots$ Let's suppose that the set of indices is linearly ordered. Consider the following bicomplex:
$$ K = \longrightarrow \underset{i<j<k}{\oplus} C_*\left(x_{i j k}\right) \longrightarrow \underset{i<j}{\oplus} C_*\left(x_{i j}\right)\longrightarrow \underset{i}{\oplus} C_*\left(x_{i}\right) $$


Here by a bicomplex we understand a bicomplex in the sense of Grothendieck [9] i.e. the differentials $d_1$ and $d_2$ commute. (The sign in this approach appears in the definition of the total differentials). The vertical arrows of the bicomplex map $C_*\left(x_i \cdots_i\right)$ into $\underset{k=0}{q} C_*\left(x_{i_0} \ldots \hat{i}_k \ldots i_q\right)$, the mapping into the kth summand differing $k=0$ by a sign $(-1)^k \quad$ from the natural embedding.

The first spectral sequence of this bicomplex degenerates and yields an isomorphism $H_{\star}(K) \cong H_{\star}(X)$. (Moreover this isomorphism is induced by the canonical map $K \rightarrow C_*(X)$). The second spectral sequence gives us a functorial spectral sequence of the first quadrant, whose limit equals $H_*(X)$, while its differential $d r$ has bidegree $(r-1,-r)$ and its $E^1$-term looks as follows: $$E_{p q q}^1=\underset{i_0<\ldots<i_q}{\otimes} H_p\left(x_{i_0} \ldots i_q\right)$$

Suppose $G$ is a group. Let $X_G$ denote the simplicial set (and its geometric realisation), whose p-simplices are sequences $\left(g_0, \ldots, g_p\right)$ of elements of $G$, with the usual faces and degeneracies. This space $X_G$ is contractible by (1.2). The group $G$ acts from the right on $X_G$ and this action is obviously free, hence $B G=X_G / G$ is a classifying space of $G$. The complex $C_*(B G)=C_*(G)$ coincides with the usual complex associated with $G$. Moreover $C_*(G)=C_*\left(X_G\right) \otimes_G Z$.

If $H$ is a subgroup of $G$, then $X_G / H$ is a classifying space for $H$ and hence $B H=X_H / H \rightarrow X_G / H$ is a homotopy equivalence. In particular $C_*(H)+C_*\left(X_G\right) \otimes_H \mathbb{Z}=C_*\left(X_G\right) \otimes_G Z|G / H|$ is a homotopy equivalence.

(2.3) The spectral sequence associated with a family of subgroups.

Suppose $G$ is a group and $G_1, \ldots, G_n$ are subgroups. Then $B G_i$ may be viewed as a simplicial subset of $B G$ and $B G_i \cap B G_j=B\left(G_i \cap G_j\right)$.. Denote $U B G_i$ by $X$ and consider the spectral sequence of the covering $X=U B G_i$. Along with the bicomplex $K$ introduced in (2.1) we also consider the following bicomplex:

$$K' = \underset{i<j<k}{\oplus} C_*\left(X_G\right) \otimes_G Z\left[G / G_{i j k}\right] \longrightarrow \underset{i<j}{\oplus} C_*\left(X_G\right) \otimes_G Z\left[G / G_{i j}\right] \longrightarrow \underset{i<j}{\oplus} C_*\left(X_G\right) \otimes_G Z\left[G / G_{i}\right] $$

There is a natural mapping of bicomplexes $K+K^{\prime}$ and because of (2.2) this mapping induces an isomorphism of second spectral sequences so that $H_{\star}(X)=H_{\star}(K)=H_*\left(K^{\prime}\right)$. The first spectral sequence of $K^{\prime}$ looks as follows: $E_{*, q}^1=C_*\left(X_G\right) \otimes_G H_q(L)$, where $L$ is the following complex of left G-modules:
$$
\oplus \mathbb{Z}\left[G / G_i\right]+\oplus \mathbb{Z}\left[G / G_{i j}\right]+\oplus \mathbb{Z}\left[G / G_{i j k}\right]+\ldots
$$


\begin{prop}
If $G_1, \ldots, G_n$ are subgroups of $G$, there exists a fuctorial spectral sequence of the first quadrart, the $E^2$ term of which looks like: $E_{p q}^2=H_p\left(G, H_q(L)\right)$, where $L$ is the complex defined above. It converges to $H_{\star}\left(U B G_j\right)$ and the differential $d^r$ has bidegree $(-r, r-1)$.   
\end{prop}





\chapter{Rings (with identity)}

Let $R$ be a ring and $S$ a nonempty subset of $R$ that is closed under the operations of addition and multiplication in $R$ . If $S$ is itself a ring under these operations then $S$ is called a subring of $R$ . A subring I of a ring R is a \textbf{left ideal} provided that
$
\mathrm{r} \in \mathrm{R} \text { and } \mathrm{x} \in \mathrm{I}$ implies $  \mathrm{rx} \in \mathrm{I} 
$.
I is a \textbf{right ideal} provided
$
\mathrm{r} \in \mathrm{R} \text { and } \mathrm{x} \in \mathrm{I}$ implies $  \mathrm{xr} \in \mathrm{I}
$.

I is an \textbf{ideal} if it is both a left and right ideal. Note that proper ideals does not contain any unit. We denote by $(X)$ the ideal generated by the subset $X$ of R , i.e., the smallest ideal containing X.
\begin{theo}
    \begin{enumerate}
        \item (a) $=\left\{\sum_{i=1}^n \mathrm{r}_{\mathrm{j}} \mathrm{as}_{\mathrm{i}} \mid \mathrm{r}_{\mathrm{i}}, \mathrm{s}_{\mathrm{i}} \in \mathrm{R} ; n \in \mathbf{N}^*\right\}$ (principal ideal).
        \item If a is in the center of R , then $\mathrm{Ra}=(\mathrm{a})=\mathrm{aR}$.
        \item If X is in the center of R , then the ideal $(\mathrm{X})$ consists of all finite sums $$\mathrm{r}_1 \mathrm{a}_1+\cdots+\mathrm{r}_{\mathrm{n}} \mathrm{a}_{\mathrm{n}}\left(\mathrm{n} \in \mathbf{N}^* ; \mathrm{r}_{\mathrm{i}} \in \mathrm{R} ; \mathrm{a}_{\mathrm{i}} \in \mathrm{X}\right).$$
        \item For ideals, multiplication and addition are distributive and associative.
    \end{enumerate}
\end{theo}


An ideal P in a ring R is said to be prime if $\mathrm{P} \neq \mathrm{R}$ and for any ideals $\mathrm{A}, \mathrm{B}$ in R
$$
\mathrm{AB} \subset \mathrm{P} \Rightarrow \mathrm{A} \subset \mathrm{P} \text { or } \mathrm{B} \subset \mathrm{P} \text {. }
$$
\begin{theo}
If P is an ideal in a ring R such that $\mathrm{P} \neq \mathrm{R}$ and for all $\mathrm{a}, \mathrm{b} \in \mathbf{R}$
$$
\mathrm{ab} \in \mathrm{P} \Rightarrow \mathrm{a} \in \mathrm{P} \text { or } \mathrm{b} \in \mathrm{P} \text {, }
$$
then P is prime. Conversely if P is prime and R is commutative, then P satisfies condition (1).
\end{theo}

An ideal [resp. left ideal] M in a ring R is said to be \textbf{maximal} if $\mathbf{M} \neq \mathrm{R}$ and for every ideal [resp. left ideal] $\mathbf{N}$ such that $\mathbf{M} \subset \mathbf{N} \subset \mathrm{R}$, either $\mathbf{N}=\mathbf{M}$ $\operatorname{or} \mathbf{N}=\mathbf{R}$.

\begin{theo}
    \begin{enumerate}
        \item In a nonzero ring R with identity maximal [left] ideals always exist. In fact every $[l e f t]$ ideal in R (except R itself) is contained in a maximal [left] ideal.
        \item (In general, for $R^2 = R$) Every maximal ideal is prime. 
        \item If $M$ is maximal and $R$ is conmutative, then the $R/M$ is a field. the converse is true in general, even when $R/M$ is noncommutative.
        \item $R$ is a field if and only if the $(0)$ is a maximal ideal.
    \end{enumerate}
\end{theo}

%\subsection*{Localization and tensor products}

%A nonempty subset S of a ring R is multiplicative provided that $ a, b \in S \Rightarrow a b \in S .$




\section{Modules}

Every module we consider are unitary, i.e., the action of the neutral multiplicative element is trivial on the module

Let $I$ be a left ideal of the ring $R, A$ an $R$-module and $S$ a nonempty subset of $A$. Then $$I S=\left\{\sum_{i=1}^n r_i a_i \mid r_i \in I ; a_i \in S ; n \in \mathbf{N}^*\right\}$$ is a submodule of $A$. Similarly if $a \in A$, then $I a=\{r a \mid r \in I\}$ is a submodule of $A$. 

If X is a subset of a module A over a ring R , then the intersection of all submodules of A containing X is called the submodule generated by X (or spanned by X). We have $$(A)=R X=\left\{\sum_{i=1}^s r_i a_i \mid s \in N^* ; a_i \in X ; r_i \in R\right\}$$

\begin{theo}[Free-modules] Let $\mathbf{R}$ be a ring with identity. The following conditions on a unitary R -module F are equivalent:
    \begin{enumerate}
        \item F has a nonempty basis;
        \item F is the internal direct sum of a family of cyclic R -modules, each of which is isomorphic as a left $\mathbf{R}$-module to $\mathbf{R}$;
        \item F is R -module isomorphic to a direct sum of copies of the left R -module R ;
        \item There exists a nonempty set X and a function $\iota: \mathrm{X} \rightarrow \mathrm{F}$ with the following property: given any unitary R -module A and function $\mathrm{f}: \mathrm{X} \rightarrow \mathrm{A}$, there exists a unique R -module homomorphism $\overline{\mathrm{f}}: \mathrm{F} \rightarrow \mathrm{A}$ such that $\overline{\mathrm{f}} \iota=\mathrm{f}$. In other words, F is a free object in the category of unitary R -modules.
    \end{enumerate}
\end{theo}

\begin{theo}
        Let R be a ring with identity and F a free R-module with an infinite basis X. Then every basis of F has the same cardinality as X.
\end{theo}

Let R be a ring with identity such that for every free R-module F, any two bases of F have the same cardinality. Then R is said to have the \textbf{invariant dimension property} and the cardinal number of any basis of F is called the dimension (or rank) of F over R. 

Note that, in this case, two free modules are isomorphic if and only if they have the same rank. Also, $R$ does not satisfy the invariant dimension property iff $R^n \simeq R^m$ for $n\neq m$ \\

\begin{theo} For ring with identity:
    \begin{enumerate}
        \item Every lineary independent subset of a vector space over a division ring can be extended to a basis.
        \item Every division ring has the invariant dimension property.
        \item Any finite-dimensional algebra over a division ring has the invariant dimension property.
        \item Every conmutative ring (and so group rings) has the invariant dimension property.
    \end{enumerate}
    \end{theo}

\begin{example}
    $\operatorname{End}_F(F^\infty)$ does not have the invariant dimension property.
\end{example}


If R is a ring, for each $\mathrm{n}>0$, $\operatorname{Mat}_n R$ is a ring. We denote the identity matrix by $I_n$.

If M has a fixed ordered basis, we call M a based free module and define the {\bfseries rank} of the based free module M to be the cardinality of its given basis. If $R$ has the invariant dimension property, then the rank of a free module is constant.
\begin{theo}
    Homomorphisms between based free modules are naturally identified with
    matrices over R.
\end{theo}

An $R$-module $P$ is called {\bfseries stably free} if $P \oplus R^m \cong R^n$ for some $m$ and $n$.\\
By the fundamental theorem of Linear algebra, the kernel of any surjective linear map $\sigma: R^n \rightarrow R^m$ is a stably free module (split on finite dimension).

\begin{lemm}
    \begin{enumerate}
    \item If $P\oplus R^m \simeq R ^ \infty$ then $P \simeq R^\infty$.
    \item Free and stably free modules are projective.
    \end{enumerate}
\end{lemm}

When are stably free modules free? The most important special case, at least for inductive purposes, is when $m=1$, i.e., $P \oplus R \cong R^n$.

A row vector $\sigma = (r_1, \ldots, r_n)$ in $R$ is called \textbf{unimodular} if they generate $R^n$. The following condition are equivalent:
\begin{itemize}
    \item $\sigma$ is unimodular.
    \item $R^n \cong P \oplus R$, where $\sigma$ is identified with the projection $R^n \rightarrow R$ and $P=\operatorname{ker}(\sigma)$.
    \item $1=r_1 s_1+\cdots+r_n s_n$ for some $s_i \in R$.
\end{itemize}
Note that $P$ is a free module if and only if $\sigma$ may be completed to an invertible transformation.

We say that a ring $R$ satisfies condition $(S_n)$ (\bf{stable base condition on dimension $n$}) if for every unimodular row $\left(r_0, r_1, \ldots, r_n\right)$ in $R^{n+1}$, there is a unimodular vector $\left(r_1^{\prime}, \ldots, r_n^{\prime}\right)$ in $R^n$ with $r_i^{\prime}=r_i-r_0 t_i$ for some $t_1, \ldots, t_n$ in $R$.\\
The {\bfseries stable range} of $R, \operatorname{sr}(R)$, is defined to be the smallest $n$ such that $R$ satisfies condition $\left(S_n\right)$.

\begin{prop}
    \begin{enumerate}
        \item (Vaserstein) $\left(S_n\right)$ holds for all $n \geq \operatorname{sr}(R)$.
        \item If $\operatorname{sr}(R)=n$ then all stably free projective modules of rank $\geq n$ are free.
        \item $\operatorname{sr}(R)=1$ for every artinian ring $R$, and stably free projective modules are free over artinian rings.
        \item If $I$ is an ideal of $R$, then $\operatorname{sr}(R) \geq \operatorname{sr}(R / I)$.
        \item (Veldkamp) If $\operatorname{sr}(R)=n$ for some $n$ then $R$ has the invariant dimension property.
    \end{enumerate}
\end{prop}

We will focus most of our attention on the category $\mathbf{P}(R)$ of finitely generated projective $R$-modules; the morphisms are the $R$-module maps. Since the direct sum of projectives is projective, $\mathbf{P}(R)$ is an additive category. We may regard $\mathbf{P}$ as a covariant functor on rings, since if $R \rightarrow S$ is a ring map, then up to coherence there is an additive functor $\mathbf{P}(R) \rightarrow \mathbf{P}(S)$ sending $P$ to $P \otimes_R S$. (Formally, there is an additive functor $\mathbf{P}^{\prime}(R) \rightarrow \mathbf{P}(S)$ and an equivalence $\mathbf{P}^{\prime}(R) \rightarrow \mathbf{P}(R)$; see Ex. 2.16.)

\begin{lemm}[Some properties of projective modules]
 \begin{enumerate}
    \item A module is projective [generated by $n$ elements] if and only if it is a direct summand of a free module [of rank $n$].
    \item Every finitely generated projective R-module arises from an idempotent element in a matrix ring $M_n(R)$.
    \item If $R$ is a principal ideal domain, then every projective module is free.
If $R$ is a local ring, then every finitely generated projective $R$-module $P$ is free of rank $\operatorname{dim}_{R / \mathrm{m}}(P / \mathrm{m} P)$. 
 \end{enumerate}   
\end{lemm}


\subsubsection{Some results for conmutative rings}

\begin{theo}[Bass' Cancelation Theorem]
    If $R$ is a conmutative noetherian ring of Krull dimension d, or more generally, if $\operatorname{SpecMax} (R)$ is a finite union of spaces of dimension $\leq d$, then $sr(R) \leq d+1$
\end{theo}

\begin{theo}
Let $R$ be a $d-$dimensional commutative noetherian ring. Then every stably free $R$-module of rank $>d$ is a free module. Equivalently, every unimodular row of length $n \geq d+2$ may be completed to an invertible matrix.
\end{theo}

Let $R$ be a commutative ring. The {\bfseries rank of a finitely generated $R$-module $M$ at a prime ideal $p$}, $\operatorname{rank}_\mathfrak{p}(M)$ is the minimal number of generators of $M_p$.

\begin{lemm}
 Let $R$ be a commutative ring, $P$ a finitely generated projective $R$-module, the functions $\operatorname{rank} P: \operatorname{Spec}(R) \rightarrow \mathbb{N}, \mathbb{Z}$ are continuous.
\end{lemm}

If a module $M$ is not projective, $\operatorname{rank}(M)$ need not be a continuous function on $\operatorname{Spec}(R)$, as the example $R=\mathbb{Z}, M=\mathbb{Z} / p$ shows.

We say that $P$ has constant $\operatorname{rank} n$ if $n=\operatorname{rank}_p(P)$ is independent of $p$. \\
If $\operatorname{Spec}(R)$ is topologically connected, every finitely generated projective $R$-module has constant rank.

We say that two $R$-modules $M, M^{\prime}$ are {\bfseries stably isomorphic} if $M \oplus R^m \cong M^{\prime} \oplus R^m$ for some $m \geq 0$.

\begin{theo}[Bass-Serre cancellation]
    Theorem 2.3 (Bass-Serre cancellation). Let $R$ be a $d-$dimensional commutative noetherian ring, and let $P$ be a projective $R$-module of constant rank $n>d$.
    \begin{enumerate}
        \item (Serre) $P \cong P_0 \oplus R^{n-d}$ for some projective $R$-module $P_0$ of constant rank d.
        \item (Bass) If $P$ is stably isomorphic to $P^{\prime}$, then $P \cong P^{\prime}$.
        \item (Bass) For all $M, M^{\prime}$, if $P \oplus M$ is stably isomorphic to $M^{\prime}$, then $P \oplus M \cong M^{\prime}$.
    \end{enumerate} 
\end{theo}

\subsubsection{Milnor squares}

It is sometimes useful to be able to build projective modules by patching free modules. The following data suffices. Suppose that $s_1, \ldots, s_c \in R$ form a unimodular row, i.e., $s_1 R+\cdots+s_c R=R$. Then $\operatorname{Spec}(R)$ is covered by the open sets $D\left(s_i\right) \cong \operatorname{Spec}\left(R\left[\frac{1}{s_i}\right]\right)$. Suppose we are given $g_{i j} \in G L_n\left(R\left[\frac{1}{s_i s_j}\right]\right)$ with $g_{i i}=1$ and $g_{i j} g_{j k}=g_{i k}$ in $G L_n\left(R\left[\frac{1}{s_i s_j s_k}\right]\right)$ for every $i, j, k$. Then
$$
P=\left\{\left(x_1, \ldots, x_c\right) \in \prod_{i=1}^c\left(R\left[\frac{1}{s_i}\right]\right)^n: g_{i j}\left(x_j\right)=x_i \text { in } R\left[\frac{1}{s_i s_j}\right]^n \text { for all } i, j\right\}
$$
is a finitely generated projective $R$-module by , because each $P\left[\frac{1}{s_i}\right]$ is isomorphic to $R\left[\frac{1}{s_i}\right]^n$.\\

Another type of patching arises from an ideal $I$ in $R$ and a ring map $f: R \rightarrow S$ such that $I$ is mapped isomorphically onto an ideal of $S$, which we also call $I$. In this case $R$ is the "pullback" of $S$ and $R / I$ :
$$
R=\{(\bar{r}, s) \in(R / I) \times S: \bar{f}(\bar{r})=s \text { modulo } I\}
$$
the square \begin{tikzcd}
    R \arrow{r}{f} \arrow{d}{} & S \arrow{d}{} \\
    R / I \arrow{r}{\bar{f}} & S / I
\end{tikzcd} is called a {\bfseries Milnor square}.
\begin{example}[Conductor square]
 This arises when $R$ is commutative and $S$ is a finite extension of $R$ with the same total ring of fractions. ( $S$ is often the integral closure of $R$.) The ideal $I$ is chosen to be the conductor ideal, i.e., the largest ideal of $S$ contained in $R$, which is just $I=\{x \in R: x S \subset R\}=\operatorname{ann}_R(S / R)$. If $S$ is reduced, then $I$ cannot lie in any minimal prime of $R$ or $S$, so the rings $R / I$ and $S / I$ have lower Krull dimension.
\end{example}

Given a Milnor square, we can construct an $R$-module $M=\left(M_1, g, M_2\right)$ from the following "descent data": an $S$-module $M_1$, an $R / I$-module $M_2$, and an $S / I$-module isomorphism $g: M_2 \otimes_{R / I} S / I \cong M_1 / I M_1$. In fact $M$ is the kernel of the $R$-module map
$$
M_1 \times M_2 \rightarrow M_1 / I M_1, \quad\left(m_1, m_2\right) \mapsto \bar{m}_1-g\left(\bar{f}\left(m_2\right)\right) .
$$
We call $M$ the $R$-module obtained by patching $M_1$ and $M_2$ together along $g$.
An important special case is when we patch $S^n$ and $(R / I)^n$ together along a matrix $g \in G L_n(S / I)$. For example, $R$ is obtained by patching $S$ and $R / I$ together along $g=1$. We will return to this point when we study $K_1(R)$ and $K_0(R)$.
\begin{theo}
    In a Milnor square;
    \begin{enumerate}
        \item If $P$ is obtained by patching together a finitely generated projective $S$-module $P_1$ and a finitely generated projective $R / I$-module $P_2$, then $P$ is a finitely generated projective $R$-module.
        \item $P \otimes_R S \cong P_1$ and $P / I P \cong P_2$.
        \item Every finitely generated projective $R$-module arises in this way.
        \item If $P$ is obtained by patching free modules along $g \in G L_n(S / I)$ and $Q$ is obtained by patching free modules along $g^{-1}$, then $P \oplus Q \cong R^{2 n}$.
    \end{enumerate} 
\end{theo}

\subsubsection{Deternimant and elementary matrices}

\begin{theo}[Existence of determinant]
Let $(R, \mathcal{M})$ be a (possibly non-commutative) local ring. There is a well-defined determinant homomorphism $\det: GL(R) \rightarrow \overline{R}^*$, where $\bar{R}^*=\left(R^*\right)^{a b}$, satisfying:
\begin{enumerate}
    \item $\operatorname{det}(A B)=\operatorname{det} A \cdot \operatorname{det} B$,
    \item $\operatorname{det} A=1$ for all $A \in E(R)$,
    \item the composite $R^*=G L_1(R) \longrightarrow G L(R) \xrightarrow{\text { det }} \bar{R}^*$ is the natural quotient map.
\end{enumerate}
\end{theo}

Let $E_n(R)$ be the subgroup of \textbf{elementary matrices}, defined to be the group generated by the matrices $e_{i j}^{(n)}(\lambda), 1 \leq i \neq j \leq n, \lambda \in R$, where $e_{i j}^{(n)}(\lambda)$ is the unipotent matrix whose only non-trivial off-diagonal entry is $\lambda$ in the $(i, j)$ th position. Thus, if $i<j$, then $e_{i j}^{(n)}(\lambda)$ has the form 
\[
e_{ij}^{(n)}(\lambda) =
\begin{pmatrix}
1 & 0 & \cdots & 0 & \cdots & 0 \\
0 & 1 & \cdots & 0 & \cdots & 0 \\
\vdots & \vdots & \ddots & \vdots & \ddots & \vdots \\
0 & 0 & \cdots & 1 & \lambda & \cdots & 0 \\
\vdots & \vdots & \ddots & 0 & 1 & \ddots & \vdots \\
0 & 0 & \cdots & 0 & \cdots & 0 & 1
\end{pmatrix}
\]
Let $G L_n(R) \hookrightarrow G L_{n+1}(R)$ by
$
A \longmapsto\left[\begin{array}{ll}
A & 0 \\
0 & 1
\end{array}\right]
$
and let $G L(R)=\lim _{\rightarrow} G L_n(R)$. Similarly, let $E(R)=\lim _{\rightarrow} E_n(R)$. Since $e_{i j}^{(n)}(\lambda) \longmapsto e_{i j}^{(n+1)}(\lambda)$ under $E_n(R) \hookrightarrow E_{n+1}(R)$, we obtain matrices $e_{i j}(\lambda)$ $\in E(R)$ as the common image of all the $e_{i j}^{(n)}(\lambda)$ for $n \geq i, j$, and $E(R)$ is the subgroup of $G L(R)$ generated by the $e_{i j}(\lambda)$. The $e_{i j}(\lambda)$ satisfy the following identities:
\begin{enumerate}
    \item $e_{i j}(\lambda) \cdot e_{i j}(\mu)=e_{i j}(\lambda+\mu), \forall \lambda, \mu \in R$
    \item $\left[e_{i j}(\lambda), e_{k \ell}(\mu)\right]=1$ for $j \neq k, i \neq \ell$
    \item $\left[e_{i j}(\lambda), e_{j k}(\mu)\right]=e_{i k}(\lambda \mu)$ for $i \neq k, \forall \lambda, \mu \in R$.
\end{enumerate}

\begin{lemm}
    \begin{enumerate}
        \item $E_n(R)$, $n \geq 3$, and $E(R)$ are perfect groups.
        \item (Whitehead) $E(R)=[E(R), E(R)]=[G L(R), G L(R)]$.
    \end{enumerate}
\end{lemm}

\subsection{The Picard group of conmutative rings}

