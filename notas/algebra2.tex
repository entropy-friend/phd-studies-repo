
\chapter{Group (Cohomology) Theory} 

A \textbf{semigroup} is a nonempty set G together with a binary operation on G which is associative. A \textbf{monoid} is a semigroup G which contains a (two-sided) identity element. A \textbf{group} is a monoid G such that for every element there exists a (two-sided) inverse element.

\begin{theo}
Let G be a finitely generated abelian group.
    \begin{enumerate}
        \item There is a unique nonnegative integer s such that the number of infinite cyclic summands in any decomposition of G as a direct sum of cyclic groups is precisely s;
        \item either G is free abelian or there is a unique list of (not necessarily distinct) positive integers $\mathrm{m}_1, \ldots, \mathrm{m}_{\mathrm{t}}$ such that $\mathrm{m}_1>1, \mathrm{~m}_1\left|\mathrm{~m}_2\right| \cdots \mid \mathrm{m}_{\mathrm{t}}$ and
        $$
        \mathrm{G} \cong \mathbf{Z}_{m_1} \oplus \cdots \oplus \mathbf{Z}_{m_t} \oplus \mathrm{F}
        $$
        with F free abelian;
        \item either G is free abelian or there is a list of positive integers $\mathrm{p}_1{ }^{s_1}, \ldots, \mathrm{p}_{\mathrm{k}}{ }^{8 k}$, which is unique except for the order of its members, such that $\mathrm{p}_1, \ldots, \mathrm{p}_{\mathrm{k}}$ are (not necessarily distinct) primes, $\mathrm{s}_1, \ldots, \mathrm{s}_{\mathrm{k}}$ are (not necessarily distinct) positive integers and
        $$
        \mathrm{G} \cong \mathrm{Z}_{\mathrm{p} 1} \mathrm{st}^1 \oplus \ldots \oplus \mathrm{Z}_{\mathrm{pk}} \mathrm{sk}_{\mathrm{k}} \oplus \mathrm{F}
        $$
        with F free abelian.
    \end{enumerate}
\end{theo}



\section{Actions}

An action of a group G on a set S is a function $\mathrm{G} \times \mathrm{S} \rightarrow \mathrm{S}$ (usually denoted by $(\mathrm{g}, \mathrm{x}) \mapsto \mathrm{gx}$ ) such that for all $\mathrm{x} \varepsilon \mathrm{S}$ and $\mathrm{g}_1, \mathrm{~g}_2 \in \mathrm{G}$ :
$$
\mathrm{ex}=\mathrm{x} \quad \text { and } \quad\left(\mathrm{g}_1 \mathrm{~g}_2\right) \mathrm{x}=\mathrm{g}_1\left(\mathrm{~g}_2 \mathrm{x}\right) .
$$

When such an action is given, we say that G acts on the set S. $Gx$ denotes the orbit of $x$ and $G_x$ denotes its stabilizer (or isotropy group).



\begin{theo}
    \begin{enumerate}
        \item Orbits have cardinality equal to the index of the corresponding stabilizer.
        \item The number of elements in the conjugacy class of $\mathrm{x} \varepsilon \mathrm{G}$ is $\left[\mathrm{G}: \mathrm{C}_{\mathrm{G}}(\mathrm{x})\right]$, which divides $|\mathrm{G}|$;
        \item (\textbf{Class equation}) if $\overline{\mathrm{x}}_1, \ldots, \overline{\mathrm{x}}_{\mathrm{n}}\left(\mathrm{x}_{\mathrm{i}} \in \mathrm{G}\right)$ are the distinct conjugacy classes of G , then $$
|\mathrm{G}|=\sum_{i=1}^n\left[\mathrm{G}: \mathrm{C}_{\mathrm{G}}\left(\mathrm{x}_{\mathrm{i}}\right)\right]
$$ In particular, we can take $G$ acting on itself by conjugation, so that the conjugacy classes are the orbits of this action.
\item the number of subgroups of G conjugate to K is $\left[\mathrm{G}: \mathrm{N}_{\mathrm{G}}(\mathrm{K})\right]$, which divides $|\mathrm{G}|$.
\end{enumerate}
\end{theo}

Let $G$ and $H$ be groups and $\theta: H \rightarrow$ Aut $G$ a homomorphism. Let $G \times_\theta H$ be the set $G \times H$ with the following binary operation: $(g, h)\left(g^{\prime}, h^{\prime}\right)=\left(g\left[\theta(h)\left(g^{\prime}\right)\right], h h^{\prime}\right)$. Show that $G \times_\theta H$ is a group with identity element $(e, e)$ and $(g, h)^{-1}=$ $\left(\theta\left(h^{-1}\right)\left(g^{-1}\right), h^{-1}\right) . G \times_\theta H$ is called the semidirect product of $G$ and $H$.
















\paragraph*{Group Ring}

Let $G$ be a group, written multiplicatively. Let $\mathbb{Z} G$ be the free $\mathbb{Z}$-module generated by the elements of $G$. The multiplication in $G$ extends uniquely to a $\mathbb{Z}$-bilinear product $\mathbb{Z G} \times \mathbb{Z G} \rightarrow \mathbb{Z G}$; this makes $\mathbb{Z G}$ a ring, called the \textbf{integral group ring} of $G$.

Note that $G$ is a subgroup of the group $(\mathbb{Z} G)^*$ of units of $\mathbb{Z G}$ 
\begin{theo}[Universal property]
Given a ring $R$ and a group homomorphism $f: G \rightarrow R^*$, there is a unique extension of $f$ to a ring homomorphism $\mathbb{Z G} \rightarrow R$. Thus we have the "adjunction formula"
    $$
    \operatorname{Hom}_{\text {(rings) }}(\mathbb{Z} G, R) \approx \operatorname{Hom}_{\text {(groups) }}\left(G, R^*\right) .
    $$
\end{theo}

A \textbf{(left) $\mathbb{Z} G$-module}, or $G$-module, consists of an abelian group $A$ together with a homomorphism from $\mathbb{Z} G$ to the ring of endomorphisms of $A$. By the universal property, $G$-module is simply an abelian group $A$ together with an action of $G$ on $A$. For example, one has for any $A$ the trivial module structure, with $g a=a$ for $g \in G, a \in A$.

One way of constructing $G$-modules is by linearizing permutation representations. More precisely, if $X$ is a $G$-set (i.e., a set with $G$-action), then one forms the free abelian group $\mathbb{Z X}$ (also denoted $\mathbb{Z}[X]$ ) generated by $X$ and one extends the action of $G$ on $X$ to a $\mathbb{Z}$-linear action of $G$ on $\mathbb{Z} X$. The resulting $G$-module is called a permutation module. In particular, one has a permutation module $\mathbb{Z}[G / H]$ for every subgroup $H$ of $G$, where $G / H$ is the set of cosets $g H$ and $G$ acts on $G / H$ by left translation.

\begin{prop}
Let $X$ be a free $G$-set and let $E$ be a set of representatives for the $G$-orbits in $X$. Then $\mathbb{Z}X$ is a free $\mathbb{Z}G$-module with basis $E$.
\end{prop}


\section{Co-invariants}
If $G$ is a group and $M$ is a $G$-module, then the group of co-invariants of $M$, denoted $M_G$, is defined to be the quotient of $M$ by the additive subgroup generated by the elements of the form $g m-m\left(g \in G, m \in M\right.$ ). Thus $M_G$ is obtained from $M$ by "dividing out" by the $G$-action. (The name "co-invariants" comes from the fact that $M_G$ is the largest quotient of $M$ on which $G$ acts trivially, whereas $M^G$, the group of invariants, is the largest submodule of $M$ on which $G$ acts trivially.) In view of exercise $1 \mathrm{a}$ of $\$ I .2$, we can also describe $M_G$ as $M / I M$, where $I$ is the augmentation ideal of $\mathbb{Z} G$ and $I M$ denotes the set of all finite sums $\sum a_i b_i\left(a_i \in I, b_i \in M\right)$.
Still another description of $M_G$ is given by:
$$
M_G \approx \mathbb{Z} \otimes_{\mathbb{Z} G} M .
$$

Here, in order for the tensor product to make sense, we regard $\mathbb{Z}$ as a right $\mathbb{Z} G$-module (with trivial $G$-action). To prove 2.1 , note that in $\mathbb{Z} \otimes_{\mathbb{Z} G} M$ we have the identity $1 \otimes g m=1 \cdot g \otimes m=1 \otimes m ;$ hence there is a map $M_G \rightarrow$ $\mathbb{Z} \otimes_{\mathbb{Z} G} M$ given by $\bar{m} \mapsto 1 \otimes m$, where $\bar{m}$ denotes the image in $M_G$ of an element $m \in M$. On the other hand, using the universal property of the tensor product, we can define a map $\mathbb{Z} \otimes_{\mathbb{Z} G} M \rightarrow M_G$ by $a \otimes m \mapsto a \bar{m}$. These two maps are inverses of one another.

In view of 2.1 and standard properties of the tensor product, we immediately obtain the following two properties of the co-invariants functor:




\section{An spectral sequence for group cohomology}

Suppose that $X$ is a simplicial set and $x_i$ are simplicial subsets such that $X=U X_i$. Then, setting $X_{i j}=X_i \cap X_j$ (etc.) we'11 obviously have for the realisations: $|x|=U\left|x_i\right|,\left|x_i\right| \cap\left|x_j\right|=\left|x_{i j}\right|, \ldots$ Let's suppose that the set of indices is linearly ordered. Consider the following bicomplex:
$$ K = \longrightarrow \underset{i<j<k}{\oplus} C_*\left(x_{i j k}\right) \longrightarrow \underset{i<j}{\oplus} C_*\left(x_{i j}\right)\longrightarrow \underset{i}{\oplus} C_*\left(x_{i}\right) $$


Here by a bicomplex we understand a bicomplex in the sense of Grothendieck [9] i.e. the differentials $d_1$ and $d_2$ commute. (The sign in this approach appears in the definition of the total differentials). The vertical arrows of the bicomplex map $C_*\left(x_i \cdots_i\right)$ into $\underset{k=0}{q} C_*\left(x_{i_0} \ldots \hat{i}_k \ldots i_q\right)$, the mapping into the kth summand differing $k=0$ by a sign $(-1)^k \quad$ from the natural embedding.

The first spectral sequence of this bicomplex degenerates and yields an isomorphism $H_{\star}(K) \cong H_{\star}(X)$. (Moreover this isomorphism is induced by the canonical map $K \rightarrow C_*(X)$). The second spectral sequence gives us a functorial spectral sequence of the first quadrant, whose limit equals $H_*(X)$, while its differential $d r$ has bidegree $(r-1,-r)$ and its $E^1$-term looks as follows: $$E_{p q q}^1=\underset{i_0<\ldots<i_q}{\otimes} H_p\left(x_{i_0} \ldots i_q\right)$$

Suppose $G$ is a group. Let $X_G$ denote the simplicial set (and its geometric realisation), whose p-simplices are sequences $\left(g_0, \ldots, g_p\right)$ of elements of $G$, with the usual faces and degeneracies. This space $X_G$ is contractible by (1.2). The group $G$ acts from the right on $X_G$ and this action is obviously free, hence $B G=X_G / G$ is a classifying space of $G$. The complex $C_*(B G)=C_*(G)$ coincides with the usual complex associated with $G$. Moreover $C_*(G)=C_*\left(X_G\right) \otimes_G Z$.

If $H$ is a subgroup of $G$, then $X_G / H$ is a classifying space for $H$ and hence $B H=X_H / H \rightarrow X_G / H$ is a homotopy equivalence. In particular $C_*(H)+C_*\left(X_G\right) \otimes_H \mathbb{Z}=C_*\left(X_G\right) \otimes_G Z|G / H|$ is a homotopy equivalence.

(2.3) The spectral sequence associated with a family of subgroups.

Suppose $G$ is a group and $G_1, \ldots, G_n$ are subgroups. Then $B G_i$ may be viewed as a simplicial subset of $B G$ and $B G_i \cap B G_j=B\left(G_i \cap G_j\right)$.. Denote $U B G_i$ by $X$ and consider the spectral sequence of the covering $X=U B G_i$. Along with the bicomplex $K$ introduced in (2.1) we also consider the following bicomplex:

$$K' = \underset{i<j<k}{\oplus} C_*\left(X_G\right) \otimes_G Z\left[G / G_{i j k}\right] \longrightarrow \underset{i<j}{\oplus} C_*\left(X_G\right) \otimes_G Z\left[G / G_{i j}\right] \longrightarrow \underset{i<j}{\oplus} C_*\left(X_G\right) \otimes_G Z\left[G / G_{i}\right] $$

There is a natural mapping of bicomplexes $K+K^{\prime}$ and because of (2.2) this mapping induces an isomorphism of second spectral sequences so that $H_{\star}(X)=H_{\star}(K)=H_*\left(K^{\prime}\right)$. The first spectral sequence of $K^{\prime}$ looks as follows: $E_{*, q}^1=C_*\left(X_G\right) \otimes_G H_q(L)$, where $L$ is the following complex of left G-modules:
$$
\oplus \mathbb{Z}\left[G / G_i\right]+\oplus \mathbb{Z}\left[G / G_{i j}\right]+\oplus \mathbb{Z}\left[G / G_{i j k}\right]+\ldots
$$


\begin{prop}
If $G_1, \ldots, G_n$ are subgroups of $G$, there exists a fuctorial spectral sequence of the first quadrart, the $E^2$ term of which looks like: $E_{p q}^2=H_p\left(G, H_q(L)\right)$, where $L$ is the complex defined above. It converges to $H_{\star}\left(U B G_j\right)$ and the differential $d^r$ has bidegree $(-r, r-1)$.   
\end{prop}

(2.5) In the notations of (2.3), let $Z(G,\{G\})$ be the simplicial set whose non-degenerate p-simplices are sequences $\left(\bar{g}_0, \ldots, \bar{g}_p\right)$, where $\bar{g}_i \varepsilon G / G_{k_i}, k_0<\ldots<k_p$, and the $\bar{g}_i$ are such that there is $g \in G$ with $\vec{g}_i=g \bmod G_{k_i}$ for all i. (If one covers $G$ by the right cosets of the $G_i$, then $Z\left(G_g\left\{G_i\right\}\right)$ is the nerve of this covering.) It is easy to see that the geometric realization of this simplicial set is an ordered simplicial space and that the complex $L=L\left(G,\left\{G_i\right\}\right)$ equals the (ordered) simplicial complex [7] of this simplicial space, or in other words, the complex $L$ equals the normalised complex of the simplicial set $Z\left(G,\left\{G_i\right\}\right)$. In particular, $H_*(L)=H_*\left(Z\left(G,\left\{G_i\right\}\right)\right)$.

(2.6) Remark. It may be shown easily that the space $Z\left(G,\left\{G_i\right\}\right)$, is homotopy equivalent to Volodin's space $V\left(G,\left\{G_i\right\}\right)$, but we will not need this fact.




\chapter{Rings (with identity)}

Let R be a ring and S a nonempty subset of R that is closed under the operations of addition and multiplication in R . If S is itself a ring under these operations then S is called a subring of R . A subring I of a ring R is a \textbf{left ideal} provided
$$
\mathrm{r} \varepsilon \mathrm{R} \text { and } \mathrm{x} \in \mathrm{I} \quad \Rightarrow \quad \mathrm{rx} \in \mathrm{I} \text {; }
$$

I is a \textbf{right ideal} provided
$$
\mathrm{r} \varepsilon \mathrm{R} \text { and } \mathrm{x} \in \mathrm{I} \quad \Rightarrow \quad \mathrm{xr} \varepsilon \mathrm{I} \text {; }
$$

I is an \textbf{ideal} if it is both a left and right ideal. Note that proper ideals does not contain any unit. We denote by $(X)$ the ideal generated by the subset $X$ of R , i.e., the smallest ideal containing X.
\begin{theo}
    \begin{enumerate}
        \item (a) $=\left\{\sum_{i=1}^n \mathrm{r}_{\mathrm{j}} \mathrm{as}_{\mathrm{i}} \mid \mathrm{r}_{\mathrm{i}}, \mathrm{s}_{\mathrm{i}} \in \mathrm{R} ; n \in \mathbf{N}^*\right\}$ (principal ideal).
        \item If a is in the center of R , then $\mathrm{Ra}=(\mathrm{a})=\mathrm{aR}$.
        \item If X is in the center of R , then the ideal $(\mathrm{X})$ consists of all finite sums $\mathrm{r}_1 \mathrm{a}_1+\cdots+\mathrm{r}_{\mathrm{n}} \mathrm{a}_{\mathrm{n}}\left(\mathrm{n} \in \mathbf{N}^* ; \mathrm{r}_{\mathrm{i}} \in \mathrm{R} ; \mathrm{a}_{\mathrm{i}} \varepsilon \mathrm{X}\right)$.
        \item for ideals, multiplication and addition are distributive and associative.
    \end{enumerate}
\end{theo}


An ideal P in a ring R is said to be prime if $\mathrm{P} \neq \mathrm{R}$ and for any ideals $\mathrm{A}, \mathrm{B}$ in R
$$
\mathrm{AB} \subset \mathrm{P} \Rightarrow \mathrm{A} \subset \mathrm{P} \text { or } \mathrm{B} \subset \mathrm{P} \text {. }
$$
\begin{theo}
If P is an ideal in a ring R such that $\mathrm{P} \neq \mathrm{R}$ and for all $\mathrm{a}, \mathrm{b} \varepsilon \mathbf{R}$
$$
\mathrm{ab} \varepsilon \mathrm{P} \Rightarrow \mathrm{a} \varepsilon \mathrm{P} \text { or } \mathrm{b} \varepsilon \mathrm{P} \text {, }
$$
then P is prime. Conversely if P is prime and R is commutative, then P satisfies condition (1).
\end{theo}

An ideal [resp. left ideal] M in a ring R is said to be \textbf{maximal} if $\mathbf{M} \neq \mathrm{R}$ and for every ideal [resp. left ideal] $\mathbf{N}$ such that $\mathbf{M} \subset \mathbf{N} \subset \mathrm{R}$, either $\mathbf{N}=\mathbf{M}$ $\operatorname{or} \mathbf{N}=\mathbf{R}$.

\begin{theo}
    \begin{enumerate}
        \item In a nonzero ring R with identity maximal [left] ideals always exist. In fact every $[l e f t]$ ideal in R (except R itself) is contained in a maximal [left] ideal.
        \item (In general, for $R^2 = R$) Every maximal ideal is prime. 
        \item If $M$ is maximal and $R$ is conmutative, then the $R/M$ is a field. the converse is true in general, even when $R/M$ is noncommutative.
        \item $R$ is a field if and only if the $(0)$ is a maximal ideal.
    \end{enumerate}
\end{theo}

%\subsection*{Localization and tensor products}

%A nonempty subset S of a ring R is multiplicative provided that $ a, b \in S \Rightarrow a b \varepsilon S .$




\section{Modules}

Let $I$ be a left ideal of the ring $R, A$ an $R$-module and $S$ a nonempty subset of $A$. Then $I S=\left\{\sum_{i=1}^n r_i a_i \mid r_i \in I ; a_i \varepsilon S ; n \in \mathbf{N}^*\right\}$ is a submodule of $A$ (Exercise 3). Similarly if $a \in A$, then $I a=\{r a \mid r \in I\}$ is a submodule of $A$. We will consider just unitary modules.\\

If X is a subset of a module A over a ring R , then the intersection of all submodules of A containing X is called the submodule generated by X (or spanned by X). We have $$(A)=R X=\left\{\sum_{i=1}^s r_i a_i \mid s \in N^* ; a_i \in X ; r_i \in R\right\}$$

\begin{theo}[Free-modules] Let $\mathbf{R}$ be a ring with identity. The following conditions on a unitary R -module F are equivalent:
    \begin{enumerate}
        \item F has a nonempty basis;
        \item F is the internal direct sum of a family of cyclic R -modules, each of which is isomorphic as a left $\mathbf{R}$-module to $\mathbf{R}$;
        \item F is R -module isomorphic to a direct sum of copies of the left R -module R ;
        \item there exists a nonempty set X and a function $\iota: \mathrm{X} \rightarrow \mathrm{F}$ with the following property: given any unitary R -module A and function $\mathrm{f}: \mathrm{X} \rightarrow \mathrm{A}$, there exists a unique R -module homomorphism $\overline{\mathrm{f}}: \mathrm{F} \rightarrow \mathrm{A}$ such that $\overline{\mathrm{f}} \iota=\mathrm{f}$. In other words, F is a free object in the category of unitary R -modules.
    \end{enumerate}
\end{theo}

\begin{theo}
        Let R be a ring with identity and F a free R -module with an infinite basis X. Then every basis of F has the same cardinality as X.
\end{theo}

Let R be a ring with identity such that for every free R -module F , any two bases of F have the same cardinality. Then R is said to have the \textbf{invariant dimension} property and the cardinal number of any basis of F is called the dimension (or rank) of F over R. \\
Note that, in this case, two modules are isomorphic if and only if they have the same rank.\\

\begin{theo} For ring with identity:
    \begin{enumerate}
        \item Every lineary independent subset of a vector space over a division ring can be extended to a basis.
        \item Every division ring has the invariant dimension property.
        \item Every conmutative ring has the invariant dimension property.
    \end{enumerate}
    \end{theo}

\subsection*{Linear Algebra}

If R is a ring, then the set of all $\mathrm{n} \times \mathrm{m}$ matrices over R forms an $\mathrm{R}-\mathrm{R}$ bimodule under addition, with the $\mathrm{n} \times \mathrm{m}$ zero matrix as the additive identity. Multiplication of matrices, when defined, is associative and distributive over addition. For each $\mathrm{n}>0$, Mat R is a ring. If R has an identity, so does Mat R (namely the identity matrix $\mathrm{I}_{\mathrm{n}}$ ).

\begin{theo}
Let R be a ring with identity. Let E be a free left R -module with a finite basis of n elements and F a free left R -module with a finite basis of m elements. Let M be the left R -module of all $\mathrm{n} \times \mathrm{m}$ matrices over R . Then there is an isomorphism of abelian groups:
$$
\operatorname{Hom}_{\mathrm{R}}(\mathrm{E}, \mathrm{F}) \cong \mathrm{M} .
$$
If R is commutative this is an isomorphism of left R -modules.
\end{theo}








