\part{Topics of Algebra}

\chapter{Category Theory}

\section*{Basic concepts}

\begin{enumerate}
    \item For a topological spaces, the category of open sets with inclusions as morphisms. The opposite of this category, denoted by $\mathfrak{U}$, is used in sheaf theory.
    \item If \ca and \cb are preordered sets, then functors between them are monotone maps.
\end{enumerate}

% A functor $F: \mathcal{A} \rightarrow \mathcal{B}$ is called an \textbf{isomorphism} provided that there is a functor $G: \mathbf{B} \rightarrow \mathbf{A}$ such that $G \circ F=i d_{\mathbf{A}}$ and $F \circ G=i d_{\mathbf{B}}$. The categories $\mathbf{A}$ and $\mathbf{B}$ are said to be \textbf{isomorphic}. \\
% Note that $G$ is uniquely determined by $F$. It will be denoted by $F^{-1}$ and called the \textbf{inverse} of $F$.\\

\noindent Let $F: \mathcal{A} \rightarrow \mathcal{B}$ be a functor.
\begin{enumerate}
    \item $F$ is \textbf{faithful} provided that all the \textbf{hom-set restrictions}
    $$
    F: \operatorname{hom}_{\mathcal{A}}\left(A, A^{\prime}\right) \rightarrow \operatorname{hom}_{\mathcal{B}}\left(F A, F A^{\prime}\right)
    $$
    are injective.
    \item $F$ is \textbf{full} if all hom-set restrictions are surjective.
    \item $F$ is an \textbf{embedding} if and it is faithful and injective on the class of objects.
    \item $F$ is \textbf{essentially surjective} if for every object $B$ of \cb, there is an object $A$ of \ca such that $F A$ is isomorphic to $B$. 
    \item If $F$ is essentially surjective and fully faithful, it is called an \textbf{equivalence of categories}, and \ca and \cb are said to be \textbf{equivalent}.
\end{enumerate}


\section{Natural transformations}
Let $F, G: \mathbf{A} \rightarrow \mathbf{B}$ be functors. A natural transformation $\tau$ from $F$ to $G$ (denoted by $\tau: F \rightarrow G$ or $F \xrightarrow{\tau} G$ ) is a function that assigns to each $\mathbf{A}$-object $A$ a $\mathbf{B}$-morphism $\tau_A: F A \rightarrow G A$ in such a way that the following naturality condition holds: for each A-morphism $A \xrightarrow{f} A^{\prime}$, the diagram
$
\begin{tikzcd}
FA \arrow[r, "\tau_A"] \arrow[d, "Ff"'] & GA \arrow[d, "Gf"] \\
FA' \arrow[r, "\tau_{A'}"'] & GA'
\end{tikzcd}
$ commutes.\\
A natural transformation $F \xrightarrow{\tau} G$ whose components $\tau_A$ are isomorphisms is called a \textbf{natural isomorphism} from $F$ to $G$. $F$ and $G$ are said to be \textbf{naturally isomorphic}, and we write $F \cong G$.


\begin{example}
    \begin{enumerate}
        \item Let $U: \operatorname{Grp} \rightarrow$ Set be the forgetful functor, and let $S: \operatorname{Grp} \rightarrow$ Set be the "squaring-functor", defined by $S(G \xrightarrow{f} H)=G^2 \xrightarrow{f^2} H^2$. For each group $G$, its multiplication is a function $\tau_G: G^2 \rightarrow G$. The family $\tau=\left(\tau_G\right)$ is a natural transformation from $S$ to $U$. The naturality condition simply means that $f(x \cdot y)=f(x) \cdot f(y)$ for any group homomorphism $G \xrightarrow{f} H$ and any $x, y \in G$. Thus "multiplication" in groups can be regarded as a natural transformation. Similar for other structures.
        \item Let $(\hat{\imath}):$ Vec $\rightarrow$ Vec be the second-dual functor for vector spaces, then $\tau_V: V \rightarrow \hat{\hat{V}}$, defined by $\left(\tau_V(x)\right)(f)=f(x)$, yield a natural transformation $i d{ }_{\mathrm{Vec}} \xrightarrow{\tau}(\hat{\wedge})$. It becomes a natural isomorphism when restricted to finite-dimensional vector spaces.
        \item The assignment of the Hurewicz homomorphism $\pi_n(X) \rightarrow H_n(X)$ to each topological space $X$ is a natural transformation from the $n$-th homotopy functor $\pi_n:$ Top $\rightarrow$ Grp to the $n$-th homology functor $H_n:$ Top $\rightarrow$ Grp.
        \item If $B \xrightarrow{f} C$ is an $\mathbf{A}$-morphism, then
        $
        \operatorname{hom}_{\mathbf{A}}(C,-) \xrightarrow{\tau_f} \operatorname{hom}_{\mathbf{A}}(B,-),
        $
        defined by $\tau_f(g)=g \circ f$, and
        $
        \operatorname{hom}_{\mathbf{A}}(-, B) \xrightarrow{\sigma_f} \operatorname{hom}_{\mathbf{A}}(-, C) \text {, }
        $
        defined by $\sigma_f(g)=f \circ g$, are natural transformations.
        \item (Good definitions of extension) Let $F:$ Set $\rightarrow$ Vec be a functor that assigns to each set $X$ a vector space $F X$ with basis $X$, and to each function $X \xrightarrow{f} Y$ the unique linear extension $F X \xrightarrow{F f} F Y$ of $f$. This actually is not a correct definition of a functor, since there are many different vector spaces with the same basis. However, the definition is "correct up to natural isomorphism". Whenever we choose, for each set $X$, a specific vector space $F X$ with basis $X$, we do obtain a functor $F:$ Set $\rightarrow$ Vec (since the above condition determines the action of $F$ on functions uniquely). Furthermore, any two functors that are obtained in this way are naturally isomorphic.
        \item For any 2-element set $A$, hom $(A,-)$ is naturally isomorphic to the squaring functor $S^2[3.20(10)]$ and hom $(-, A)$ is naturally isomorphic to the contravariant power-set functor $\mathcal{Q}[3.20(9)]$. If $B$ is isomorphic to $A$, then hom $(A,-)$ and hom $(B,-)$ are naturally isomorphic with those functors, the converse is true.
    \end{enumerate}
\end{example}

\begin{prop}
    A functor $\mathbf{A} \xrightarrow{F} \mathbf{B}$ is an equivalence if and only if there exists a functor $\mathbf{B} \xrightarrow{G} \mathbf{A}$ such that $i d_{\mathbf{A}} \cong G \circ F$ and $F \circ G \cong i d_{\mathbf{B}}$.
\end{prop}


\section{Limits and colimits}
Let $A \stackrel{\text { f }}{\underset{g}{\rightrightarrows}} B$ be a pair of morphisms. A morphism $E \xrightarrow{e} A$ is called an equalizer of $f$ and $g$ provided that the following conditions hold:
(1) $f \circ e=g \circ e$,
(2) for any morphism $e^{\prime}: E^{\prime} \rightarrow A$ with $f \circ e^{\prime}=g \circ e^{\prime}$, there exists a unique morphism $\bar{e}: E^{\prime} \rightarrow E$ such that $e^{\prime}=e \circ \bar{e}$, i.e., such that the triangle
\begin{tikzcd}
    E' \arrow[d, "\overline{e}"'] \arrow[dr, "e'"] & \\
    E \arrow[r, "e"'] & A \arrow[r, shift left, "f"] \arrow[r, shift right, "g"'] & B
    \end{tikzcd} commutes.
    

A \textbf{source} is a pair $\left(A,\left(f_i\right)_{i \in I}\right)$ consisting of an object $A$ and a family of morphisms $f_i: A \rightarrow A_i$ with domain $A$, indexed by some class $I$.

%A source $\mathcal{P}=\left(P \xrightarrow{p_i} A_i\right)_I$ is called a product provided that for every source $\mathcal{S}=$ $\left(A \xrightarrow{f_i} A_i\right)_I$ with the same codomain as $\mathcal{P}$ there exists a unique morphism $A \xrightarrow{f} P$ with $\mathcal{S}=\mathcal{P} \circ f$. A product with codomain $\left(A_i\right)_I$ is called a product of the family $\left(A_i\right)_I$.

\begin{prop}
    A category has finite products if and only if it has terminal objects and products of pairs of objects.
    A category that has products for all class-indexed families must be thin.
A small category has products if and only if it is equivalent to a complete lattice.
\end{prop}

A \textbf{diagram} in a category $\mathbf{A}$ is a functor $D: \mathbf{I} \rightarrow \mathbf{A}$ with codomain $\mathbf{A}$. The domain, $\mathbf{I}$, is called the \textbf{scheme} of the diagram. A diagram with a small (or finite) scheme is said to be \textbf{small} (or finite).

\begin{enumerate}
    \item An A-source $\left(A \xrightarrow{f_i} D_i\right)_{i \in O b(\mathrm{I})}$ is said to be natural for $D$ provided that for each I-morphism $i \xrightarrow{d} j$, the triangle 
\begin{tikzcd}
    A \arrow[d, "f_i"'] \arrow[dr, "f_j"] & \\
    D_i \arrow[r, "Dd"'] & D_j 
\end{tikzcd} commutes.
    \item A \textbf{limit} of $D$ is a natural source $\left(L \xrightarrow{\ell_i} D_i\right)$ for $D$ with the \textbf{universal property} that for each natural source $\left(A \xrightarrow{f_i} D_i\right)$ there exists a unique morphism $f: A \rightarrow L$ with $f_i=$ $\ell_i \circ f$ for each $i \in O b(\mathbf{I})$.
    %corregir

    \item For a diagram with discrete scheme, every source is natural. A source is a limit of if it is a product . Expressed briefly: products are limits of diagrams with discrete schemes. In particular, an object, considered as an empty source, is a limit of the empty diagram (i.e., the one with empty scheme) if and only if it is a terminal object.

    For a pair of A-morphisms $A \underset{g}{\stackrel{f}{\rightrightarrows}} B$, considered as a diagram $D$ with scheme $\bullet \Rightarrow \bullet$, a source $(A \stackrel{e}{\longleftarrow} C \xrightarrow{h} B)$ is natural provided that $g \circ e=h=f \circ e$. Observe that in this case $h$ is determined by $e$. Hence, $C \xrightarrow{e} A$ is an equalizer of $A \xrightarrow[g]{\stackrel{f}{\longrightarrow}} B$ if and only if the source $(A \stackrel{e}{\leftarrow} C \xrightarrow{f \circ e} B)$ is a limit of $D$. Thus we may say (imprecisely) that equalizers are limits of diagrams with scheme $\bullet \Rightarrow \bullet$. If, in the above scheme, the two arrows are replaced by an arbitrary set of arrows, then limits of diagrams with such schemes are called multiple equalizers.
    If $\mathbf{I}$ is a down-directed ${ }^{57}$ poset (considered as a category), then limits of diagrams with scheme $\mathbf{I}$ are called projective (or inverse) limits. If, e.g., $\mathbf{I}=\mathbb{N}^{o p}$ is the poset of all non-negative integers with the opposite of the usual ordering, a diagram $D: \mathbf{I} \rightarrow \mathbf{A}$ with this scheme is essentially a sequence
    $$
    \ldots \xrightarrow{d_2} D_2 \xrightarrow{d_1} D_1 \xrightarrow{d_0} D_0
    $$
    of A-morphisms (where $D(n+1 \rightarrow n)=D_{n+1} \xrightarrow{d_n} D_n, D(n+2 \rightarrow n)=d_n \circ d_{n+1}$, etc.). A natural source for $D$ is a source $\left(A \xrightarrow{f_n} D_n\right)_{n \in \mathbb{N}}$ with $f_n=d_n \circ f_{n+1}$ for each $n$. In Set a projective limit of a diagram $D$ with scheme $\mathbb{N}^{\text {op }}$ is a source $\left(L \xrightarrow{\ell_n} D_n\right)_{n \in \mathbb{N}}$, where $L$ is the set of all sequences $\left(x_n\right)_{n \in \mathbb{N}}$ with $x_n \in D_n$ and $d_n\left(x_{n+1}\right)=x_n$ for each $n \in \mathbb{N}$; and where each $\ell_m$ is a restriction of the $m$ th projection $\pi_m: \prod_{n \in \mathbb{N}} D_n \rightarrow D_m$
    If $D: \mathbf{A} \rightarrow \mathbf{A}$ is the identity functor, then a source $\mathcal{L}=\left(L \xrightarrow{\ell_A} A\right)_{A \in O b \mathbf{A}}$ is a limit of $D$ if and only if $L$ is an initial object of $\mathbf{A}$. The sufficiency is obvious. For the necessity let $\mathcal{L}$ be a limit of $D$ and let $L \xrightarrow{f} A$ be a morphism. By the naturality of $\mathcal{L}$ for $D$ we obtain $f \circ \ell_L=\ell_A$. Application of this to $f=\ell_A$ yields $\ell_A \circ \ell_L=\ell_A=\ell_A \circ i d_L$ for each object $A$. Hence, by the uniqueness requirement
\end{enumerate}



\section{Adjoint functors}







\section{Concrete categories}

A way to talk of \textit{low level structures} present on the objects of a category. Often it is easier to work with less structures, and there results like Yoneda's lemma that show us that it is possible to restrict our study to them.\\

Let \cc be a category. A \textbf{concrete category} over \cc is a category $\mathbf{A}$ together wih a faithful functor $U: \mathbf{A} \rightarrow \mathcal{C}$, called the \textbf{forgetful} (or underlying) functor of the concrete category. $\mathcal{C}$ is called the \textbf{base category}. A concrete category over Set is called a \textbf{construct}.\\
The category of groups (or topological spaces, rings, etc.), with the forgetful functor to Set, is a construct.\\

A functor $F: \mathbf{A} \rightarrow$ Set is called representable (by an $\mathbf{A}$-object $A$ ) provided that $F$ is naturally isomorphic to the hom-functor $\operatorname{hom}(A,-): \mathbf{A} \rightarrow$ Set. Note that objects that represents the same functor are isomorphic.\\

\begin{example}
    \begin{enumerate}
        \item Forgetful functors are often representable. For example,
        (a) Vec $\rightarrow$ Set is represented by the vector space $\mathbb{R}$,
        (b) $\operatorname{Grp} \rightarrow$ Set is represented by the group of integers $\mathbb{Z}$,
        (c) Top $\rightarrow$ Set is represented by any one-point topological space.
        \item The underlying functor $U$ for the construct Ban [5.2(3)] is not representable (see Exercise 10J). However, the faithful unit ball functor $O: \operatorname{Ban} \rightarrow$ Set is represented in the complex case by the Banach space $\mathbb{C}$ of complex numbers.
    \end{enumerate}
\end{example}

\begin{prop}[Representative of Constructs]
    For constructs $(\mathbf{A}, U)$ the forgetful functor is represented by an object $A$ if and only if $A$ is a free object over a singleton set [see Definition 8.22(2)]. This provides many additional examples of representations.
\end{prop}

For small categories $\mathbf{A}$ and $\mathbf{B}$ the \textbf{functor category} $[\mathbf{A}, \mathbf{B}]$ has as objects all functors from $\mathbf{A}$ to $\mathbf{B}$, as morphisms from $F$ to $G$ all natural transformations from $F$ to $G$, as identities the identity natural transformations, and as composition the (horizontal) composition of natural transformations.

\begin{theo}[uniqueness of representations]
    For any functor $F: \mathbf{A} \rightarrow$ Set, any $\mathbf{A}$-object $A$ and any element $a \in F(A)$, there exists a unique natural transformation $\tau: \operatorname{hom}(A,-) \rightarrow F$ with $\tau_A\left(i d_A\right)=a$.
\end{theo}

\begin{coro}[Yoneda Lemma]
    If $F: \mathbf{A} \rightarrow$ Set is a functor and $A$ is an $\mathbf{A}$-object, then the following function
    $$
    Y:[\operatorname{hom}(A,-), F] \rightarrow F(A) \text { defined by } Y(\sigma)=\sigma_A\left(i d_A\right),
    $$
    is a bijection (where $[\operatorname{hom}(A,-), F]$ is the set of all natural transformations from hom $(A,-)$ to $F$ ).
    
\end{coro}

\begin{coro}[Yoneda Embedding]
    For any category $\mathbf{A}$, the functor $E: \mathbf{A} \rightarrow\left[\mathbf{A}^{\mathrm{op}} Set \right]$, defined by
$$
E(A \xrightarrow{f} B)=\operatorname{hom}(-, A) \xrightarrow{\sigma_f} \operatorname{hom}(-, B) \text {, }
$$
where $\sigma_f(g)=f \circ g$, is a full embedding.
\end{coro}












\chapter{Homological Algebra}

In homological algebra one constructs homological invariants of algebraic objects by the following process, or some variant of it:

Let $R$ be a ring and $T$ a covariant additive functor from $R$-modules to abelian groups. Thus the map $\operatorname{Hom}_R(M, N) \rightarrow \operatorname{Hom}_{\mathbf{z}}(T M, T N)$ defined by $T$ is a homomorphism of abelian groups for all $R$-modules $M, N$. For any $R$ module $M$, choose a free (or projective) resolution $\varepsilon: F \rightarrow M$ and consider the chain complex $T F$ of abelian groups obtained by applying $T$ to $F$ termwise. Now $T$, being additive, preserves chain homotopies; so we can apply the uniqueness theorem for resolutions (I.7.5) to deduce that the complex $T F$ is independent, up to canonical homotopy equivalence, of the choice of resolution. Passing to homology, we obtain groups $H_n(T F)$ which depend only on $T$ and $M$ (up to canonical isomorphism).

This construction is of no interest, of course, if $T$ is an exact functor; for then the augmented complex
$$
\cdots \rightarrow T F_1 \rightarrow T F_0 \rightarrow T M \rightarrow 0
$$
is acyclic, so that $H_n(T F)=0$ for $n>0$ and $H_0(T F)=T M$. Thus we can regard the groups $H_n(T F)$ in the general case as a measure of the failure of $T$ to be exact.



\section{Spectral Sequences}




\section{Abelian categories}

\section{Derived functors}

\section{Derived categories}





\chapter{Group (Cohomology) Theory} 


\section{Actions}

\section{Representations}


\paragraph*{Group Ring}

Let $G$ be a group, written multiplicatively. Let $\mathbb{Z} G$ be the free $\mathbb{Z}$-module generated by the elements of $G$. The multiplication in $G$ extends uniquely to a $\mathbb{Z}$-bilinear product $\mathbb{Z G} \times \mathbb{Z G} \rightarrow \mathbb{Z G}$; this makes $\mathbb{Z G}$ a ring, called the \textbf{integral group ring} of $G$.

Note that $G$ is a subgroup of the group $(\mathbb{Z} G)^*$ of units of $\mathbb{Z G}$ 
\begin{theo}[Universal property]
Given a ring $R$ and a group homomorphism $f: G \rightarrow R^*$, there is a unique extension of $f$ to a ring homomorphism $\mathbb{Z G} \rightarrow R$. Thus we have the "adjunction formula"
    $$
    \operatorname{Hom}_{\text {(rings) }}(\mathbb{Z} G, R) \approx \operatorname{Hom}_{\text {(groups) }}\left(G, R^*\right) .
    $$
\end{theo}

A \textbf{(left) $\mathbb{Z} G$-module}, or $G$-module, consists of an abelian group $A$ together with a homomorphism from $\mathbb{Z} G$ to the ring of endomorphisms of $A$. By the universal property, $G$-module is simply an abelian group $A$ together with an action of $G$ on $A$. For example, one has for any $A$ the trivial module structure, with $g a=a$ for $g \in G, a \in A$.

One way of constructing $G$-modules is by linearizing permutation representations. More precisely, if $X$ is a $G$-set (i.e., a set with $G$-action), then one forms the free abelian group $\mathbb{Z X}$ (also denoted $\mathbb{Z}[X]$ ) generated by $X$ and one extends the action of $G$ on $X$ to a $\mathbb{Z}$-linear action of $G$ on $\mathbb{Z} X$. The resulting $G$-module is called a permutation module. In particular, one has a permutation module $\mathbb{Z}[G / H]$ for every subgroup $H$ of $G$, where $G / H$ is the set of cosets $g H$ and $G$ acts on $G / H$ by left translation.

\begin{prop}
Let $X$ be a free $G$-set and let $E$ be a set of representatives for the $G$-orbits in $X$. Then $\mathbb{Z}X$ is a free $\mathbb{Z}G$-module with basis $E$.
\end{prop}


\section{Co-invariants}
If $G$ is a group and $M$ is a $G$-module, then the group of co-invariants of $M$, denoted $M_G$, is defined to be the quotient of $M$ by the additive subgroup generated by the elements of the form $g m-m\left(g \in G, m \in M\right.$ ). Thus $M_G$ is obtained from $M$ by "dividing out" by the $G$-action. (The name "co-invariants" comes from the fact that $M_G$ is the largest quotient of $M$ on which $G$ acts trivially, whereas $M^G$, the group of invariants, is the largest submodule of $M$ on which $G$ acts trivially.) In view of exercise $1 \mathrm{a}$ of $\$ I .2$, we can also describe $M_G$ as $M / I M$, where $I$ is the augmentation ideal of $\mathbb{Z} G$ and $I M$ denotes the set of all finite sums $\sum a_i b_i\left(a_i \in I, b_i \in M\right)$.
Still another description of $M_G$ is given by:
$$
M_G \approx \mathbb{Z} \otimes_{\mathbb{Z} G} M .
$$

Here, in order for the tensor product to make sense, we regard $\mathbb{Z}$ as a right $\mathbb{Z} G$-module (with trivial $G$-action). To prove 2.1 , note that in $\mathbb{Z} \otimes_{\mathbb{Z} G} M$ we have the identity $1 \otimes g m=1 \cdot g \otimes m=1 \otimes m ;$ hence there is a map $M_G \rightarrow$ $\mathbb{Z} \otimes_{\mathbb{Z} G} M$ given by $\bar{m} \mapsto 1 \otimes m$, where $\bar{m}$ denotes the image in $M_G$ of an element $m \in M$. On the other hand, using the universal property of the tensor product, we can define a map $\mathbb{Z} \otimes_{\mathbb{Z} G} M \rightarrow M_G$ by $a \otimes m \mapsto a \bar{m}$. These two maps are inverses of one another.

In view of 2.1 and standard properties of the tensor product, we immediately obtain the following two properties of the co-invariants functor:




\section{An spectral sequence for group cohomology}

Suppose that $X$ is a simplicial set and $x_i$ are simplicial subsets such that $X=U X_i$. Then, setting $X_{i j}=X_i \cap X_j$ (etc.) we'11 obviously have for the realisations: $|x|=U\left|x_i\right|,\left|x_i\right| \cap\left|x_j\right|=\left|x_{i j}\right|, \ldots$ Let's suppose that the set of indices is linearly ordered. Consider the following bicomplex:
$$ K = \longrightarrow \underset{i<j<k}{\oplus} C_*\left(x_{i j k}\right) \longrightarrow \underset{i<j}{\oplus} C_*\left(x_{i j}\right)\longrightarrow \underset{i}{\oplus} C_*\left(x_{i}\right) $$


Here by a bicomplex we understand a bicomplex in the sense of Grothendieck [9] i.e. the differentials $d_1$ and $d_2$ commute. (The sign in this approach appears in the definition of the total differentials). The vertical arrows of the bicomplex map $C_*\left(x_i \cdots_i\right)$ into $\underset{k=0}{q} C_*\left(x_{i_0} \ldots \hat{i}_k \ldots i_q\right)$, the mapping into the kth summand differing $k=0$ by a sign $(-1)^k \quad$ from the natural embedding.

The first spectral sequence of this bicomplex degenerates and yields an isomorphism $H_{\star}(K) \cong H_{\star}(X)$. (Moreover this isomorphism is induced by the canonical map $K \rightarrow C_*(X)$). The second spectral sequence gives us a functorial spectral sequence of the first quadrant, whose limit equals $H_*(X)$, while its differential $d r$ has bidegree $(r-1,-r)$ and its $E^1$-term looks as follows: $$E_{p q q}^1=\underset{i_0<\ldots<i_q}{\otimes} H_p\left(x_{i_0} \ldots i_q\right)$$

Suppose $G$ is a group. Let $X_G$ denote the simplicial set (and its geometric realisation), whose p-simplices are sequences $\left(g_0, \ldots, g_p\right)$ of elements of $G$, with the usual faces and degeneracies. This space $X_G$ is contractible by (1.2). The group $G$ acts from the right on $X_G$ and this action is obviously free, hence $B G=X_G / G$ is a classifying space of $G$. The complex $C_*(B G)=C_*(G)$ coincides with the usual complex associated with $G$. Moreover $C_*(G)=C_*\left(X_G\right) \otimes_G Z$.

If $H$ is a subgroup of $G$, then $X_G / H$ is a classifying space for $H$ and hence $B H=X_H / H \rightarrow X_G / H$ is a homotopy equivalence. In particular $C_*(H)+C_*\left(X_G\right) \otimes_H \mathbb{Z}=C_*\left(X_G\right) \otimes_G Z|G / H|$ is a homotopy equivalence.

(2.3) The spectral sequence associated with a family of subgroups.

Suppose $G$ is a group and $G_1, \ldots, G_n$ are subgroups. Then $B G_i$ may be viewed as a simplicial subset of $B G$ and $B G_i \cap B G_j=B\left(G_i \cap G_j\right)$.. Denote $U B G_i$ by $X$ and consider the spectral sequence of the covering $X=U B G_i$. Along with the bicomplex $K$ introduced in (2.1) we also consider the following bicomplex:

$$K' = \underset{i<j<k}{\oplus} C_*\left(X_G\right) \otimes_G Z\left[G / G_{i j k}\right] \longrightarrow \underset{i<j}{\oplus} C_*\left(X_G\right) \otimes_G Z\left[G / G_{i j}\right] \longrightarrow \underset{i<j}{\oplus} C_*\left(X_G\right) \otimes_G Z\left[G / G_{i}\right] $$

There is a natural mapping of bicomplexes $K+K^{\prime}$ and because of (2.2) this mapping induces an isomorphism of second spectral sequences so that $H_{\star}(X)=H_{\star}(K)=H_*\left(K^{\prime}\right)$. The first spectral sequence of $K^{\prime}$ looks as follows: $E_{*, q}^1=C_*\left(X_G\right) \otimes_G H_q(L)$, where $L$ is the following complex of left G-modules:
$$
\oplus \mathbb{Z}\left[G / G_i\right]+\oplus \mathbb{Z}\left[G / G_{i j}\right]+\oplus \mathbb{Z}\left[G / G_{i j k}\right]+\ldots
$$


\begin{prop}
If $G_1, \ldots, G_n$ are subgroups of $G$, there exists a fuctorial spectral sequence of the first quadrart, the $E^2$ term of which looks like: $E_{p q}^2=H_p\left(G, H_q(L)\right)$, where $L$ is the complex defined above. It converges to $H_{\star}\left(U B G_j\right)$ and the differential $d^r$ has bidegree $(-r, r-1)$.   
\end{prop}

(2.5) In the notations of (2.3), let $Z(G,\{G\})$ be the simplicial set whose non-degenerate p-simplices are sequences $\left(\bar{g}_0, \ldots, \bar{g}_p\right)$, where $\bar{g}_i \varepsilon G / G_{k_i}, k_0<\ldots<k_p$, and the $\bar{g}_i$ are such that there is $g \in G$ with $\vec{g}_i=g \bmod G_{k_i}$ for all i. (If one covers $G$ by the right cosets of the $G_i$, then $Z\left(G_g\left\{G_i\right\}\right)$ is the nerve of this covering.) It is easy to see that the geometric realization of this simplicial set is an ordered simplicial space and that the complex $L=L\left(G,\left\{G_i\right\}\right)$ equals the (ordered) simplicial complex [7] of this simplicial space, or in other words, the complex $L$ equals the normalised complex of the simplicial set $Z\left(G,\left\{G_i\right\}\right)$. In particular, $H_*(L)=H_*\left(Z\left(G,\left\{G_i\right\}\right)\right)$.

(2.6) Remark. It may be shown easily that the space $Z\left(G,\left\{G_i\right\}\right)$, is homotopy equivalent to Volodin's space $V\left(G,\left\{G_i\right\}\right)$, but we will not need this fact.








\chapter{(General) Module Theory}

\section{Linear Algebra}

