\part{Topics of Algebra}

\chapter{Category Theory}

Reference \cite{adamekAbstractConcreteCategories}


\begin{example}
    \begin{enumerate}
        \item For a topological spaces, the category of open sets with inclusions as morphisms. The opposite of this category, denoted by $\mathfrak{U}$, is used in sheaf theory.
        \item If \ca and \cb are preordered sets, then functors between them are monotone maps.
    \end{enumerate}
\end{example}

% A functor $F: \mathcal{A} \rightarrow \mathcal{B}$ is called an \textbf{isomorphism} provided that there is a functor $G: \mathcal{B} \rightarrow \mathcal{A}$ such that $G \circ F=i d_{\mathcal{A}}$ and $F \circ G=i d_{\mathcal{B}}$. The categories $\mathcal{A}$ and $\mathcal{B}$ are said to be \textbf{isomorphic}. \\
% Note that $G$ is uniquely determined by $F$. It will be denoted by $F^{-1}$ and called the \textbf{inverse} of $F$.\\

\noindent Let $F: \mathcal{A} \rightarrow \mathcal{B}$ be a functor.
\begin{enumerate}
    \item $F$ is \textbf{faithful} provided that all the \textbf{hom-set restrictions}
    $$
    F: \operatorname{hom}_{\mathcal{A}}\left(A, A^{\prime}\right) \rightarrow \operatorname{hom}_{\mathcal{B}}\left(F A, F A^{\prime}\right)
    $$
    are injective.
    \item $F$ is \textbf{full} if all hom-set restrictions are surjective.
    \item $F$ is an \textbf{embedding} if and it is faithful and injective on the class of objects.
    \item $F$ is \textbf{essentially surjective} if for every object $B$ of \cb, there is an object $A$ of \ca such that $F A$ is isomorphic to $B$. 
    \item If $F$ is essentially surjective and fully faithful, it is called an \textbf{equivalence of categories}, and \ca and \cb are said to be \textbf{equivalent}.
\end{enumerate}

Let $F, G: \mathcal{A} \rightarrow \mathcal{B}$ be functors. A natural transformation $\tau$ from $F$ to $G$ (denoted by $\tau: F \rightarrow G$ or $F \xrightarrow{\tau} G$ ) is a function that assigns to each $\mathcal{A}$-object $A$ a $\mathcal{B}$-morphism $\tau_A: F A \rightarrow G A$ in such a way that the following naturality condition holds: for each A-morphism $A \xrightarrow{f} A^{\prime}$, the diagram
$
\begin{tikzcd}
FA \arrow[r, "\tau_A"] \arrow[d, "Ff"'] & GA \arrow[d, "Gf"] \\
FA' \arrow[r, "\tau_{A'}"'] & GA'
\end{tikzcd}
$ commutes.\\
A natural transformation $F \xrightarrow{\tau} G$ whose components $\tau_A$ are isomorphisms is called a \textbf{natural isomorphism} from $F$ to $G$, and $F$ and $G$ are said to be \textbf{naturally isomorphic}, denoted by $F \cong G$.


\begin{example}
    \begin{enumerate}
        \item Let $U: \operatorname{Grp} \rightarrow$ Set be the forgetful functor, and let $S: \operatorname{Grp} \rightarrow$ Set be the "squaring-functor", defined by $S(G \xrightarrow{f} H)=G^2 \xrightarrow{f^2} H^2$. For each group $G$, its multiplication is a function $\tau_G: G^2 \rightarrow G$. The family $\tau=\left(\tau_G\right)$ is a natural transformation from $S$ to $U$. The naturality condition simply means that $f(x \cdot y)=f(x) \cdot f(y)$ for any group homomorphism $G \xrightarrow{f} H$ and any $x, y \in G$. Thus "multiplication" in groups can be regarded as a natural transformation. Similar for other structures.
        \item Let $(\,\hat{} \,):$ Vec $\rightarrow$ Vec be the second-dual functor for vector spaces, then $\tau_V: V \rightarrow \hat{\hat{V}}$, defined by $\left(\tau_V(x)\right)(f)=f(x)$, yield a natural transformation $i d{ }_{\mathrm{Vec}} \xrightarrow{\tau}(\,\hat{}\,)$. It becomes a natural isomorphism when restricted to finite-dimensional vector spaces.
        \item The assignment of the Hurewicz homomorphism $\pi_n(X) \rightarrow H_n(X)$ to each topological space $X$ is a natural transformation from the $n$-th homotopy functor $\pi_n:$ Top $\rightarrow$ Grp to the $n$-th homology functor $H_n:$ Top $\rightarrow$ Grp.
        \item If $B \xrightarrow{f} C$ is an $\mathcal{A}$-morphism, then
        $
        \operatorname{hom}_{\mathcal{A}}(C,-) \xrightarrow{\tau_f} \operatorname{hom}_{\mathcal{A}}(B,-),
        $
        defined by $\tau_f(g)=g \circ f$, and
        $
        \operatorname{hom}_{\mathcal{A}}(-, B) \xrightarrow{\sigma_f} \operatorname{hom}_{\mathcal{A}}(-, C) \text {, }
        $
        defined by $\sigma_f(g)=f \circ g$, are natural transformations.
        \item (Good definitions of extension) Let $F:$ Set $\rightarrow$ Vec be a functor that assigns to each set $X$ a vector space $F X$ with basis $X$, and to each function $X \xrightarrow{f} Y$ the unique linear extension $F X \xrightarrow{F f} F Y$ of $f$. This actually is not a correct definition of a functor, since there are many different vector spaces with the same basis. However, the definition is "correct up to natural isomorphism". Whenever we choose, for each set $X$, a specific vector space $F X$ with basis $X$, we do obtain a functor $F:$ Set $\rightarrow$ Vec (since the above condition determines the action of $F$ on functions uniquely). Furthermore, any two functors that are obtained in this way are naturally isomorphic.
        \item For any 2-element set $A$, hom $(A,-)$ is naturally isomorphic to the squaring functor $S^2[3.20(10)]$ and hom $(-, A)$ is naturally isomorphic to the contravariant power-set functor $\mathcal{Q}[3.20(9)]$. If $B$ is isomorphic to $A$, then hom $(A,-)$ and hom $(B,-)$ are naturally isomorphic with those functors, the converse is true.
    \end{enumerate}
\end{example}

\begin{prop}
    A functor $\mathcal{A} \xrightarrow{F} \mathcal{B}$ is an equivalence if and only if there exists a functor $\mathcal{B} \xrightarrow{G} \mathcal{A}$ such that $i d_{\mathcal{A}} \cong G \circ F$ and $F \circ G \cong i d_{\mathcal{B}}$.
\end{prop}


\section{Limits and colimits}
Let $A \stackrel{\text { f }}{\underset{g}{\rightrightarrows}} B$ be a pair of morphisms. A morphism $E \xrightarrow{e} A$ is called an equalizer of $f$ and $g$ provided that the following conditions hold:
(1) $f \circ e=g \circ e$,
(2) for any morphism $e^{\prime}: E^{\prime} \rightarrow A$ with $f \circ e^{\prime}=g \circ e^{\prime}$, there exists a unique morphism $\bar{e}: E^{\prime} \rightarrow E$ such that $e^{\prime}=e \circ \bar{e}$, i.e., such that the triangle
\begin{tikzcd}
    E' \arrow[d, "\overline{e}"'] \arrow[dr, "e'"] & \\
    E \arrow[r, "e"'] & A \arrow[r, shift left, "f"] \arrow[r, shift right, "g"'] & B
    \end{tikzcd} commutes.
    

A \textbf{source} is a pair $\left(A,\left(f_i\right)_{i \in I}\right)$ consisting of an object $A$ and a family of morphisms $f_i: A \rightarrow A_i$ with domain $A$, indexed by some class $I$. 

%A source $\mathcal{P}=\left(P \xrightarrow{p_i} A_i\right)_I$ is called a product provided that for every source $\mathcal{S}=$ $\left(A \xrightarrow{f_i} A_i\right)_I$ with the same codomain as $\mathcal{P}$ there exists a unique morphism $A \xrightarrow{f} P$ with $\mathcal{S}=\mathcal{P} \circ f$. A product with codomain $\left(A_i\right)_I$ is called a product of the family $\left(A_i\right)_I$.

\begin{prop}
    A category has finite products if and only if it has terminal objects and products of pairs of objects.
    A category that has products for all class-indexed families must be thin.
A small category has products if and only if it is equivalent to a complete lattice.
\end{prop}

A \textbf{diagram} in a category $\mathcal{A}$ is a functor $D: \mathbf{I} \rightarrow \mathcal{A}$ with codomain $\mathcal{A}$. The domain, $\mathbf{I}$, is called the \textbf{scheme} of the diagram. A diagram with a small (or finite) scheme is said to be \textbf{small} (or finite).\\
An A-source $\left(A \xrightarrow{f_i} D_i\right)_{i \in O b(\mathrm{I})}$ is said to be \textbf{natural for $D$} provided that for each I-morphism $i \xrightarrow{d} j$, the triangle 
\begin{tikzcd}
    A \arrow[d, "f_i"'] \arrow[dr, "f_j"] & \\
    D_i \arrow[r, "Dd"'] & D_j 
\end{tikzcd} commutes.\\
Equivalently, natural sources can be regarded as natural transformations from constant functors $C: \mathbf{I} \rightarrow \mathbf{A}$ to the functor $D$.\\
A \textbf{limit} of $D$ is a natural source $\left(L \xrightarrow{\ell_i} D_i\right)$ for $D$ with the \textbf{universal property} that for each natural source $\left(A \xrightarrow{f_i} D_i\right)$ there exists a unique morphism $f: A \rightarrow L$ with $f_i=$ $\ell_i \circ f$ for each $i \in O b(\mathbf{I})$.

\begin{enumerate}
    \item Every source is natural for a diagram with discrete scheme. Products are limits of diagrams with discrete scheme. An object, considered as an empty source, is a limit of the empty diagram if and only if it is a terminal object.

    \item For A-morphisms $A \underset{g}{\stackrel{f}{\rightrightarrows}} B$, considered as a diagram $D$ with scheme $\bullet \Rightarrow \bullet$, a source $(A \stackrel{e}{\longleftarrow} C \xrightarrow{h} B)$ is natural provided that $g \circ e=h=f \circ e$.\\
    $C \xrightarrow{e} A$ is an equalizer of $A \xrightarrow[g]{\stackrel{f}{\longrightarrow}} B$ if and only if the source $(A \stackrel{e}{\leftarrow} C \xrightarrow{f \circ e} B)$ is a limit of $D$. 

    \item A poset $\mathbf{I}$ is \textbf{down-directed} if every pair of elements has a lower bound. Limits of diagrams with scheme $\mathbf{I}$ are called \textbf{projective} (or \textbf{inverse}) limits. 
\end{enumerate}

\begin{prop}
If $\mathcal{L}=\left(L \xrightarrow{\ell_i} D_i\right)_{i \in O b(\mathbf{I})}$ is a limit of $D: \mathbf{I} \rightarrow \mathbf{A}$, then
    \begin{enumerate}
        \item for each limit $\mathcal{K}=\left(K \xrightarrow{k_i} D_i\right)_{i \in O b(\mathrm{I})}$ of $D$, there exist an isomorphism $K \xrightarrow{h} L$ with $\mathcal{K}=\mathcal{L} \circ h$,
        \item for each isomorphism $A \xrightarrow{h} L$, the source $\mathcal{L} \circ h$ is a limit of $D$.
    \end{enumerate}
\end{prop}














\section{Adjoint functors}







\section{Concrete categories}

A way to talk of \textit{low level structures} present on the objects of a category. Often it is easier to work with less structures, and there results like Yoneda's lemma that show us that it is possible to restrict our study to them.\\

Let \cc be a category. A \textbf{concrete category} over \cc is a category $\mathcal{A}$ together wih a faithful functor $U: \mathcal{A} \rightarrow \mathcal{C}$, called the \textbf{forgetful} (or underlying) functor of the concrete category. $\mathcal{C}$ is called the \textbf{base category}. A concrete category over Set is called a \textbf{construct}.\\
The category of groups (or topological spaces, rings, etc.), with the forgetful functor to Set, is a construct.\\

\begin{enumerate}
    \item A \textbf{structured arrow} with domain $X$ is a pair $(f, A)$ consisting of an A-object $A$ and an X-morphism $X \xrightarrow{f}|A|$,
    \item if $(f, A)$ is \textbf{generating} provided that for any pair of A-morphisms $r, s: A \rightarrow B$ the equality $r \circ f=s \circ f$ implies that $r=s$,
    \item and this $(f, A)$ is called \textbf{extremally generating} (resp. \textbf{concretely generating}) provided that each A-monomorphism (resp. A-embedding) $m: A^{\prime} \rightarrow A$, through which $f$ factors (i.e., $f=m \circ g$ for some $\mathbf{X}$-morphism $g$ ), is an $\mathbf{A}$-isomorphism.
    \item In a construct, an object $A$ is (extremally resp. concretely) generated by a subset $X$ of $|A|$ provided that the inclusion map $X \hookrightarrow|A|$ is (extremally resp. concretely) generating.
\end{enumerate}

\begin{prop}
    In a concrete category $\mathbf{A}$ over $\mathbf{X}$ the following hold for each structured arrow $f: X \rightarrow|A|:$
    \begin{enumerate}
        \item If $(f, A)$ is extremally generating, then $(f, A)$ is concretely generating.
        \item If $(f, A)$ is concretely generating, then $(f, A)$ is generating.
        \item If $X \xrightarrow{f}|A|$ is an $\mathbf{X}$-epimorphism, then $(f, A)$ is generating.
        \item If $X \xrightarrow{f}|A|$ is an extremal epimorphism in $\mathbf{X}$, and if $||$ preserves monomorphisms, then $(f, A)$ is extremally generating.
    \end{enumerate}
\end{prop}

\begin{example}
    \begin{enumerate}
        \item If an abstract category $\mathbf{A}$ is considered to be concrete over itself via the identity functor, then an A-morphism $A \xrightarrow{f} B$, considered as a structured arrow $(f, B)$, is generating (resp. extremally or concretely generating) if and only if $f$ is an epimorphism (resp. an extremal epimorphism). That is,
        $$
        \operatorname{Gen}(\mathbf{A})=\operatorname{Epi}(\mathbf{A}) \text { and } \operatorname{ExtrGen}(\mathbf{A})=\operatorname{ConcGen}(\mathbf{A})=\operatorname{ExtrEpi}(\mathbf{A})
        $$
        \begin{enumerate}
            \item In Vec, Grp, Sgr, Rng, and other algebraic constructs, the concepts of concrete generation and of extremal generation coincide with the familiar (non-categorical) concept of generation.
            In the constructs Sgr and Rng the inclusion map $\mathbb{Z} \hookrightarrow \mathbb{Q}$ is generating, but is not concretely generating [cf. 7.40(5)].

            \item In the construct $\mathbf{A}=$ Top we have
            $$
            \begin{aligned}
            & \text { ConcGen(A) }=\operatorname{Gen}(\mathbf{A})=\text { Surjective maps, and } \\
            & \operatorname{ExtrGen}(\mathbf{A})=\text { Surjective maps with discrete codomain. }
            \end{aligned}
            $$

            \item In the construct $\mathbf{A}=$ Haus we have
            $$
            \begin{aligned}
            \operatorname{Gen}(\mathbf{A}) & =\text { Dense maps } \\
            \text { ConcGen(A) } & =\text { Surjective maps, and } \\
            \text { ExtrGen(A) } & =\text { Surjective maps with discrete codomain. }
            \end{aligned}
            $$
        \item $A \xrightarrow{f} B$ is an epimorphism if and only if $(f, B)$ is generating.
\item If $(f, B)$ is extremally generating and the forgetful functor preserves monomorphisms, then $A \xrightarrow{f} B$ is an extremal epimorphism.
\item If $A \xrightarrow{f} B$ is an extremal epimorphism, then $(f, B)$ is concretely generating.    
        \end{enumerate}
    \end{enumerate}
\end{example}

A \textbf{universal arrow} over an $\mathbf{X}$-object $X$ is a structured arrow $X \xrightarrow{u}|A|$ with domain $X$ such that, for each structured arrow $X \xrightarrow{f}|B|$ with domain $X$, there exists a unique $A$-morphism $\hat{f}: A \rightarrow B$ such that the triangle 
\begin{tikzcd}
    X \arrow[r, "u"] \arrow[dr, "f"'] &  {|A|} \arrow[d, "\overline{f}"]\\
    & {|B|} 
\end{tikzcd} commutes. The pair $(u,A)$ is called a \textbf{free object}.

\begin{example}
\begin{enumerate}
    \item In a construct, an object $A$ is a free object over the empty set if and only if $A$ is an initial object, and over a singleton set if and only if $A$ represents the forgetful functor.
    \item In the construct Vec each object is a free object over any basis for it.
    \item In the constructs Top and Pos the free objects are precisely the discrete ones.
    \item In the construct $\mathbf{A b}$ free objects over $X$ are the free abelian groups generated by $X$.
    Similarly, the familiar free group generated by a set $X$ is a free object over $X$ in the construct Grp.
    \item To construct a universal arrow in (Ban, $O$ ) over a set $X$, let $\ell_1(X)$ be the subspace of the vector space $K^X$ consisting of all $r=\left(r_x\right)_{x \in X}$ in $K^X$ whose norm $\|r\|=$ $\sum_{x \in X}\left|r_x\right|$ is finite. Then $\ell_1(X)$ is a Banach space. Define $X \xrightarrow{u} O\left(\ell_1(X)\right)$ at $y$ by the Dirac function $u(y)=\left(\delta_{y x}\right)_{x \in X}$. Then $\left(u, \ell_1(X)\right)$ is a universal arrow over $X$. Observe, for comparison, that for the construct (Ban, $U$ ) the only set having a universal arrow is the empty set, and that for the construct Ban $\mathrm{B}_{\mathrm{b}}$ the only sets having universal arrows are the finite ones.
\end{enumerate}
\end{example}

\begin{prop}
    \begin{enumerate}
        \item Every universal arrow is extremally generating.
        \item Any two universal arrows with domain $X$ are isomorphic. Conversely, if $X \xrightarrow{u}|A|$ is a universal arrow and $A \xrightarrow{k} A^{\prime}$ is an $\mathbf{A}$-isomorphism, then $X \xrightarrow{k o u}\left|A^{\prime}\right|$ is also universal.
        \item If a concrete category $\mathbf{A}$ over $\mathbf{X}$ has free objects, then an $\mathbf{A}$-morphism is an $\mathbf{A}$-monomorphism if and only if it is an $\mathbf{X}$-monomorphism.
        \item If a construct $\mathbf{A}$ has a free object over a singleton set, then the monomorphisms in $\mathbf{A}$ are precisely those morphisms that are injective functions.
    \end{enumerate}
\end{prop}

A concrete category over $\mathbf{X}$ is said to have free objects provided that for each $\mathbf{X}$-object $X$ there exists a universal arrow over $X$.\\
The constructs Vec, Grp, Ab, Mon, Sgr, Alg $(\Omega)$, Top, Pos, and $($ Ban,$O)$ have free objects; but the constructs Ban$_b$.

A functor $F: \mathcal{A} \rightarrow$ Set is called representable (by an $\mathcal{A}$-object $A$ ) provided that $F$ is naturally isomorphic to the hom-functor $\operatorname{hom}(A,-): \mathcal{A} \rightarrow$ Set. Note that objects that represents the same functor are isomorphic.

\begin{example}
    \begin{enumerate}
        \item Forgetful functors are often representable. For example,
        (a) Vec $\rightarrow$ Set is represented by the vector space $\mathbb{R}$,
        (b) $\operatorname{Grp} \rightarrow$ Set is represented by the group of integers $\mathbb{Z}$,
        (c) Top $\rightarrow$ Set is represented by any one-point topological space.
        \item The underlying functor $U$ for the construct Ban [5.2(3)] is not representable (see Exercise 10J). However, the faithful unit ball functor $O: \operatorname{Ban} \rightarrow$ Set is represented in the complex case by the Banach space $\mathbb{C}$ of complex numbers.
    \end{enumerate}
\end{example}

\begin{prop}[Representative of Constructs]
    For constructs $(\mathcal{A}, U)$ the forgetful functor is represented by an object $A$ if and only if $A$ is a free object over a singleton set [see Definition 8.22(2)]. This provides many additional examples of representations.
\end{prop}

For small categories $\mathcal{A}$ and $\mathcal{B}$ the \textbf{functor category} $[\mathcal{A}, \mathcal{B}]$ has as objects all functors from $\mathcal{A}$ to $\mathcal{B}$, as morphisms from $F$ to $G$ all natural transformations from $F$ to $G$, as identities the identity natural transformations, and as composition the (horizontal) composition of natural transformations.

\begin{theo}[uniqueness of representations]
    For any functor $F: \mathcal{A} \rightarrow$ Set, any $\mathcal{A}$-object $A$ and any element $a \in F(A)$, there exists a unique natural transformation $\tau: \operatorname{hom}(A,-) \rightarrow F$ with $\tau_A\left(i d_A\right)=a$.
\end{theo}

\begin{coro}[Yoneda Lemma]
    If $F: \mathcal{A} \rightarrow$ Set is a functor and $A$ is an $\mathcal{A}$-object, then the following function
    $$
    Y:[\operatorname{hom}(A,-), F] \rightarrow F(A) \text { defined by } Y(\sigma)=\sigma_A\left(i d_A\right),
    $$
    is a bijection (where $[\operatorname{hom}(A,-), F]$ is the set of all natural transformations from hom $(A,-)$ to $F$ ).
    
\end{coro}

\begin{coro}[Yoneda Embedding]
    For any category $\mathcal{A}$, the functor $E: \mathcal{A} \rightarrow\left[\mathcal{A}^{\mathrm{op}} Set \right]$, defined by
$$
E(A \xrightarrow{f} B)=\operatorname{hom}(-, A) \xrightarrow{\sigma_f} \operatorname{hom}(-, B) \text {, }
$$
where $\sigma_f(g)=f \circ g$, is a full embedding.
\end{coro}


\chapter{Homological Algebra}

In homological algebra one constructs homological invariants of algebraic objects by the following process, or some variant of it:

Let $R$ be a ring and $T$ a covariant additive functor from $R$-modules to abelian groups. Thus the map $\operatorname{Hom}_R(M, N) \rightarrow \operatorname{Hom}_{\mathbf{z}}(T M, T N)$ defined by $T$ is a homomorphism of abelian groups for all $R$-modules $M, N$. For any $R$ module $M$, choose a free (or projective) resolution $\varepsilon: F \rightarrow M$ and consider the chain complex $T F$ of abelian groups obtained by applying $T$ to $F$ termwise. Now $T$, being additive, preserves chain homotopies; so we can apply the uniqueness theorem for resolutions (I.7.5) to deduce that the complex $T F$ is independent, up to canonical homotopy equivalence, of the choice of resolution. Passing to homology, we obtain groups $H_n(T F)$ which depend only on $T$ and $M$ (up to canonical isomorphism).

This construction is of no interest, of course, if $T$ is an exact functor; for then the augmented complex
$$
\cdots \rightarrow T F_1 \rightarrow T F_0 \rightarrow T M \rightarrow 0
$$
is acyclic, so that $H_n(T F)=0$ for $n>0$ and $H_0(T F)=T M$. Thus we can regard the groups $H_n(T F)$ in the general case as a measure of the failure of $T$ to be exact.



\chapter{Categorías Aditivas}


Una \textbf{categoria pre-aditiva} es uma categoria tal que os $\operatorname{Hom}(A,B)$ possuem estrutura de grupo abeliano, e composição de morfismos é bilinear. Em particular, existem morfismos nulos.\\
Um funtor entra categorias preaditivas é \textbf{aditivo} se os mapas $F:\operatorname{Hom}(A,B)\rightarrow \operatorname{Hom}(F(A),F(B))$ são homomorfismos de grupos. 

\begin{prop}
	Numa categoria pre-aditiva, tudo produto finito é um coproduto, e vice-versa (chamado de \textbf{biproduto}). 
\end{prop}

Uma \textbf{categoria aditiva} é uma categoria pre-aditiva que admite biprodutos finitos. Em particular, os biprodutos vazios são objetos zero.\\

Uma \textbf{categoria abeliana} é uma categoria aditiva tal que:
\begin{enumerate}
	\item Todo morfismo possui núcleo e conúcelo,
	\item todo monomorfismo (resp. epimorfismo) é o kernel (resp. cokernel) de um morfismo.
\end{enumerate}


Um \textbf{complexo} numa categoria aditiva $\mathcal{C}$ é uma sequência de objetos $\{X_j\}$

\section{Objetos projetivos e injetivos}

Um objeto $Q$ numa categoria é \textbf{injetivo} se para todo monomorfismo $f:X\rightarrow Y$ e todo morfismo $g:X \rightarrow Q$ existe um morfismo $h: Y\rightarrow Q$ tal que $h\circ f =g$.\\
Uma categoria \textbf{tem suficientes injetivos} se para todo objeto $X$ existe um monomorfismo $X\rightarrow Q$, com $Q$ injetivo.


Um objeto $P$ numa categoria é \textbf{injetivo} se para todo epimorfismo $e:E\rightarrow X$ e todo morfismo $f:P \rightarrow X$ existe um morfismo $h: P\rightarrow E$ tal que $e\circ h =f$.\\
Uma categoria \textbf{tem suficientes projetivos} se para todo objeto $A$ existe um epimorfismo $P\rightarrow A$, com $P$ projetivo.

\begin{prop}
	Numa categoria abeliana, 
	\begin{itemize}
		\item um objeto é injetivo se e somente se $\operatorname{Hom}(\cdot , Q)$ é exato
		\item um objeto é projetivo se e somente se $\operatorname{Hom}(\cdot , Q)$ é exato.
	\end{itemize}
\end{prop}



\subsection{Resoluciones}


\section{Categoría Trianguladas}



\subsection{La categoría homotópica}





\section{Categorías derivadas}








\section{Spectral Sequences}


\section{Abelian categories}

\section{Derived functors}

\section{Derived categories}




