\part{Topics of Algebra}

%\chapter{Foundations of Mathematics}

\chapter{Category Theory}

Reference \cite{adamekAbstractConcreteCategories,richterCategoriesHomotopyTheory2020} %ritcher


\begin{example}
    \begin{enumerate}
        \item For a topological spaces, the category of open sets with inclusions as morphisms. The opposite of this category, denoted by $\mathfrak{U}$, is used in sheaf theory.
        \item If \ca and \cb are preordered sets, then functors between them are monotone maps.
        \item $f: \mathbb{Z} \rightarrow \mathbb{Q}$ is a monomorphism and epimorphism, but not an isomorphism.
    \end{enumerate}
\end{example}

% A functor $F: \mathcal{A} \rightarrow \mathcal{B}$ is called an \textbf{isomorphism} provided that there is a functor $G: \mathcal{B} \rightarrow \mathcal{A}$ such that $G \circ F=i d_{\mathcal{A}}$ and $F \circ G=i d_{\mathcal{B}}$. The categories $\mathcal{A}$ and $\mathcal{B}$ are said to be \textbf{isomorphic}. \\
% Note that $G$ is uniquely determined by $F$. It will be denoted by $F^{-1}$ and called the \textbf{inverse} of $F$.\\

\noindent Let $F: \mathcal{A} \rightarrow \mathcal{B}$ be a functor.
\begin{enumerate}
    \item $F$ is \textbf{faithful} provided that all the \textbf{hom-set restrictions}
    $$
    F: \operatorname{hom}_{\mathcal{A}}\left(A, A^{\prime}\right) \rightarrow \operatorname{hom}_{\mathcal{B}}\left(F A, F A^{\prime}\right)
    $$
    are injective.
    \item $F$ is \textbf{full} if all hom-set restrictions are surjective.
    \item $F$ is an \textbf{embedding} if and it is faithful and injective on the class of objects.
    \item $F$ is \textbf{essentially surjective} if for every object $B$ of \cb, there is an object $A$ of \ca such that $F A$ is isomorphic to $B$. 
    \item If $F$ is essentially surjective and fully faithful, it is called an \textbf{equivalence of categories}, and \ca and \cb are said to be \textbf{equivalent}.
\end{enumerate}

Let $F, G: \mathcal{A} \rightarrow \mathcal{B}$ be functors. A natural transformation $\tau$ from $F$ to $G$ (denoted by $\tau: F \rightarrow G$ or $F \xrightarrow{\tau} G$ ) is a function that assigns to each $\mathcal{A}$-object $A$ a $\mathcal{B}$-morphism $\tau_A: F A \rightarrow G A$ in such a way that the following naturality condition holds: for each A-morphism $A \xrightarrow{f} A^{\prime}$, the diagram
$
\begin{tikzcd}
FA \arrow[r, "\tau_A"] \arrow[d, "Ff"'] & GA \arrow[d, "Gf"] \\
FA' \arrow[r, "\tau_{A'}"'] & GA'
\end{tikzcd}
$ commutes.\\

\begin{example}
    A typical example of a natural transformation is the connecting homomorphism for singular homology. Consider, for instance, the $n$th singular homology group of a pair of spaces $(X, A)$. There is a connecting homomorphism
$$
\delta: H_n(X, A) \rightarrow H_{n-1}(A) .
$$

This morphism is natural for morphisms of pairs of topological spaces: if $f:(X, A) \rightarrow(Y, B)$ is a continuous map $f: X \rightarrow Y$ with $f(A) \subset B$, then the following diagram commutes:
$$
\begin{aligned}
& H_n(X, A) \xrightarrow{\delta} H_{n-1}(A) \\
& H_n(f) \underset{\downarrow}{H_n(Y, B)} \xrightarrow{\delta} H_{n-1}(B)
\end{aligned}
$$

Thus, $\delta$ is a morphism between functors from the category of pairs of topological spaces to the category of abelian groups, the functor $(X, A) \mapsto H_n(X, A)$ and the functor $(X, A) \mapsto$ $H_{n-1}(A)$. \\

\textit{Anoter example.} Let $\mathcal{E}_G$ be the translation category of a discrete group $G$, as in Example 1.2.3. Then, $\mathcal{E}_G$ is equivalent to the category $[0]$ with one object and one identity morphism. There is a unique functor $P: \mathcal{E}_G \rightarrow[0]$, sending every object to 0 and every morphism to the identity morphism on 0 . We define $F:[0] \rightarrow \mathcal{E}_G$ via $F(0)=e$, where $e$ denotes the neutral element in $G$, and we set $F\left(1_0\right)=e$. The composite $P \circ F$ is the identity functor on the category [0], whereas the composite $F \circ P$ sends any morphism $h: g \rightarrow h g$ in $\mathcal{E}_G$ to $e: e \rightarrow e$.
We define $\eta: F \circ P \Rightarrow \operatorname{Id}_{\mathcal{E}_G}$ by setting
$$
\eta_g: F \circ P(g)=e \rightarrow g=\operatorname{Id}_{\mathcal{E}_G}(g)
$$
to be the morphism $g: e \rightarrow g$ in the translation category. As the diagram
$$
\begin{aligned}
& \eta_g: F \circ P(g)=e \longrightarrow g \\
& F \circ P(h)=e \downarrow \quad \backslash \operatorname{Id}(h)=h \\
& \eta_{h g}: F \circ P(h g)=e \xrightarrow{h g} h g \\
&
\end{aligned}
$$
commutes for all $h, g \in G$ and as $\mathcal{E}_G$ is a groupoid, this defines a natural isomorphism.
\end{example}

A natural transformation $F \xrightarrow{\tau} G$ whose components $\tau_A$ are isomorphisms is called a \textbf{natural isomorphism} from $F$ to $G$, and $F$ and $G$ are said to be \textbf{naturally isomorphic}, denoted by $F \cong G$.\\



\begin{example}
    \begin{enumerate}
        \item Let $U: \operatorname{Grp} \rightarrow$ Set be the forgetful functor, and let $S: \operatorname{Grp} \rightarrow$ Set be the "squaring-functor", defined by $S(G \xrightarrow{f} H)=G^2 \xrightarrow{f^2} H^2$. For each group $G$, its multiplication is a function $\tau_G: G^2 \rightarrow G$. The family $\tau=\left(\tau_G\right)$ is a natural transformation from $S$ to $U$. The naturality condition simply means that $f(x \cdot y)=f(x) \cdot f(y)$ for any group homomorphism $G \xrightarrow{f} H$ and any $x, y \in G$. Thus "multiplication" in groups can be regarded as a natural transformation. Similar for other structures.
        \item Let $(\,\hat{} \,):$ Vec $\rightarrow$ Vec be the second-dual functor for vector spaces, then $\tau_V: V \rightarrow \hat{\hat{V}}$, defined by $\left(\tau_V(x)\right)(f)=f(x)$, yield a natural transformation $i d{ }_{\mathrm{Vec}} \xrightarrow{\tau}(\,\hat{}\,)$. It becomes a natural isomorphism when restricted to finite-dimensional vector spaces.
        \item The assignment of the Hurewicz homomorphism $\pi_n(X) \rightarrow H_n(X)$ to each topological space $X$ is a natural transformation from the $n$-th homotopy functor $\pi_n:$ Top $\rightarrow$ Grp to the $n$-th homology functor $H_n:$ Top $\rightarrow$ Grp.
        \item If $B \xrightarrow{f} C$ is an $\mathcal{A}$-morphism, then
        $
        \operatorname{hom}_{\mathcal{A}}(C,-) \xrightarrow{\tau_f} \operatorname{hom}_{\mathcal{A}}(B,-),
        $
        defined by $\tau_f(g)=g \circ f$, and
        $
        \operatorname{hom}_{\mathcal{A}}(-, B) \xrightarrow{\sigma_f} \operatorname{hom}_{\mathcal{A}}(-, C) \text {, }
        $
        defined by $\sigma_f(g)=f \circ g$, are natural transformations.
        \item (Good definitions of extension) Let $F:$ Set $\rightarrow$ Vec be a functor that assigns to each set $X$ a vector space $F X$ with basis $X$, and to each function $X \xrightarrow{f} Y$ the unique linear extension $F X \xrightarrow{F f} F Y$ of $f$. This actually is not a correct definition of a functor, since there are many different vector spaces with the same basis. However, the definition is "correct up to natural isomorphism". Whenever we choose, for each set $X$, a specific vector space $F X$ with basis $X$, we do obtain a functor $F:$ Set $\rightarrow$ Vec (since the above condition determines the action of $F$ on functions uniquely). Furthermore, any two functors that are obtained in this way are naturally isomorphic.
        \item For any 2-element set $A$, hom $(A,-)$ is naturally isomorphic to the squaring functor $S^2[3.20(10)]$ and hom $(-, A)$ is naturally isomorphic to the contravariant power-set functor $\mathcal{Q}[3.20(9)]$. If $B$ is isomorphic to $A$, then hom $(A,-)$ and hom $(B,-)$ are naturally isomorphic with those functors, the converse is true.
    \end{enumerate}
\end{example}






\section{Limits and colimits}

An object $P$ in a category $\mathcal{C}$ is called projective if for every epimorphism $f: M \rightarrow Q$ in $\mathcal{C}$ and every $p: P \rightarrow Q$, there is a $\xi \in \operatorname{Hom}(P, M)$ with $f \circ \xi=p$, called the \textbf{lift} of $p$ to $M$.\\
Dually, an object $I$ in a category $\mathcal{C}$ is called injective if for every monomorphism $f: U \rightarrow M$ in $\mathcal{C}$ and every $j: U \rightarrow I$, there is a $\zeta \in \operatorname{Hom}(M, I)$ with $\zeta \circ f=j$, called and \textbf{extension} of $j$ to $M$.\\
$\mathcal{C}$ with $[0], \mathcal{C} *[0]$, has 0 as a terminal object and that $[0] * \mathcal{C}$ has 0 as an initial object. The category $\mathcal{C} *[0]$ is the \textbf{inductive cone} with base $\mathcal{C}$, and [0]* $\mathcal{C}$ is the \textbf{projective cone} with base $\mathcal{C}$.


\begin{example}
    \begin{enumerate}
        \item In the category of sets, every set is injective and projective.
        \item In the category of left $R$-modules, a module is projective if and only if it is a direct summand of a free module. A module $M$ is injective if and only if the functor $\operatorname{Hom}_R(-, M)$ is exact.
        %\item An abelian group $A$ is a \textbf{divisible ableian group} if $nA= A$ for every $n \in \mathbb{N}$. Every divisible abelian group is an injective $\mathbb{Z}$-module, an viceversa. 
    \end{enumerate}
\end{example}

\begin{prop}
\begin{enumerate}
    \item If $P$ is a projective object of a category $\mathcal{C}$ and if $i: U \rightarrow P$ is a monomorphism in $\mathcal{C}$ with a retraction $r: P \rightarrow U$, then $U$ is projective. Similarly, if $i: J \rightarrow I$ is a monomorphism with retraction $r: I \rightarrow J$ and $I$ is injective, then $J$ is injective.    
    \item If $q: Q \rightarrow P$ is an epimorphism and if $P$ is projective, then $q$ has a section. Dually, if $j: I \rightarrow J$ is a monomorphism and $I$ is injective, then $j$ has a retraction.
    \item The object $C$ is projective if and only if $\mathcal{C}(C,-): \mathcal{C} \rightarrow$ Sets preserves epimorphisms. 
    \item The object $C$ is injective if and only if $\mathcal{C}(-, C): \mathcal{C}^o \rightarrow$ Sets sends monomorphisms to epimorphisms.
\end{enumerate}
\end{prop}




Let $A \stackrel{\text { f }}{\underset{g}{\rightrightarrows}} B$ be a pair of morphisms. A morphism $E \xrightarrow{e} A$ is called an equalizer of $f$ and $g$ provided that the following conditions hold:
(1) $f \circ e=g \circ e$,
(2) for any morphism $e^{\prime}: E^{\prime} \rightarrow A$ with $f \circ e^{\prime}=g \circ e^{\prime}$, there exists a unique morphism $\bar{e}: E^{\prime} \rightarrow E$ such that $e^{\prime}=e \circ \bar{e}$, i.e., such that the triangle
\begin{tikzcd}
    E' \arrow[d, "\overline{e}"'] \arrow[dr, "e'"] & \\
    E \arrow[r, "e"'] & A \arrow[r, shift left, "f"] \arrow[r, shift right, "g"'] & B
    \end{tikzcd} commutes.
    

A \textbf{source} is a pair $\left(A,\left(f_i\right)_{i \in I}\right)$ consisting of an object $A$ and a family of morphisms $f_i: A \rightarrow A_i$ with domain $A$, indexed by some class $I$. 

%A source $\mathcal{P}=\left(P \xrightarrow{p_i} A_i\right)_I$ is called a product provided that for every source $\mathcal{S}=$ $\left(A \xrightarrow{f_i} A_i\right)_I$ with the same codomain as $\mathcal{P}$ there exists a unique morphism $A \xrightarrow{f} P$ with $\mathcal{S}=\mathcal{P} \circ f$. A product with codomain $\left(A_i\right)_I$ is called a product of the family $\left(A_i\right)_I$.

\begin{prop}
    A category has finite products if and only if it has terminal objects and products of pairs of objects.
    A category that has products for all class-indexed families must be thin.
A small category has products if and only if it is equivalent to a complete lattice.
\end{prop}

A \textbf{diagram} in a category $\mathcal{A}$ is a functor $D: \mathbf{I} \rightarrow \mathcal{A}$ with codomain $\mathcal{A}$. The domain, $\mathbf{I}$, is called the \textbf{scheme} of the diagram. A diagram with a small (or finite) scheme is said to be \textbf{small} (or finite).\\
An A-source $\left(A \xrightarrow{f_i} D_i\right)_{i \in O b(\mathrm{I})}$ is said to be \textbf{natural for $D$} provided that for each I-morphism $i \xrightarrow{d} j$, the triangle 
\begin{tikzcd}
    A \arrow[d, "f_i"'] \arrow[dr, "f_j"] & \\
    D_i \arrow[r, "Dd"'] & D_j 
\end{tikzcd} commutes.\\
Equivalently, natural sources can be regarded as natural transformations from constant functors $C: \mathbf{I} \rightarrow \mathbf{A}$ to the functor $D$.\\
A \textbf{limit} of $D$ is a natural source $\left(L \xrightarrow{\ell_i} D_i\right)$ for $D$ with the \textbf{universal property} that for each natural source $\left(A \xrightarrow{f_i} D_i\right)$ there exists a unique morphism $f: A \rightarrow L$ with $f_i=$ $\ell_i \circ f$ for each $i \in O b(\mathbf{I})$.

\begin{enumerate}
    \item Every source is natural for a diagram with discrete scheme. Products are limits of diagrams with discrete scheme. An object, considered as an empty source, is a limit of the empty diagram if and only if it is a terminal object.

    \item For A-morphisms $A \underset{g}{\stackrel{f}{\rightrightarrows}} B$, considered as a diagram $D$ with scheme $\bullet \Rightarrow \bullet$, a source $(A \stackrel{e}{\longleftarrow} C \xrightarrow{h} B)$ is natural provided that $g \circ e=h=f \circ e$.\\
    $C \xrightarrow{e} A$ is an equalizer of $A \xrightarrow[g]{\stackrel{f}{\longrightarrow}} B$ if and only if the source $(A \stackrel{e}{\leftarrow} C \xrightarrow{f \circ e} B)$ is a limit of $D$. 

    \item A poset $\mathbf{I}$ is \textbf{down-directed} if every pair of elements has a lower bound. Limits of diagrams with scheme $\mathbf{I}$ are called \textbf{projective} (or \textbf{inverse}) limits. 
\end{enumerate}

\begin{prop}
If $\mathcal{L}=\left(L \xrightarrow{\ell_i} D_i\right)_{i \in O b(\mathbf{I})}$ is a limit of $D: \mathbf{I} \rightarrow \mathbf{A}$, then
    \begin{enumerate}
        \item for each limit $\mathcal{K}=\left(K \xrightarrow{k_i} D_i\right)_{i \in O b(\mathrm{I})}$ of $D$, there exist an isomorphism $K \xrightarrow{h} L$ with $\mathcal{K}=\mathcal{L} \circ h$,
        \item for each isomorphism $A \xrightarrow{h} L$, the source $\mathcal{L} \circ h$ is a limit of $D$.
    \end{enumerate}
\end{prop}

%cone over a limit (Ritcher)
% cambio de notacion, no se mencionan diagramas

\begin{prop}
If $G: \mathcal{D} \rightarrow \mathcal{C}$ is another functor and if $\alpha: F \Rightarrow G$ is a natural transformation, then $\alpha$ induces a morphism $\operatorname{colim}_{\mathcal{D}} \alpha \in \mathcal{C}\left(\operatorname{colim}_D F, \operatorname{colim}_{\mathcal{D}} G\right)$. Prove that this turns $\operatorname{colim}_{\mathcal{D}}$ into a functor from $\operatorname{Fun}(\mathcal{D}, \mathcal{C})$ to $\mathcal{C}$.    
\end{prop}

\begin{prop}
If the colimit ( $\operatorname{colim}_{\mathcal{D}} F, \tau$ ) exists for all functors $F: \mathcal{D} \rightarrow \mathcal{C}$, then the functor $\operatorname{colim}_{\mathcal{D}}: \operatorname{Fun}(\mathcal{D}, \mathcal{C}) \rightarrow \mathcal{C}$ is left adjoint to the diagonal functor $\Delta: \mathcal{C} \rightarrow \operatorname{Fun}(\mathcal{D}, \mathcal{C})$, that is, there are natural isomorphisms
$$
\mathcal{C}\left(\operatorname{colim}_{\mathcal{D}} F, C\right) \cong \operatorname{Fun}(\mathcal{D}, \mathcal{C})(F, \Delta(C))
$$
for all functors $F$ and all object $C$ of $\mathcal{C}$.
\end{prop}

%definicion dual de colimite

If you build the colimit over a discrete diagram category (small category $\mathcal{D}$ that has only identity morphisms), then the colimit of a functor $F: \mathcal{D} \rightarrow \mathcal{C}$ is called the \textbf{coproduct} of the $F(D)$ for $D$ an object of $\mathcal{D}$, denoted by
        $
        \bigsqcup_{\mathcal{D}} F(D) .
        $
        Coproducts in the category of sets and in the category of topological spaces are the disjoint unions
        Every coproduct comes with canonical structure maps, called \textbf{inclusions}.\\
\textbf{Pushouts} are colimits over a diagram category $\mathcal{D}$ of the form
        $
        D_1 \leftarrow D_0 \rightarrow D_2 .
        $.\\
Another important class of examples is \textbf{coequalizers}. These are colimits of diagrams of the form
        $$
        F\left(D_0\right) \underset{\alpha}{\stackrel{\beta}{\Longrightarrow}} F\left(D_1\right) .
        $$


\begin{example}[Colimits]
    \begin{enumerate}
        \item Colimits exist in the category of Sets:
        $$
        \operatorname{colim}_{\mathcal{D}} F=\bigsqcup_{D \text { object of } \mathcal{D}} F(D) / \sim,
        $$
        where we declare that an $x \in F(D)$ is equivalent to a $y \in F\left(D^{\prime}\right)$ if there is a morphism $f \in \mathcal{D}\left(D, D^{\prime}\right)$, such that $F(f)(x)=y$. This relation is not symmetric, so one has to consider the equivalence relation generated by this relation. 
        \item If all structure maps $F(i<j)$ are monomorphisms, then we might interpret the colimit $\operatorname{colim}_{\mathcal{D}} F$ as the union of the $F(i)$ s. Typical examples are increasing sequences of sets or topological spaces
        $$
        X_0 \subset X_1 \subset X_2 \subset \ldots
        $$
        or increasing sequences of abelian groups, vector spaces, and other algebraic objects. 
        \item An important class of examples is CW complexes. These are the colimits of their skeleta.
        \item In stable homotopy theory, the stable homotopy groups of spheres are a central object of study. Let $\mathbb{S}^n$ denote the unit sphere in $\mathbb{R}^{n+1}$. As the smash product of spheres satisfies $\mathbb{S}^1 \wedge \mathbb{S}^n \cong \mathbb{S}^{n+1}$ we have stabilization maps
        $$
        \pi_n\left(\mathbb{S}^m\right)=\left[\mathbb{S}^n, \mathbb{S}^{m+1}\right]_* \rightarrow\left[\mathbb{S}^{n+1}, \mathbb{S}^{m+1}\right]_*=\pi_{n+1}\left(\mathbb{S}^m\right)
        $$
        that send a homotopy class $[f]$ to the homotopy class of $\mathbb{S}^1 \wedge f$. Therefore, for every $m$, we get a sequential colimit and as $\pi_n\left(\mathbb{S}^m\right)=0$ for $n<m$, we can express $\pi_n\left(\mathbb{S}^m\right)$ as $\pi_{k+m}\left(\mathbb{S}^m\right)$ in the nontrivial cases, with $k \geq 0$, and get the $k$ th stable homotopy group of spheres as
        $$
        \pi_k^s=\operatorname{colim}\left(\pi_{k+m}\left(\mathbb{S}^m\right) \rightarrow \pi_{k+m+1}\left(\mathbb{S}^{m+1}\right) \rightarrow \pi_{k+m+2}\left(\mathbb{S}^{m+2}\right) \rightarrow \ldots\right)
        $$
        \item The first groups are $\pi_0^s=\mathbb{Z}, \pi_1^s=\mathbb{Z} / 2 \mathbb{Z}$ generated by the stabilization of the Hopf $\operatorname{map} \eta: \mathbb{S}^3 \rightarrow \mathbb{S}^2, \pi_2^s=\mathbb{Z} / 2 \mathbb{Z}, \pi_3^s=\mathbb{Z} / 24 \mathbb{Z}$, and so on.
        \item In the category of pointed topological spaces the pointed sum (also known as the bouquet of spaces) is the coproduct.
        \item Coproducts in the category of abelian groups are given by the direct sum. Coproducts in the category of general groups is the free product.
        \item If $A$ is a topological space, together with continuous maps $f: A \rightarrow X$ and $g: A \rightarrow Y$, the pushout of $X \leftarrow A \rightarrow Y$ is the quotient space of the disjoint union $X \sqcup Y$ by the equivalence relation that identifies $f(a)$ with $g(a)$ for all $a \in A$.
        \item Pushouts of groups are given by amalgamated products, given by $G_1 *_{G_0} G_2$, which is the quotient of the free product $G_1 * G_2$ by the normal subgroup generated by words of the form $f\left(g_0\right) h\left(g_0\right)^{-1}$ for $g_0 \in G_0$.
        \item The cokernel of a homomorphism $f$ is the coequalizer of the diagram $A \underset{f}{\stackrel{0 }{\longrightarrow}} B$ in the category Ab.
    \end{enumerate}
\end{example}

Limits, products... are defined dually

\begin{example}[Limits]
    \begin{enumerate}
        \item Let $\left(X_n\right)_{n \in \mathbb{N}_0}$ be a family of sets with $X_{n+1} \subset X_n$. Then, the limit of the system
        $$
        \ldots \subset X_{n+1} \subset X_n \subset \ldots \subset X_1 \subset X_0
        $$
        is the intersection of the sets $X_n$.
        \item Let $p$ be a fixed prime. The inverse limit of the diagram
        is the ring of $p$-adic integers, $\mathbb{Z}_p$. Here, the maps $p_i$ are the canonical projection maps. An explicit model of the limit is
        $$
        \left\{\left(x_1, x_2, x_3, \ldots\right) \in \prod_{n \geq 1} \mathbb{Z} / p^n \mathbb{Z} \mid p_i\left(x_i\right)=x_{i-1} \text { for all } i \geq 2\right\} .
        $$
        This carries a ring structure, where addition and multiplication are defined coordinatewise.
        \item Kernels in the category of abelian groups are limits of diagrams of the form $A \underset{f}{\stackrel{0}{\longrightarrow}} B$.
        \item The presheaf $F$ is a sheaf if for every $U \in \mathfrak{U}(X)$ and for every open covering $\left(U_i\right)_{i \in I}$ of $U$, the following diagram is an equalizer:
        $$
        F(U) \longrightarrow \prod_{i \in I} F\left(U_i\right) \Longrightarrow \prod_{i, j \in I} F\left(U_i \cap U_j\right) .
        $$
        
        Here, the first map is induced by the restriction maps res $U_U^{U_i}$, and the second pair of arrows is induced by two sets of restriction maps. $U_i \cap U_j$ is a subset of $U_i$ and of $U_j$.
        Sheaves form a category as a full subcategory of the category of presheaves.
        \item Fiber products in the category of sets are limits of diagrams of the form $A \underset{f}{\stackrel{g}{\longrightarrow}} C$. A concrete model for this pullback in these categories is $f^*(p):=Z \times_Y X:=\{(z, x) \in Z \times X \mid f(z)=p(x)\}$
    \end{enumerate}
\end{example}


\section{Adjoint functors}

Let $\mathcal{C}$ and $\mathcal{C}^{\prime}$ be categories. An \textbf{adjunction} between $\mathcal{C}$ and $\mathcal{C}^{\prime}$ is a pair of functors $L: \mathcal{C} \rightarrow \mathcal{C}^{\prime}, R: \mathcal{C}^{\prime} \rightarrow \mathcal{C}$, such that for each pair of objects $C$ of $\mathcal{C}$ and $C^{\prime}$ of $\mathcal{C}^{\prime}$, there is a bijection of sets
$$
\varphi_{C, C^{\prime}}: \mathcal{C}^{\prime}\left(L(C), C^{\prime}\right) \cong \mathcal{C}\left(C, R\left(C^{\prime}\right)\right),
$$
which is natural in $C$ and $C^{\prime}$.
The functor $L$ is then left adjoint to $R$, and $R$ is right adjoint to $L$. We call $(L, R)$ an adjoint pair of functors.

The naturality condidition on the bijections $\varphi_{C, C^{\prime}}$ can be spelled out explicitly as follows: For all morphisms $f: C \rightarrow D$ in $\mathcal{C}$ and $g: C^{\prime} \rightarrow D^{\prime}$ in $\mathcal{C}^{\prime}$, the diagram commutes.
$$
\begin{aligned}
& \mathcal{C}^{\prime}\left(L(D), C^{\prime}\right) \xrightarrow{\mathcal{C}^{\prime}\left(L f, C^{\prime}\right)} \mathcal{C}^{\prime}\left(L(C), C^{\prime}\right) \xrightarrow{\mathcal{C}^{\prime}(L(C), g)} \mathcal{C}^{\prime}\left(L(C), D^{\prime}\right) \\
& \varphi_{D, C^{\prime}} \downarrow \downarrow \varphi_{C, C^{\prime}} \downarrow \varphi_{C, D^{\prime}} \\
& \mathcal{C}\left(D, R\left(C^{\prime}\right)\right) \xrightarrow{\mathcal{C}\left(f, R\left(C^{\prime}\right)\right)} \mathcal{C}\left(C, R\left(C^{\prime}\right)\right) \xrightarrow{\mathcal{C}(C, R(g))} \mathcal{C}\left(C, R\left(D^{\prime}\right)\right) \\
&
\end{aligned}
$$

\begin{example}
A prototypical example of an adjunction is a forgetful functor and a 'free' functor: if $R=U$ is a forgetful functor and if a left adjoint of $U$ exists, then the defining property means that for each morphism from $C$ to $U\left(C^{\prime}\right)$ in the underlying category, there is a unique corresponding morphism from $L(C)$ to $C^{\prime}$, so, in this sense, $L(C)$ is the free object associated with $C$. For topological spaces, the free topological space on a set is the set with discrete topology.
\end{example}

\begin{prop}
    \begin{enumerate}
      \item The functor $L$ is left adjoint to $R$ iff there are natural transformations $\eta$ : Id $\Rightarrow R \circ L$ and $\varepsilon: L \circ R \Rightarrow$ Id with the properties that
      $$
      \varepsilon_L \circ L(\eta)=\operatorname{Id}_L \text { and } R(\varepsilon) \circ \eta_R=\operatorname{Id}_R
      $$
      hence, the diagrams
      $$
      L(C) \xrightarrow{L(\eta)} L R L(C) \text { and } R\left(C^{\prime}\right) \xrightarrow{\eta_{R\left(C^{\prime}\right)}} R L R\left(C^{\prime}\right)
      $$
      commute for all objects $C$ of $\mathcal{C}$ and $C^{\prime \prime}$ of $\mathcal{C}^{\prime}$.
      \item Adjunction can be composed.
      \item Each of the functors $L$ and $R$ determines the other functor uniquely up to isomorphism.
      \item $G$ has a left-adjoint $F$ if and only if $\operatorname{Hom}_C(X, G-)$ is representable for all $X$ in $C$. The natural isomorphism $\Phi_X: \operatorname{Hom}_D(F X,-) \rightarrow \operatorname{Hom}_C(X, G-)$ yields the adjointness; that is
$$
\Phi_{X, Y}: \operatorname{Hom}_{\mathcal{D}}(F X, Y) \rightarrow \operatorname{Hom}_{\mathcal{C}}(X, G Y)
$$
is a bijection for all $X$ and $Y$.
    \end{enumerate}
\end{prop}
The transformation $\eta$ is called the \textbf{unit of the adjunction} and $\varepsilon$ is the \textbf{counit}.

%%%%% adicionar sobre diagramas refelectivos!!!!

\begin{theo}
Let $F: \mathcal{C} \rightarrow \mathcal{D}$ be an arbitrary functor. Then the following are equivalent.
\begin{enumerate}
    \item The functor $F$ possesses a left adjoint $L$, and the corresponding natural transformations $\varepsilon: L F \Rightarrow \operatorname{Id}$ and $\eta$ : Id $\Rightarrow F L$ are natural isomorphisms.
    \item There is a functor $L: \mathcal{D} \rightarrow \mathcal{C}$ and two arbitrary natural isomorphisms $\mathrm{Id} \cong F L$ and $L F \cong \mathrm{Id}$.
    \item The functor $F$ is fully faithful and essentially surjective.
\end{enumerate}
\end{theo}

\subsubsection*{Skeleta of categories}
\begin{example}
    
    \begin{enumerate}
        \item A category is called \textbf{reduced} if isomorphic objects are identical. A subcategory $\mathcal{S}$ of a category $\mathcal{C}$ is a \textbf{skeleton} if $\mathcal{S}$ is reduced and if the inclusion $\mathcal{S} \hookrightarrow \mathcal{C}$ is an equivalence of categories.
        
        \item Consider the category of finite sets and functions. It contains the full subcategory whose objects are the sets of the form $\{1, \ldots, n\}$ for $n \geq 0$. Here, we use the convention that the empty set is encoded by $n=0$. The inclusion functor is full and faithful. As every finite set is in bijection with a standardized set of the form $\{1, \ldots, n\}$ as above, the inclusion functor is also essentially surjective. Therefore, these finite sets build a skeleton.
        
        \item A similar example is the category of finite-dimensional $K$-vector spaces. This has as a skeleton the full subcategory of vector spaces of the form $K^n$ for some finite natural number $n$. Here, $n=0$ encodes the zero vector space.
    \end{enumerate}
\end{example}


\begin{prop}
    Every category has an skeleton
\end{prop}





\section{Concrete categories and representable functors}

A way to talk of \textit{low level structures} present on the objects of a category. Often it is easier to work with less structures, and there results like Yoneda's lemma that show us that it is possible to restrict our study to them.\\

Let \cc be a category. A \textbf{concrete category} over \cc is a category $\mathcal{A}$ together wih a faithful functor $U: \mathcal{A} \rightarrow \mathcal{C}$, called the \textbf{forgetful} (or underlying) functor of the concrete category. $\mathcal{C}$ is called the \textbf{base category}. A concrete category over Set is called a \textbf{construct}.\\
The category of groups (or topological spaces, rings, etc.), with the forgetful functor to Set, is a construct.\\

\begin{enumerate}
    \item A \textbf{structured arrow} with domain $X$ is a pair $(f, A)$ consisting of an A-object $A$ and an X-morphism $X \xrightarrow{f}|A|$,
    \item if $(f, A)$ is \textbf{generating} provided that for any pair of A-morphisms $r, s: A \rightarrow B$ the equality $r \circ f=s \circ f$ implies that $r=s$,
    \item and this $(f, A)$ is called \textbf{extremally generating} (resp. \textbf{concretely generating}) provided that each A-monomorphism (resp. A-embedding) $m: A^{\prime} \rightarrow A$, through which $f$ factors (i.e., $f=m \circ g$ for some $\mathbf{X}$-morphism $g$ ), is an $\mathbf{A}$-isomorphism.
    \item In a construct, an object $A$ is (extremally resp. concretely) generated by a subset $X$ of $|A|$ provided that the inclusion map $X \hookrightarrow|A|$ is (extremally resp. concretely) generating.
\end{enumerate}

\begin{prop}
    In a concrete category $\mathbf{A}$ over $\mathbf{X}$ the following hold for each structured arrow $f: X \rightarrow|A|:$
    \begin{enumerate}
        \item If $(f, A)$ is extremally generating, then $(f, A)$ is concretely generating.
        \item If $(f, A)$ is concretely generating, then $(f, A)$ is generating.
        \item If $X \xrightarrow{f}|A|$ is an $\mathbf{X}$-epimorphism, then $(f, A)$ is generating.
        \item If $X \xrightarrow{f}|A|$ is an extremal epimorphism in $\mathbf{X}$, and if $||$ preserves monomorphisms, then $(f, A)$ is extremally generating.
    \end{enumerate}
\end{prop}

\begin{example}
    \begin{enumerate}
        \item If an abstract category $\mathbf{A}$ is considered to be concrete over itself via the identity functor, then an A-morphism $A \xrightarrow{f} B$, considered as a structured arrow $(f, B)$, is generating (resp. extremally or concretely generating) if and only if $f$ is an epimorphism (resp. an extremal epimorphism). That is,
        $$
        \operatorname{Gen}(\mathbf{A})=\operatorname{Epi}(\mathbf{A}) \text { and } \operatorname{ExtrGen}(\mathbf{A})=\operatorname{ConcGen}(\mathbf{A})=\operatorname{ExtrEpi}(\mathbf{A})
        $$
        \begin{enumerate}
            \item In Vec, Grp, Sgr, Rng, and other algebraic constructs, the concepts of concrete generation and of extremal generation coincide with the familiar (non-categorical) concept of generation.
            In the constructs Sgr and Rng the inclusion map $\mathbb{Z} \hookrightarrow \mathbb{Q}$ is generating, but is not concretely generating [cf. 7.40(5)].

            \item In the construct $\mathbf{A}=$ Top we have
            $$
            \begin{aligned}
            & \text { ConcGen(A) }=\operatorname{Gen}(\mathbf{A})=\text { Surjective maps, and } \\
            & \operatorname{ExtrGen}(\mathbf{A})=\text { Surjective maps with discrete codomain. }
            \end{aligned}
            $$

            \item In the construct $\mathbf{A}=$ Haus we have
            $$
            \begin{aligned}
            \operatorname{Gen}(\mathbf{A}) & =\text { Dense maps } \\
            \text { ConcGen(A) } & =\text { Surjective maps, and } \\
            \text { ExtrGen(A) } & =\text { Surjective maps with discrete codomain. }
            \end{aligned}
            $$
        \item $A \xrightarrow{f} B$ is an epimorphism if and only if $(f, B)$ is generating.
\item If $(f, B)$ is extremally generating and the forgetful functor preserves monomorphisms, then $A \xrightarrow{f} B$ is an extremal epimorphism.
\item If $A \xrightarrow{f} B$ is an extremal epimorphism, then $(f, B)$ is concretely generating.    
        \end{enumerate}
    \end{enumerate}
\end{example}

A \textbf{universal arrow} over an $\mathbf{X}$-object $X$ is a structured arrow $X \xrightarrow{u}|A|$ with domain $X$ such that, for each structured arrow $X \xrightarrow{f}|B|$ with domain $X$, there exists a unique $A$-morphism $\hat{f}: A \rightarrow B$ such that the triangle 
\begin{tikzcd}
    X \arrow[r, "u"] \arrow[dr, "f"'] &  {|A|} \arrow[d, "\overline{f}"]\\
    & {|B|} 
\end{tikzcd} commutes. The pair $(u,A)$ is called a \textbf{free object}.

\begin{example}
\begin{enumerate}
    \item In a construct, an object $A$ is a free object over the empty set if and only if $A$ is an initial object, and over a singleton set if and only if $A$ represents the forgetful functor.
    \item In the construct Vec each object is a free object over any basis for it.
    \item In the constructs Top and Pos the free objects are precisely the discrete ones.
    \item In the construct $\mathbf{A b}$ free objects over $X$ are the free abelian groups generated by $X$.
    Similarly, the familiar free group generated by a set $X$ is a free object over $X$ in the construct Grp.
    \item To construct a universal arrow in (Ban, $O$ ) over a set $X$, let $\ell_1(X)$ be the subspace of the vector space $K^X$ consisting of all $r=\left(r_x\right)_{x \in X}$ in $K^X$ whose norm $\|r\|=$ $\sum_{x \in X}\left|r_x\right|$ is finite. Then $\ell_1(X)$ is a Banach space. Define $X \xrightarrow{u} O\left(\ell_1(X)\right)$ at $y$ by the Dirac function $u(y)=\left(\delta_{y x}\right)_{x \in X}$. Then $\left(u, \ell_1(X)\right)$ is a universal arrow over $X$. Observe, for comparison, that for the construct (Ban, $U$ ) the only set having a universal arrow is the empty set, and that for the construct Ban $\mathrm{B}_{\mathrm{b}}$ the only sets having universal arrows are the finite ones.
\end{enumerate}
\end{example}

\begin{prop}
    \begin{enumerate}
        \item Every universal arrow is extremally generating.
        \item Any two universal arrows with domain $X$ are isomorphic. Conversely, if $X \xrightarrow{u}|A|$ is a universal arrow and $A \xrightarrow{k} A^{\prime}$ is an $\mathbf{A}$-isomorphism, then $X \xrightarrow{k o u}\left|A^{\prime}\right|$ is also universal.
        \item If a concrete category $\mathbf{A}$ over $\mathbf{X}$ has free objects, then an $\mathbf{A}$-morphism is an $\mathbf{A}$-monomorphism if and only if it is an $\mathbf{X}$-monomorphism.
        \item If a construct $\mathbf{A}$ has a free object over a singleton set, then the monomorphisms in $\mathbf{A}$ are precisely those morphisms that are injective functions.
    \end{enumerate}
\end{prop}

A concrete category over $\mathbf{X}$ is said to have free objects provided that for each $\mathbf{X}$-object $X$ there exists a universal arrow over $X$.\\
The constructs Vec, Grp, Ab, Mon, Sgr, Alg $(\Omega)$, Top, Pos, and $($ Ban,$O)$ have free objects; but the constructs Ban$_b$.

A functor $F: \mathcal{A} \rightarrow$ Set is called representable (by an $\mathcal{A}$-object $A$ ) provided that $F$ is naturally isomorphic to the hom-functor $\operatorname{hom}(A,-): \mathcal{A} \rightarrow$ Set. Note that objects that represents the same functor are isomorphic.

\begin{example}
    \begin{enumerate}
        \item Forgetful functors are often representable. For example,
        (a) Vec $\rightarrow$ Set is represented by the vector space $\mathbb{R}$,
        (b) $\operatorname{Grp} \rightarrow$ Set is represented by the group of integers $\mathbb{Z}$,
        (c) Top $\rightarrow$ Set is represented by any one-point topological space.
        \item The underlying functor $U$ for the construct Ban [5.2(3)] is not representable (see Exercise 10J). However, the faithful unit ball functor $O: \operatorname{Ban} \rightarrow$ Set is represented in the complex case by the Banach space $\mathbb{C}$ of complex numbers.
    \end{enumerate}
\end{example}

\begin{prop}[Representative of Constructs]
    For constructs $(\mathcal{A}, U)$ the forgetful functor is represented by an object $A$ if and only if $A$ is a free object over a singleton set [see Definition 8.22(2)]. This provides many additional examples of representations.
\end{prop}

For small categories $\mathcal{A}$ and $\mathcal{B}$ the \textbf{functor category} $[\mathcal{A}, \mathcal{B}]$ has as objects all functors from $\mathcal{A}$ to $\mathcal{B}$, as morphisms from $F$ to $G$ all natural transformations from $F$ to $G$, as identities the identity natural transformations, and as composition the (horizontal) composition of natural transformations.

\begin{theo}[uniqueness of representations]
    For any functor $F: \mathcal{A} \rightarrow$ Set, any $\mathcal{A}$-object $A$ and any element $a \in F(A)$, there exists a unique natural transformation $\tau: \operatorname{hom}(A,-) \rightarrow F$ with $\tau_A\left(i d_A\right)=a$.
\end{theo}

\begin{coro}[Yoneda Lemma]
    If $F: \mathcal{A} \rightarrow$ Set is a functor and $A$ is an $\mathcal{A}$-object, then the following function
    $$
    Y:[\operatorname{hom}(A,-), F] \rightarrow F(A) \text { defined by } Y(\sigma)=\sigma_A\left(i d_A\right),
    $$
    is a bijection (where $[\operatorname{hom}(A,-), F]$ is the set of all natural transformations from hom $(A,-)$ to $F$ ).
    
\end{coro}

\begin{coro}[Yoneda Embedding]
    For any category $\mathcal{A}$, the functor $E: \mathcal{A} \rightarrow\left[\mathcal{A}^{\mathrm{op}} Set \right]$, defined by
$$
E(A \xrightarrow{f} B)=\operatorname{hom}(-, A) \xrightarrow{\sigma_f} \operatorname{hom}(-, B) \text {, }
$$
where $\sigma_f(g)=f \circ g$, is a full embedding.
\end{coro}

\begin{prop}
Consider the representable functor $\mathcal{D}(D,-): \mathcal{D} \rightarrow$ Sets for some object $D$ of $\mathcal{D}$. A useful fact is that
    $$
    \operatorname{colim}_{\mathcal{D}} \mathcal{D}(D,-) \cong\{*\} .
    $$  
\end{prop}







\section{Grupoids}

If we want a limited amount of interaction between $\mathcal{C}$ and $\mathcal{D}$, we can form the join of $\mathcal{C}$ and $\mathcal{D}$, denoted by $\mathcal{C} * \mathcal{D}$. The objects of $\mathcal{C} * \mathcal{D}$ are the disjoint union of the objects of $\mathcal{C}$ and the objects of $\mathcal{D}$ and as morphism we have
$$
(\mathcal{C} * \mathcal{D})(X, Y)=\left\{\begin{array}{l}
\mathcal{C}(X, Y), \text { if } X \text { and } Y \text { are objects of } \mathcal{C} \\
\mathcal{D}(X, Y), \text { if } X \text { and } Y \text { are objects of } \mathcal{D} \\
\{*\}, \text { if } X \text { is an object of } \mathcal{C} \text { and } Y \text { is an object of } \mathcal{D} \\
\varnothing, \text { otherwise. }
\end{array}\right.
$$

A category is a grupoid if all morphisms are isomorphisms.

\begin{example}
    \begin{enumerate}
        \item If $G$ is a group, then we denote by $\mathcal{C}_G$ the category with one object $*$ and $\mathcal{C}_G(*, *)=G$ with group multiplication as composition of maps. Then, $\mathcal{C}_G$ is a groupoid. Hence every group gives rise to a groupoid. Vice versa, a groupoid can be thought of as a group with many objects.
        \item Let $X$ be a topological space. The fundamental groupoid of $X, \Pi(X)$, is the category whose objects are the points of $X$, and $\Pi(X)(x, y)$ is the set of homotopy classes of paths from $x$ to $y$ :
        $$
        \Pi(X)(x, y)=[[0,1], 0,1 ; X, x, y] .
        $$
        
        The endomorphisms $\Pi(x, x)$ of $x \in X$ constitute the fundamental group of $X$ with respect to the basepoint $x, \pi_1(X, x)$.
        \item Another important example of a groupoid is the translation category of a group. If $G$ is a discrete group, then we denote by $\mathcal{E}_G$ the category whose objects are the elements of the group and $$\mathcal{E}_G(g, h)=\left\{h g^{-1}\right\}, g \xrightarrow{h g^{-1}} h.$$

        This category has the important feature that there is precisely one morphism from one object to any other object, so every object has equal rights.

    \end{enumerate}
\end{example}










\chapter{Homological Algebra}

References \cite{richterCategoriesHomotopyTheory2020}

\section{Preadditive categories}

A \textbf{preaddititve category} is a category $\mathcal{A}$, such that for every pair of objects $A_1, A_2$, there is an abelian group of morphisms from $A_1$ to $A_2$ and the composition of morphisms is a bilinear map.\\
A preadditive category with only one object is nothing but a ring. The endomorphisms of that object are an abelian group, and the composition of morphisms defines the multiplicative structure. Thus, a preadditive category can be thought of as a ring with many objects. A group with many objects in this sense is a groupoid, so one might call a preadditive category a ringoid.\\
Let $\mathcal{A}$ and $\mathcal{A}^{\prime}$ be preadditive categories. A functor $F: \mathcal{A} \rightarrow \mathcal{A}^{\prime}$ is additive if for any two objects $A_1, A_2$ of $\mathcal{A}$, the map $F: \mathcal{A}\left(A_1, A_2\right) \rightarrow \mathcal{A}^{\prime}\left(F\left(A_1\right), F\left(A_2\right)\right)$ is a group homomorphism.\\

Assume that a category $\mathcal{C}$ has zero morphisms. Then, the kernel of a morphism $f \in \mathcal{C}\left(C_1, C_2\right)$ is the equalizer of the morphisms $f, 0: C_1 \rightarrow C_2$. Dually, the cokernel of a morphism $f \in \mathcal{C}\left(C_1, C_2\right)$ is the coequalizer of the morphisms $f, 0: C_1 \rightarrow C_2$.

\begin{prop}
    \begin{enumerate}
        \item In a preadditive categoty, all equalizers are kernels.
        \item Initial object exists if and only if zero object exists.
        \item A finite product exists if and only if the finite coproduct exists, called \textbf{biproduct}.
    \end{enumerate}
\end{prop}

A preadditive category is called \textbf{additive} if it has all finite biproducts.\\

\begin{prop}
A functor between additive categories is additive if and only if it preserves biproducts or just products.
\end{prop}

A preadditive category is an \textbf{abelian} category if it satisfies the following: 
\begin{itemize}
    \item There exists a zero object in $\mathcal{A}$.
    \item The category $\mathcal{A}$ has finite biproducts.
    \item Every morphism $f \in \mathcal{A}(A, B)$ has a cokernel and a kernel.
    \item Every monomorphism is a kernel, and every epimorphism is a cokernel.
\end{itemize}

\begin{theo} Let \ca be an abelian category:
\begin{enumerate}
    \item a morphism is an isomorphism if and only if it is both a monomorphism and an epimorphism.
    \item A morphism is a monomorphism if and only if its kernel is zero.
    \item Let $f$ be a morphism. Then, we can factor $f$ as $f=i \circ p$, where $p$ is an epimorphism and $i$ is a monomorphism. Here, $i$ is the kernel of the cokernel of $f$ and $p$ is the cokernel of the kernel of $f$.
    \item A monomorphism is the kernel of its cokernel, and an epimorphism is the cokernel of its kernel.
\end{enumerate}   
\end{theo}

\begin{prop}
Let $\mathcal{D}$ be a small category and let $\mathcal{A}$ be abelian. Then, the functor category $\operatorname{Fun}(\mathcal{D}, \mathcal{A})$ is abelian.   
\end{prop}

In homological algebra one constructs homological invariants of algebraic objects by the following process, or some variant of it:

Let $R$ be a ring and $T$ a covariant additive functor from $R$-modules to abelian groups. Thus the map $\operatorname{Hom}_R(M, N) \rightarrow \operatorname{Hom}_{\mathbf{z}}(T M, T N)$ defined by $T$ is a homomorphism of abelian groups for all $R$-modules $M, N$. For any $R$ module $M$, choose a free (or projective) resolution $\varepsilon: F \rightarrow M$ and consider the chain complex $T F$ of abelian groups obtained by applying $T$ to $F$ termwise. Now $T$, being additive, preserves chain homotopies; so we can apply the uniqueness theorem for resolutions (I.7.5) to deduce that the complex $T F$ is independent, up to canonical homotopy equivalence, of the choice of resolution. Passing to homology, we obtain groups $H_n(T F)$ which depend only on $T$ and $M$ (up to canonical isomorphism).

This construction is of no interest, of course, if $T$ is an exact functor; for then the augmented complex
$$
\cdots \rightarrow T F_1 \rightarrow T F_0 \rightarrow T M \rightarrow 0
$$
is acyclic, so that $H_n(T F)=0$ for $n>0$ and $H_0(T F)=T M$. Thus we can regard the groups $H_n(T F)$ in the general case as a measure of the failure of $T$ to be exact.


\section{Spectral Sequences}


\section{Abelian categories}-+-

\section{Derived functors}

\section{Derived categories}




