\documentclass{book}

% Packages
\usepackage[english]{babel}
\usepackage[T1]{fontenc}
\usepackage[utf8]{inputenc}
\usepackage{biblatex} % Add this line for bibliography support
\newtheorem{theo}{Theorem}
\newtheorem{prop}{Proposition}
\newtheorem{defi}{Definition}
\newtheorem{coro}{Corollary}
\newtheorem{lemm}{Lemma}
\newtheorem{exam}{Example}
\newtheorem{rema}{Remark}

% Title and Author
\title{PhD Studies}
\author{Abraham Rojas Vega}
\addbibresource{references.bib} % Specify the path to your bibliography file


\begin{document}

\maketitle

\tableofcontents
\part*{Topics of Algebraic Topology}
% Chapters




\chapter{Simplical sets and complexes}


\section{Simplical sets}

Let $\mathbf{\Delta}$ be the category of finite ordinal numbers, with order-preserving maps between them. More precisely, the objects for $\Delta$ consist of elements $\mathbf{n}, n \geq 0$, where $\mathbf{n}$ is a string of relations
$$
0 \rightarrow 1 \rightarrow 2 \rightarrow \cdots \rightarrow n
$$
(in other words $\mathbf{n}$ is a totally ordered set with $n+1$ elements). A morphism $\theta: \mathbf{m} \rightarrow \mathbf{n}$ is an order-preserving set function, or alternatively a functor. We usually commit the abuse of saying that $\mathbf{\Delta}$ is the ordinal number category.

A simplicial set is a contravariant functor $X: \Delta^{o p} \rightarrow$ Sets, where Sets is the category of sets.

\begin{example}
    Let $\boldsymbol{\Delta}$ be the category of finite ordinal numbers, with order-preserving maps between them. More precisely, the objects for $\Delta$ consist of elements $\mathbf{n}, n \geq 0$, where $\mathbf{n}$ is a string of relations
$$
0 \rightarrow 1 \rightarrow 2 \rightarrow \cdots \rightarrow n
$$
(in other words $\mathbf{n}$ is a totally ordered set with $n+1$ elements). A morphism $\theta: \mathbf{m} \rightarrow \mathbf{n}$ is an order-preserving set function, or alternatively a functor. We usually commit the abuse of saying that $\boldsymbol{\Delta}$ is the ordinal number category.

A simplicial set is a contravariant functor $X: \Delta^{o p} \rightarrow$ Sets, where Sets is the category of sets.
\end{example}
















\section{Simplical complexes}
\chapter{}

\section{Section 2.1}
\section{Section 2.2}
% Add more chapters and sections as neededed


\printbibliography % Add this line to print the bibliography section

\end{document}