\documentclass{book}

% Packages

\usepackage[english]{babel}
\usepackage{fouriernc}
\usepackage[T1]{fontenc}
\usepackage[utf8]{inputenc}
\usepackage{biblatex} 
\usepackage{amsmath,amsfonts,amssymb,amsthm}
\usepackage{extarrows}
\usepackage{csquotes}
\usepackage[margin=2cm]{geometry}
\usepackage{hyperref}
\newtheorem{theo}{Theorem}
\newtheorem{prop}{Proposition}
\newtheorem{defi}{Definition}
\newtheorem{coro}{Corollary}
\newtheorem{lemm}{Lemma}
\newtheorem{example}{Example}
\newtheorem{rema}{Remark}

% Title and Author
\title{PhD Studies}
\author{Abraham Rojas Vega}
\addbibresource{basicos.bib} % Specify the path to your bibliography file


\begin{document}

\maketitle

\tableofcontents
\part{Topics of Algebra}
% Chapters


\chapter{Category Theory}

\chapter{Homological Algebra}

In homological algebra one constructs homological invariants of algebraic objects by the following process, or some variant of it:

Let $R$ be a ring and $T$ a covariant additive functor from $R$-modules to abelian groups. Thus the map $\operatorname{Hom}_R(M, N) \rightarrow \operatorname{Hom}_{\mathbf{z}}(T M, T N)$ defined by $T$ is a homomorphism of abelian groups for all $R$-modules $M, N$. For any $R$ module $M$, choose a free (or projective) resolution $\varepsilon: F \rightarrow M$ and consider the chain complex $T F$ of abelian groups obtained by applying $T$ to $F$ termwise. Now $T$, being additive, preserves chain homotopies; so we can apply the uniqueness theorem for resolutions (I.7.5) to deduce that the complex $T F$ is independent, up to canonical homotopy equivalence, of the choice of resolution. Passing to homology, we obtain groups $H_n(T F)$ which depend only on $T$ and $M$ (up to canonical isomorphism).

This construction is of no interest, of course, if $T$ is an exact functor; for then the augmented complex
$$
\cdots \rightarrow T F_1 \rightarrow T F_0 \rightarrow T M \rightarrow 0
$$
is acyclic, so that $H_n(T F)=0$ for $n>0$ and $H_0(T F)=T M$. Thus we can regard the groups $H_n(T F)$ in the general case as a measure of the failure of $T$ to be exact.


\section{Spectral Sequences}








\chapter{Group (Cohomology) Theory} 

\paragraph*{Group Ring}

Let $G$ be a group, written multiplicatively. Let $\mathbb{Z} G$ be the free $\mathbb{Z}$-module generated by the elements of $G$. The multiplication in $G$ extends uniquely to a $\mathbb{Z}$-bilinear product $\mathbb{Z G} \times \mathbb{Z G} \rightarrow \mathbb{Z G}$; this makes $\mathbb{Z G}$ a ring, called the \textbf{integral group ring} of $G$.

Note that $G$ is a subgroup of the group $(\mathbb{Z} G)^*$ of units of $\mathbb{Z G}$ 
\begin{theo}[Universal property]
Given a ring $R$ and a group homomorphism $f: G \rightarrow R^*$, there is a unique extension of $f$ to a ring homomorphism $\mathbb{Z G} \rightarrow R$. Thus we have the "adjunction formula"
    $$
    \operatorname{Hom}_{\text {(rings) }}(\mathbb{Z} G, R) \approx \operatorname{Hom}_{\text {(groups) }}\left(G, R^*\right) .
    $$
\end{theo}

A \textbf{(left) $\mathbb{Z} G$-module}, or $G$-module, consists of an abelian group $A$ together with a homomorphism from $\mathbb{Z} G$ to the ring of endomorphisms of $A$. By the universal property, $G$-module is simply an abelian group $A$ together with an action of $G$ on $A$. For example, one has for any $A$ the trivial module structure, with $g a=a$ for $g \in G, a \in A$.

One way of constructing $G$-modules is by linearizing permutation representations. More precisely, if $X$ is a $G$-set (i.e., a set with $G$-action), then one forms the free abelian group $\mathbb{Z X}$ (also denoted $\mathbb{Z}[X]$ ) generated by $X$ and one extends the action of $G$ on $X$ to a $\mathbb{Z}$-linear action of $G$ on $\mathbb{Z} X$. The resulting $G$-module is called a permutation module. In particular, one has a permutation module $\mathbb{Z}[G / H]$ for every subgroup $H$ of $G$, where $G / H$ is the set of cosets $g H$ and $G$ acts on $G / H$ by left translation.

\begin{prop}
Let $X$ be a free $G$-set and let $E$ be a set of representatives for the $G$-orbits in $X$. Then $\mathbb{Z}X$ is a free $\mathbb{Z}G$-module with basis $E$.
\end{prop}


\section{Co-invariants}
If $G$ is a group and $M$ is a $G$-module, then the group of co-invariants of $M$, denoted $M_G$, is defined to be the quotient of $M$ by the additive subgroup generated by the elements of the form $g m-m\left(g \in G, m \in M\right.$ ). Thus $M_G$ is obtained from $M$ by "dividing out" by the $G$-action. (The name "co-invariants" comes from the fact that $M_G$ is the largest quotient of $M$ on which $G$ acts trivially, whereas $M^G$, the group of invariants, is the largest submodule of $M$ on which $G$ acts trivially.) In view of exercise $1 \mathrm{a}$ of $\$ I .2$, we can also describe $M_G$ as $M / I M$, where $I$ is the augmentation ideal of $\mathbb{Z} G$ and $I M$ denotes the set of all finite sums $\sum a_i b_i\left(a_i \in I, b_i \in M\right)$.
Still another description of $M_G$ is given by:
$$
M_G \approx \mathbb{Z} \otimes_{\mathbb{Z} G} M .
$$

Here, in order for the tensor product to make sense, we regard $\mathbb{Z}$ as a right $\mathbb{Z} G$-module (with trivial $G$-action). To prove 2.1 , note that in $\mathbb{Z} \otimes_{\mathbb{Z} G} M$ we have the identity $1 \otimes g m=1 \cdot g \otimes m=1 \otimes m ;$ hence there is a map $M_G \rightarrow$ $\mathbb{Z} \otimes_{\mathbb{Z} G} M$ given by $\bar{m} \mapsto 1 \otimes m$, where $\bar{m}$ denotes the image in $M_G$ of an element $m \in M$. On the other hand, using the universal property of the tensor product, we can define a map $\mathbb{Z} \otimes_{\mathbb{Z} G} M \rightarrow M_G$ by $a \otimes m \mapsto a \bar{m}$. These two maps are inverses of one another.

In view of 2.1 and standard properties of the tensor product, we immediately obtain the following two properties of the co-invariants functor:




\section{An spectral sequence for group cohomology}

Suppose that $X$ is a simplicial set and $x_i$ are simplicial subsets such that $X=U X_i$. Then, setting $X_{i j}=X_i \cap X_j$ (etc.) we'11 obviously have for the realisations: $|x|=U\left|x_i\right|,\left|x_i\right| \cap\left|x_j\right|=\left|x_{i j}\right|, \ldots$ Let's suppose that the set of indices is linearly ordered. Consider the following bicomplex:
$$ K = \longrightarrow \underset{i<j<k}{\oplus} C_*\left(x_{i j k}\right) \longrightarrow \underset{i<j}{\oplus} C_*\left(x_{i j}\right)\longrightarrow \underset{i}{\oplus} C_*\left(x_{i}\right) $$


Here by a bicomplex we understand a bicomplex in the sense of Grothendieck [9] i.e. the differentials $d_1$ and $d_2$ commute. (The sign in this approach appears in the definition of the total differentials). The vertical arrows of the bicomplex map $C_*\left(x_i \cdots_i\right)$ into $\underset{k=0}{q} C_*\left(x_{i_0} \ldots \hat{i}_k \ldots i_q\right)$, the mapping into the kth summand differing $k=0$ by a sign $(-1)^k \quad$ from the natural embedding.

The first spectral sequence of this bicomplex degenerates and yields an isomorphism $H_{\star}(K) \cong H_{\star}(X)$. (Moreover this isomorphism is induced by the canonical map $K \rightarrow C_*(X)$). The second spectral sequence gives us a functorial spectral sequence of the first quadrant, whose limit equals $H_*(X)$, while its differential $d r$ has bidegree $(r-1,-r)$ and its $E^1$-term looks as follows: $$E_{p q q}^1=\underset{i_0<\ldots<i_q}{\otimes} H_p\left(x_{i_0} \ldots i_q\right)$$

Suppose $G$ is a group. Let $X_G$ denote the simplicial set (and its geometric realisation), whose p-simplices are sequences $\left(g_0, \ldots, g_p\right)$ of elements of $G$, with the usual faces and degeneracies. This space $X_G$ is contractible by (1.2). The group $G$ acts from the right on $X_G$ and this action is obviously free, hence $B G=X_G / G$ is a classifying space of $G$. The complex $C_*(B G)=C_*(G)$ coincides with the usual complex associated with $G$. Moreover $C_*(G)=C_*\left(X_G\right) \otimes_G Z$.

If $H$ is a subgroup of $G$, then $X_G / H$ is a classifying space for $H$ and hence $B H=X_H / H \rightarrow X_G / H$ is a homotopy equivalence. In particular $C_*(H)+C_*\left(X_G\right) \otimes_H \mathbb{Z}=C_*\left(X_G\right) \otimes_G Z|G / H|$ is a homotopy equivalence.

(2.3) The spectral sequence associated with a family of subgroups.

Suppose $G$ is a group and $G_1, \ldots, G_n$ are subgroups. Then $B G_i$ may be viewed as a simplicial subset of $B G$ and $B G_i \cap B G_j=B\left(G_i \cap G_j\right)$.. Denote $U B G_i$ by $X$ and consider the spectral sequence of the covering $X=U B G_i$. Along with the bicomplex $K$ introduced in (2.1) we also consider the following bicomplex:

$$K' = \underset{i<j<k}{\oplus} C_*\left(X_G\right) \otimes_G Z\left[G / G_{i j k}\right] \longrightarrow \underset{i<j}{\oplus} C_*\left(X_G\right) \otimes_G Z\left[G / G_{i j}\right] \longrightarrow \underset{i<j}{\oplus} C_*\left(X_G\right) \otimes_G Z\left[G / G_{i}\right] $$

There is a natural mapping of bicomplexes $K+K^{\prime}$ and because of (2.2) this mapping induces an isomorphism of second spectral sequences so that $H_{\star}(X)=H_{\star}(K)=H_*\left(K^{\prime}\right)$. The first spectral sequence of $K^{\prime}$ looks as follows: $E_{*, q}^1=C_*\left(X_G\right) \otimes_G H_q(L)$, where $L$ is the following complex of left G-modules:
$$
\oplus \mathbb{Z}\left[G / G_i\right]+\oplus \mathbb{Z}\left[G / G_{i j}\right]+\oplus \mathbb{Z}\left[G / G_{i j k}\right]+\ldots
$$


\begin{prop}
If $G_1, \ldots, G_n$ are subgroups of $G$, there exists a fuctorial spectral sequence of the first quadrart, the $E^2$ term of which looks like: $E_{p q}^2=H_p\left(G, H_q(L)\right)$, where $L$ is the complex defined above. It converges to $H_{\star}\left(U B G_j\right)$ and the differential $d^r$ has bidegree $(-r, r-1)$.   
\end{prop}

(2.5) In the notations of (2.3), let $Z(G,\{G\})$ be the simplicial set whose non-degenerate p-simplices are sequences $\left(\bar{g}_0, \ldots, \bar{g}_p\right)$, where $\bar{g}_i \varepsilon G / G_{k_i}, k_0<\ldots<k_p$, and the $\bar{g}_i$ are such that there is $g \in G$ with $\vec{g}_i=g \bmod G_{k_i}$ for all i. (If one covers $G$ by the right cosets of the $G_i$, then $Z\left(G_g\left\{G_i\right\}\right)$ is the nerve of this covering.) It is easy to see that the geometric realization of this simplicial set is an ordered simplicial space and that the complex $L=L\left(G,\left\{G_i\right\}\right)$ equals the (ordered) simplicial complex [7] of this simplicial space, or in other words, the complex $L$ equals the normalised complex of the simplicial set $Z\left(G,\left\{G_i\right\}\right)$. In particular, $H_*(L)=H_*\left(Z\left(G,\left\{G_i\right\}\right)\right)$.

(2.6) Remark. It may be shown easily that the space $Z\left(G,\left\{G_i\right\}\right)$, is homotopy equivalent to Volodin's space $V\left(G,\left\{G_i\right\}\right)$, but we will not need this fact.








\chapter{(General) Module Theory}

\section{Linear Algebra}









\part{Topics of Algebraic Topology}



\chapter{Simplical sets and complexes}
\cite{weibelIntroductionHomologicalAlgebra1994}
Simplicial complexes are more intuitive, and are the foundation of algebraic topology. Simplicial complexes were also called \textit{simplicial schemes} and simplicial sets, \textit{semi-simplicial} complexes. 

\section{(Abstract) simplical complexes}

A set (of \textbf{vertices}) together with a  family of finite subsets (\textbf{simplexes}) such that every subset of every simplex is a simplex and every subset consisting of a single vertex is a simplex.  

\begin{example}
    \begin{enumerate}
        \item The \textbf{standard n-simplex} $\Delta^n$ is the set of all $(n+1)$-tuples $(t_0, \ldots, t_n)$ of non-negative real numbers such that $t_0 + \cdots + t_n = 1$. The standard 0-simplex is a point, the standard 1-simplex is a line segment, the standard 2-simplex is a triangle, and so on.

        \item The \textbf{boundary} of the standard n-simplex $\Delta^n$ is the set of all $(n+1)$-tuples $(t_0, \ldots, t_n)$ of non-negative real numbers such that $t_0 + \cdots + t_n = 1$ and at least one of the $t_i$ is zero. The boundary of the standard 0-simplex is empty, the boundary of the standard 1-simplex is the set of its two endpoints, the boundary of the standard 2-simplex is the set of its three edges, and so on.

        \item (\textbf{Concrete simplicial complexes}) It is subset of $\mathbb{R}^n$ that is a union of standard simplices, that satisfies the previous conditions.

        \item If Y is a subset of the vertex set of a simplicial scheme $S$, then we can introduce on it the induced simplicial scheme structure $ Y \cap S$, by defining the simplexes as the subsets of $ Y $ that are simplexes of $S$.  

        \item Let $X$ be a set and let $\{p(y): y \in Y\}$ be a covering of $X$. Then we can consider two simplicial complexes. 
        \begin{enumerate}
            \item The nerve $\operatorname{Nerv}(p)$ of the covering is the simplicial scheme with the vertex set $Y$, and a subset $Z$ of $Y$ is counted as a simplex if the intersection $\underset{Z}{\cap} p(y)$ is non-empty. 
            \item The simplicial complex $V(p)$ is the simplicial scheme with the vertex set $X$, and a subset $Z$ of $X$ is counted as a simplex if $Z$ is contained in some $p(y)$.
        \end{enumerate}
    \end{enumerate}
\end{example}

\subsection*{Geometric realization}

The construction goes as follows. First, define $|K|$ as a subset of $[0,1]^S$ consisting of functions $t: S \rightarrow[0,1]$ satisfying the two conditions: $\square$
$$
\begin{aligned}
& \left\{s \in S: t_s>0\right\} \in K \\
& \sum_{s \in S} t_s=1
\end{aligned}
$$

Now think of the set of elements of $[0,1]^S$ with finite support as the direct limit of $[0,1]^A$ where $A$ ranges over finite subsets of $S$, and give that direct limit the induced topology. Now give $|K|$ the subspace topology. \textit{It is always Hausdorff}. We will identify an abstract simplicial complex with its geometric realization.





\section{Simplical sets}



Let $\mathbf{\Delta}$ be the category of finite ordinal numbers, with order-preserving maps between them. More precisely, the objects for $\Delta$ consist of elements $\mathbf{n}, n \geq 0$, where $\mathbf{n}$ is a string of relations
$$
0 \rightarrow 1 \rightarrow 2 \rightarrow \cdots \rightarrow n
$$
(in other words $\mathbf{n}$ is a totally ordered set with $n+1$ elements). A morphism $\theta: \mathbf{m} \rightarrow \mathbf{n}$ is an order-preserving set function, or alternatively a functor. We usually commit the abuse of saying that $\mathbf{\Delta}$ is the ordinal number category.

A simplicial set is a contravariant functor $X: \Delta^{o p} \rightarrow$ Sets, where Sets is the category of sets.

\begin{rema}
    The standard covariant functor: $\mathbf{n} \mapsto |\Delta^n| $ from $\Delta$ to \textbf{Top}. The singular set $S(T)$ is the simplicial set given by
    $$
    \mathbf{n} \mapsto \operatorname{hom}\left(\left|\Delta^n\right|, T\right) .
    $$
    
    This is the object that gives the singular homology of the space $T$.\\

    The standard $n$-simplex, simplicial $\Delta^n$ in the simplicial set category $\mathbf{S}$ is defined by
$$
\Delta^n=\operatorname{hom}_{\Delta}(, \mathbf{n}) .
$$

In other words, $\Delta^n$ is the contravariant functor on $\Delta$ which is represented by n.
\end{rema}

A map $f: X \rightarrow Y$ of simplicial sets (or, more simply, a simplicial map) is a natural transformation of contravariant set-valued functors defined on $\boldsymbol{\Delta}$. We shall use $\mathbf{S}$ to denote the resulting category of simplicial sets and simplicial maps.\\

From a simplicial set $Y$, one may construct a simplicial abelian group $\mathbb{Z} Y$ (ie. a contravariant functor $\boldsymbol{\Delta}^{o p} \rightarrow \mathbf{A b}$ ), with $\mathbb{Z} Y_n$ set equal to the free abelian group on $Y_n$. The simplicial abelian group $\mathbb{Z} Y$ has associated to it a chain complex, called its Moore complex and also written $\mathbb{Z} Y$, with
$$
\begin{gathered}
\mathbb{Z} Y_0 \stackrel{\partial}{\leftarrow} \mathbb{Z} Y_1 \stackrel{\partial}{\leftarrow} \mathbb{Z} Y_2 \leftarrow \ldots \quad \text { and } \\
\partial=\sum_{i=0}^n(-1)^i d_i
\end{gathered}
$$
in degree $n$. Recall that the integral singular homology groups $H_*(X ; \mathbb{Z})$ of the space $X$ are defined to be the homology groups of the chain complex $\mathbb{Z} S X$. The homology groups $H_n(Y, A)$ of a simplicial set $Y$ with coefficients in an abelian group $A$ are defined to be the homology groups $H_n(\mathbb{Z} Y \otimes A)$ of the chain complex $\mathbb{Z} Y \otimes A$.


\subsection*{Classifying space}

Suppose that $\mathcal{C}$ is a (small) category. The classifying space (or nerve ) $B \mathcal{C}$ of $\mathcal{C}$ is the simplicial set with
$$
B \mathcal{C}_n=\operatorname{hom}_{\text {cat }}(\mathbf{n}, \mathcal{C}),
$$
$n$-simplex is a string
$$
a_0 \xrightarrow{\alpha_1} a_1 \xrightarrow{\alpha_2} \ldots \xrightarrow{\alpha_n} a_n
$$
of composeable arrows of length $n$ in $\mathcal{C}$.\\

If $G$ is a group, then $G$ can be identified with a category (or groupoid) with one object $*$ and one morphism $g: * \rightarrow *$ for each element $g$ of $G$, and so the classifying space $B G$ of $G$ is defined. Moreover $|B G|$ is an Eilenberg-Mac Lane space of the form $K(G, 1)$, as the notation suggests; this is now the standard construction.


\subsection*{Geometric realization}

\textbf{The simplex category:} $\Delta \downarrow X$ of a simplicial set $X$. The objects of $\Delta \downarrow X$ are the maps $\sigma: \Delta^n \rightarrow X$, or simplices of $X$. An arrow of $\Delta \downarrow X$ is a commutative diagram of simplicial maps .....

Observe that $\theta$ is induced by a unique ordinal number $\operatorname{map} \theta: \mathbf{m} \rightarrow \mathbf{n}$.
\begin{lemm} There is an isomorphism
$$
\begin{aligned}
& X \cong \underset{\Delta^n \longrightarrow X}{\lim _{\longrightarrow}} \Delta^n . \\
& \text { in } \Delta \downarrow X \\
&
\end{aligned}
$$
\end{lemm}

The realization $|X|$ of a simplicial set $X$ is defined by the colimit
$$
\begin{aligned}
|X|= & \xrightarrow{\lim }\left|\Delta^n\right| . \\
& \Delta^n \rightarrow X \\
& \text { in } \Delta \downarrow X
\end{aligned}
$$
in the category of topological spaces. The construction $X \mapsto|X|$ is seen to be functorial in simplicial sets $X$, by using the fact that any simplicial map $f: X \rightarrow Y$ induces a functor $f_*: \Delta \downarrow X \rightarrow \Delta \downarrow Y$ by composition with $f$.

\begin{prop}
    The realization functor is left adjoint to the singular functor in the sense that there is an isomorphism
$$
\operatorname{hom}_{\text {Top }}(|X|, Y) \cong \operatorname{hom}_{\mathbf{S}}(X, S Y)
$$
which is natural in simplicial sets $X$ and topological spaces $Y$. In particular, since $\mathbf{S}$ has all colimits and the realization functor, || preserves them.
\end{prop} 

\begin{prop}
    $|X|$ is a $C W$-complex for each simplicial set $X$. In particular it is a compactly generated Hausdorff space.
\end{prop}

\section{CW-complexes}

They can be defined in an inductive way:

\begin{enumerate}
    \item Start with a discrete set $X^0$, whose points are regarded as 0 -cells.
    \item Inductively, form the $\boldsymbol{n}$-skeleton $X^n$ from $X^{n-1}$ by attaching $n$-cells $e_\alpha^n$ via maps $\varphi_\alpha: S^{n-1} \rightarrow X^{n-1}$. This means that $X^n$ is the quotient space of the disjoint union $X^{n-1} \amalg_\alpha D_\alpha^n$ of $X^{n-1}$ with a collection of $n$-disks $D_\alpha^n$ under the identifications $x \sim \varphi_\alpha(x)$ for $x \in \partial D_\alpha^n$. Thus as a set, $X^n=X^{n-1} \amalg_\alpha e_\alpha^n$ where each $e_\alpha^n$ is an open $n$-disk.
    \item One can either stop this inductive process at a finite stage, setting $X=X^n$ for some $n<\infty$, or one can continue indefinitely, setting $X=\cup_n X^n$. In the latter case $X$ is given the weak topology: A set $A \subset X$ is open (or closed) iff $A \cap X^n$ is open (or closed) in $X^n$ for each $n$.
    
\end{enumerate}

\begin{example}
    \begin{enumerate}
        \item A 1-dimensional cell complex $X=X^1$ is what is called a graph in algebraic topology. It consists of vertices (the 0 -cells) to which edges (the 1-cells) are attached. The two ends of an edge can be attached to the same vertex.
        \item The sphere $S^n$ has the structure of a cell complex with just two cells, $e^0$ and $e^n$, the $n$-cell being attached by the constant map $S^{n-1} \rightarrow e^0$. This is equivalent to regarding $S^n$ as the quotient space $D^n / \partial D^n$.
        \item \textbf{Real projective $\boldsymbol{n}$-space $\mathbb{R} \mathrm{P}^n$.} It is equivalent to the quotient space of a hemisphere $D^n$ with antipodal points of $\partial D^n$ identified. Since $\partial D^n$ with antipodal points identified is just $\mathbb{R P} \mathrm{P}^{n-1}$, we see that $\mathbb{R} \mathrm{P}^n$ is obtained from $\mathbb{R} \mathrm{P}^{n-1}$ by attaching an $n$-cell, with the quotient projection $S^{n-1} \rightarrow \mathbb{R} P^{n-1}$ as the attaching map. It follows by induction on $n$ that $\mathbb{R P}^n$ has a cell complex structure $e^0 \cup e^1 \cup \cdots \cup e^n$ with one cell $e^i$ in each dimension $i \leq n$.\\
        The infinite union $\mathbb{R} P^{\infty}=U_n \mathbb{R} P^n$ becomes a cell complex with one cell in each dimension. We can view $\mathbb{R} P^{\infty}$ as the space of lines through the origin in $\mathbb{R}^{\infty}=\bigcup_n \mathbb{R}^n$.
        
        \item \textbf{Complex projective space $\mathbb{C} P^n$.} It is equivalent to the quotient of the unit sphere $S^{2 n+1} \subset \mathbb{C}^{n+1}$ with $v \sim \lambda v$ for $|\lambda|=1$. \\
        It is also possible to obtain $\mathbb{C P}^n$ as a quotient space of the disk $D^{2 n}$ under the identifications $v \sim \lambda v$ for $v \in \partial D^{2 n}$, in the following way. The vectors in $S^{2 n+1} \subset \mathbb{C}^{n+1}$ with last coordinate real and nonnegative are precisely the vectors of the form $\left(w, \sqrt{1-|w|^2}\right) \in \mathbb{C}^n \times \mathbb{C}$ with $|w| \leq 1$. Such vectors form the graph of the function $w \mapsto \sqrt{1-|w|^2}$. This is a disk $D_{+}^{2 n}$ bounded by the sphere $S^{2 n-1} \subset S^{2 n+1}$ consisting of vectors $(w, 0) \in \mathbb{C}^n \times \mathbb{C}$ with $|w|=1$. Each vector in $S^{2 n+1}$ is equivalent under the identifications $v \sim \lambda v$ to a vector in $D_{+}^{2 n}$, and the latter vector is unique if its last coordinate is nonzero. If the last coordinate is zero, we have just the identifications $v \sim \lambda v$ for $v \in S^{2 n-1}$.\\
        It follows that $\mathbb{P}^n$ is obtained from $\mathbb{C} \mathrm{P}^{n-1}$ by attaching a cell $e^{2 n}$ via the quotient map $S^{2 n-1} \rightarrow \mathbb{C P}^{n-1}$. So by induction on $n$ we obtain a cell structure $\mathbb{C P}^n=e^0 \cup e^2 \cup \cdots \cup e^{2 n}$ with cells only in even dimensions. Similarly, $\mathbb{C P}^{\infty}$ has a cell structure with one cell in each even dimension.
    \end{enumerate}
\end{example}

Each cell $e_\alpha^n$ in a cell complex $X$ has a \textbf{characteristic map} $\Phi_\alpha: D_\alpha^n \rightarrow X$ which extends the attaching map $\varphi_\alpha$ and is a homeomorphism from the interior of $D_\alpha^n$ onto $e_\alpha^n$. Namely, we can take $\Phi_\alpha$ to be the composition $D_\alpha^n \hookrightarrow X^{n-1} \coprod_\alpha D_\alpha^n \rightarrow X^n \hookrightarrow X$ where the middle map is the quotient map defining $X^n$. 


\chapter{Geometric Group Theory}

By a \textbf{$G$-complex} we will mean a $C W$-complex $X$ together with an action of $G$ on $X$ which permutes the cells. Thus we have for each $g \in G$ a homeomorphism $x \mapsto g x$ of $X$ such that the image go of any cell $\sigma$ of $X$ is again a cell. For example, if $X$ is a simplicial complex on which $G$ acts simplicially, then $X$ is a $G$-complex.

If $X$ is a $G$-complex then the action of $G$ on $X$ induces an action of $G$ on the cellular chain complex $C_*(X)$, which thereby becomes a chain complex of $G$-modules. Moreover, the canonical augmentation $\varepsilon: C_0(X) \rightarrow \mathbb{Z}$ (defined by $\varepsilon(v)=1$ for every 0 -cell $v$ of $X$ ) is a map of $G$-modules.

We will say that $X$ is a free $G$-complex if the action of $G$ freely permutes the cells of $X$ (i.e., $g \sigma \neq \sigma$ for all $\sigma$ if $g \neq 1$ ). In this case each chain module $C_n(X)$ has a $\mathbb{Z}$-basis which is freely permuted by $G$, hence by $3.1 C_n(X)$ is a free $\mathbb{Z} G$-module with one basis element for every $G$-orbit of cells. (Note that to obtain a specific basis we must choose a representative cell from each orbit and we must choose an orientation of each such representative.)

Finally, if $X$ is contractible, then $H_*(X) \approx H_*$ (pt.); in other words, the sequence
$$
\cdots \rightarrow C_n(X) \stackrel{\partial}{\rightarrow} C_{n-1}(X) \rightarrow \cdots \rightarrow C_0(X) \stackrel{\varepsilon}{\rightarrow} \mathbb{Z} \rightarrow 0
$$
is exact. We have, therefore:

\begin{prop}
    
    Let $X$ be a contractible free $G$-complex. Then the augmented cellular chain complex of $X$ is a free resolution of $\mathbb{Z}$ over $\mathbb{Z} G$.
\end{prop}









\chapter{Homotopy theory}

Let $I^n$ be the $n$-dimensional unit cube, the product of $n$ copies of the interval $[0,1]$. The boundary $\partial I^n$ of $I^n$ is the subspace consisting of points with at least one coordinate equal to 0 or 1 . For a space $X$ with basepoint $x_0 \in X$, define $\pi_n\left(X, x_0\right)$ to be the set of homotopy classes of maps $f:\left(I^n, \partial I^n\right) \rightarrow\left(X, x_0\right)$, where homotopies $f_t$ are required to satisfy $f_t\left(\partial I^n\right)=x_0$ for all $t$. The definition extends to the case $n=0$ by taking $I^0$ to be a point and $\partial I^0$ to be empty, so $\pi_0\left(X, x_0\right)$ is just the set of path-components of $X$.

When $n \geq 2$, a sum operation in $\pi_n\left(X, x_0\right)$, generalizing the composition operation in $\pi_1$, is defined by
$$
(f+g)\left(s_1, s_2, \cdots, s_n\right)= \begin{cases}f\left(2 s_1, s_2, \cdots, s_n\right), & s_1 \in[0,1 / 2] \\ g\left(2 s_1-1, s_2, \cdots, s_n\right), & s_1 \in[1 / 2,1]\end{cases}
$$

It is evident that this sum is well-defined on homotopy classes. Since only the first coordinate is involved in the sum operation, the same arguments as for $\pi_1$ show that $\pi_n\left(X, x_0\right)$ is a group, with identity element the constant map sending $I^n$ to $x_0$ and with inverses given by $-f\left(s_1, s_2, \cdots, s_n\right)=f\left(1-s_1, s_2, \cdots, s_n\right)$.


\begin{prop}
    If $n \geq 2$, then $\pi_n\left(X, x_0\right)$ is abelian.
\end{prop}


\section{Covering spaces}
%estamos solo usando estos resultados... deberiamos dar mejores definiciones

The reader who has studied covering spaces has, of course, seen many examples of free $G$-complexes. Indeed, suppose $p: Y \rightarrow Y$ is a regular covering map with $G$ as group of deck transformations. (See the appendix to this chapter for a review of regular covers.) If $Y$ is a $C W$-complex, then it is an elementary fact that $\boldsymbol{Y}$ inherits a $C W$-structure such that the $G$-action permutes the cells, cf. Schubert [1968], III.6.9. Explicitly, the open cells of $\boldsymbol{Y}$ lying over an open cell $\sigma$ of $Y$ are simply the connected components of $p^{-1} \sigma$; these cells are permuted freely and transitively by $G$, and each is mapped homeomorphically onto $\sigma$ under $p$. Thus $\tilde{Y}$ is a free $G$-complex and $C_* \tilde{Y}$ is a complex of free $\mathbb{Z G}$-modules with one basis element for each cell of $Y$.

In view of 4.1, it is natural now to consider $C W$-complexes $Y$ satisfying the following three conditions:
\begin{enumerate}
    \item $Y$ is connected.
    \item $\pi_1(Y)$ is isomorphic to $G$.
    \item The universal covering space $X$ of $Y$ is contractible.
\end{enumerate}








\part{Topics of Geometry}




\part{K-theory}
\chapter{K-theory constructions} 

\section{Milnor's K-theory}
% Add more chapters and sections as neededed

For $n \geq 3$ the \textbf{Steinberg group} $\operatorname{St}_n(R)$ of a ring $R$ is the group defined by generators $x_{i j}(r)$, with $i, j$ a pair of distinct integers between 1 and $n$ and $r \in R$, subject to the following "Steinberg relations":
$$
\begin{gathered}
x_{i j}(r) x_{i j}(s)=x_{i j}(r+s), \\
{\left[x_{i j}(r), x_{k \ell}(s)\right]= \begin{cases}1 & \text { if } j \neq k \text { and } i \neq \ell, \\
x_{i \ell}(r s) & \text { if } j=k \text { and } i \neq \ell, \\
x_{k j}(-s r) & \text { if } j \neq k \text { and } i=\ell .\end{cases} }
\end{gathered}
$$

As observed in (1.3.1), the Steinberg relations are also satisfied by the elementary matrices $e_{i j}(r)$ which generate the subgroup $E_n(R)$ of $G L_n(R)$. Hence there is a canonical group surjection $\phi_n: S t_n(R) \rightarrow E_n(R)$ sending $x_{i j}(r)$ to $e_{i j}(r)$.

The Steinberg relations for $n+1$ include the Steinberg relations for $n$, so there is an obvious map $S t_n(R) \rightarrow S t_{n+1}(R)$. We write $S t(R)$ for $\lim _{\longrightarrow} S t_n(R)$ and observe that by stabilizing, the $\phi_n$ induce a surjection $\phi: S t(R) \rightarrow E(R)$.


The group $K_2(R)$ is the kernel of $\phi: S t(R) \rightarrow E(R)$. Thus there is an exact sequence of groups
$$
1 \rightarrow K_2(R) \rightarrow S t(R) \xrightarrow{\phi} G L(R) \rightarrow K_1(R) \rightarrow 1 .
$$

It will follow from Theorem 5.2.1 below that $K_2(R)$ is an abelian group. Moreover, it is clear that $S t$ and $K_2$ are both covariant functors from rings to groups, just as $G L$ and $K_1$ are.

\begin{theo}
$K_2(R)$ is an abelian group. In fact it is precisely the center of $\operatorname{St}(R)$.
\end{theo}

We'll define right actions of the symmetric group $S_n$ on ${ }_{G L}(R)$ and on $S t_n(R)$ by setting
$$
\left(\alpha^s\right)_{k, \ell}=\alpha_s(k), s(\ell) ; \quad x_{k \ell}(a)^s=x_s^{-1}(k), s^{-1}(\ell)(a) .
$$

These actions are compatible with the projections $S t_n(R) \rightarrow E_n(R)$ and with the homomorphisms $S t_n(R)+S t_{n+1}(R)$ and $G L_n(R)+G L_{n+1}(R)$. In particular, they induce an action on $\overline{S t}_n(R)$.

\begin{lemm}
    For any $s \in S_{n+1}$ the embeddings $u_n$ and $u_n^s$ are homotopic.
\end{lemm} 



\section{Volodin's K-theory}

Let $G$ be a group and $\left\{G_i\right\}{ }_{i \varepsilon I}$ a family of subgroups. Define $V\left(G,\left\{G_i\right\}\right)$, or just $V(G)$ to be the simplicial complex, whose vertices are the elements of $G$, where $g_0, \ldots, g_p\left(g_i \neq g_j\right)$ form a $p$-simplex if for some $G_i$ all the elements $g_j g_k^{-1}$ lie in $G_i$. If $H$ is another group with a family of subgroups $\left\{H_j\right\}$ and $\phi: G \rightarrow H$ is a homomorphism sending each $G_i$ into some $H_j$, then $\phi$ induces a simplicial map $V(\phi): V(G) \rightarrow V(H)$.

In many situations it is more convenient to use simplicial sets instead of simplicial complexes: Denote by $W\left(G,\left\{G_i\right\}\right)$ the geometric realization of the simplicial set whose p-simplices are the sequences $\left(g_0, \ldots, g_p\right)$ of elements of $G$ (not necessarily distinct) such that for some $G_i$ al1 $g_j g_k^{-1}$ lie in $G_i$, the $r$-th face (resp. degeneracy) of this simplex being obtained by omitting $g_r$ (resp., repeating $g_r$ ). Associating with any p-simplex $\left(g_0, \ldots, g_p\right)$ the linear singular simplex of the space $V(G)$ which sends the $i$-th vertex of the standard simplex to $g_j$, we obtain a map of simplicial sets from $W(G)$ to the simplicial set of singular simplices of $V(G)$ and hence a cellular map (linear on any simplex) from $W(G)$ to $V(G)$. This map is a homotopy equivalence .... %put in simplicial sets

Suppose that $R$ is a ring, $n$ a natural number and $\sigma$ a partial ordering of $\{1, \ldots, n\}$. Define $T_n^\sigma(R)$ to be the subgroup of $G L_n(R)$ consisting of the $\alpha$ with $\alpha_{i j}=1$ and $\alpha_{i j}=0$ if $i \& j$. Subgroups of this form will be called triangular subgroups of $G L_n(R)$. The space $V\left(G L_n(R),\left\{T_n^\sigma(R)\right\}\right)$ will be denoted by $V_n(R)$. Since any partial ordering may be extended to a linear ordering, it suffices to consider linear orderings when defining $V_n(R)$. The natural embedding $G L_n \hookrightarrow G L_{n+1}(R)$ defines an embedding $V_n(R) \longleftrightarrow V_{n+1}(R)$ and we'l1 define $V_{\infty}(R)$ as $\underset{\rightarrow}{\lim _n} V_n(R)$. \\
Finally for $i \geq 1$, put $$k_{i, n}(R)=\pi_{i-1}\left(V_n(R)\right)$$ and $k_i(R)=k_{i, \infty}(R)=\lim _{\rightarrow} k_{i, n}(R)$ (compare [26], [27]). Evidently $K_{1, n}(R)=G L_n(R) / E_n(R)$ and $K_{i, n}(R)$ is a group if $i \geq 2$, and this group is abelian if $i \geq 3$. Moreover the $K_i(R)$ are abelian groups for all $i \geq 1$ (see [26], [27]). The connected component of $V_n(R)$ passing through $T_n$ equals $V\left(E_n(R),\left\{T_n^\sigma(R)\right\}\right)$. It is easy to show that the universal covering space of $V_n\left(E_n(R),\left\{T_n^\sigma(R)\right\}\right)$ equals $V\left(S t(R),\left\{T_n^\sigma(R)\right\}\right)$, where $T_n^\sigma$ is identified with the subgroup of $S t_n(R)$ generated by the $x_{i j}(a)$ with a $\varepsilon R, i \stackrel{\sigma}{<} j(n \geq 3)$. Hence

\begin{lemm}
    $K_{2, n}(R)=\operatorname{ker}\left(S t_n(R)+E_n(R)\right)$, and $K_{i, n}(R)=\pi_{i-1}\left(V\left(S t_n(R)\right)\right)=\pi_{i-1}\left(W\left(S t_n(R)\right)\right) \quad$ if $\quad i \geq 3 \quad(n \geq 3)$.
\end{lemm}    

Let's define $\overline{S t}_n(R)$ to be the inverse image of $G L_n(R)$ under the projection $S t(R) \rightarrow E(R)$. There is a canonical homomorphism $S t_n(R) \rightarrow \overline{s t}_n(R)$ and stability for $K_1, k_2$ ([10], [20], [22]) shows that this homomorphism is surjective if $n \geq s . r . R+1$ and bijective if $n \geq s . r . R+2$. The spaces $W\left(S t_n(R)\right)$ and $W\left(\overline{S t}_n(R)\right)$ will play an essential role in the sequel. We'll denote them by $W_n(R), \bar{W}_n(R)$, resp. (So $W_n(R)=\bar{W}_n(R)$ if $n \geq$ s.r. $R+2$. )

\begin{lemm}
Denote the canonical embedding $\bar{W}_n(R) \longleftrightarrow \bar{W}_{n+1}(R)$ by $u_n$. If $n \geq s \cdot r . R$ and $x \in \overline{S t}_{n+1}(R)$, then $u_n$ and $u_n \cdot x$ are homotopic. (Here $\left.\left(u_n \cdot x\right)(g)=\left(u_n(g)\right) \cdot x \cdot\right)$)
\end{lemm}

\begin{lemm}
For any $s \in S_{n+1}$ the embeddings $u_n$ and $u_n^s$ are homotopic.   
\end{lemm}


For any simplicial set $X$ we'll denote by $C_*(X)$ its chain complex, i.e., the complex of abelian groups with $C_p(x)$ equal to the free abelian group generated by the p-simplices of $X$ and each differential equal to an alternating sum of homomorphisms induced by taking faces. It is well known that $C_*(X)$ is homotopy equivalent to the singular complex of the geometric realization of $X$. In view of (1.5) the maps of complexes $C_*\left(u_n\right), C_*\left(u_n(n, n+1)\right): C_*\left(\bar{W}_n(R)\right)+C_*\left(\bar{W}_{n+1}(R)\right)$ are homotopic. Looking through the proof of (1.5) one sees that the corresponding homotopy operator $\phi_{n+1}^k: C_p\left(\bar{W}_n(R)\right)+C_{p+1}\left(\bar{W}_{n+1}(R)\right)$ may be taken in the following form: (We denote $x_{k, n+1}(1)$ by $x_k$ and
$$
\begin{aligned}
& \left.x_{n+1, k}(-1) \text { by } y_k\right) \\
& \phi_{n+1}^k\left(\alpha_0, \ldots, \alpha_p\right)=\sum_{i=0}^p(-1)^{i+1}\left[\left(\alpha_0^{x_k y_k}, \ldots, \alpha_i x_k y_k, \alpha_i^{(k, n+1)}, \ldots, \alpha_p^{(k, n+1)}\right)\right. \\
& \quad-\left(\alpha_0^{x_k y_k}, \ldots, \alpha_i{ }^x y_k, \alpha_i x_k y_k, \ldots, \alpha_p^{x_k y_k}\right) \\
& \quad+\left(\alpha_0^{x_k} \cdot y_k, \ldots, \alpha_i^{x_k} \cdot y_k, \alpha_i{ }^{x_k y_k}, \ldots, \alpha_p{ }_k y_k\right)-\left(\alpha_0 y_k, \ldots, \alpha_i y_k, \alpha_i, \ldots, \alpha_p\right) \\
& \left.\quad+\left(\alpha_0 y_k, \ldots, \alpha_i y_k, \alpha_i^{x_k} \cdot y_k, \ldots, \alpha_p^{x_k} \cdot y_k\right)-\left(\alpha_0 y_k, \ldots, \alpha_i y_k, \alpha_i y_k, \ldots, \alpha_p y_k\right)\right]
\end{aligned}
$$


\begin{lemm}
The homotopy operators $\phi_{n+1}^k$ have the following properties:
    \begin{enumerate}
        \item $(\partial - \alpha(k, n+1))=d \phi_{n+1}^k(\alpha)+\phi_{n+1}^k(d \alpha)$, where $\alpha=\left(\alpha_0, \ldots, \alpha_p\right)$ is a $p$-simplex of $\bar{W}_n(R)$.
        \item $\phi_{n+1}^n \mid C_*\left(\bar{W}_{n-1}(R)\right)=0$.
        \item For any $s \in S_n$ the following formula is valid:
        $$
        \phi_{n+1}^k\left(\alpha^s\right)=\left[\phi_{n+1}^s(k)(\alpha)\right]^s
        $$
        \item $\phi_{n+1}^k \mid C_*\left(\bar{W}_{n-1}(R)\right)=\left(\phi_n^k\right)(n+1, n)$   
    \end{enumerate}
\end{lemm}

\begin{lemm}
Suppose $c \in C_p\left(\bar{W}_{n-q}(R)\right)$, dc $\& C_{p-1}\left(\bar{W}_{n-q-1}(R)\right)$. Set
    $$
    \begin{aligned}
    & c_0=c, c_1=\phi_{n-q+1}^{n-q}\left(c_0\right) \& c_{p+1}\left(\bar{W}_{n-q+1}(R)\right), \ldots, c_k \\
    & =\phi_{n-q+k}^{n-q+k-1}\left(c_{k-1}\right) \varepsilon c_{p+k}\left(\bar{w}_{n-q+k}(R)\right) \text {. Then, if } k \geq 1 \text {, we have: } \\
    & d c_k=c_{k-1}-c_{k-1}^{(n-q+k, n-q+k-1)}+\ldots+(-1)^k c_{k-1}^{(n-q+k, \ldots, n-q)} . \\
    &
    \end{aligned}$$
\end{lemm}


\subsection{The Aciclicity Theorem}

If $X$ is an arbitrary set, we'll denote by $F_m(X)$ the partially ordered set of functions defined on non-empty subsets of $\{1, \ldots, m\}$ and taking values in $X$. The partial ordering is defined as follows:
$$
f \leq g \Leftrightarrow \operatorname{dom} f \subset \operatorname{dom} g,\left.g\right|_{\text {dom }} f=f .
$$
(Here dom $f$ is the subset of $\{1, \ldots, m\}$ where $f$ is defined). Following van der Kallen [11] we'll say that $F \subset F_m(X)$ satisfies the chain condition if $F$ contains with any function all its restrictions (to non-empty subsets of its domain). It is clear that $f$ and $g$ have a common restriction if and only if there exists i $\varepsilon\{1, \ldots, m\}$ such that $f$ and $g$ are defined at $i$ and equal at $i$. In this case there obviously exists a maximal common restriction inf( $f, g)$.

If $F \subset F_m(X)$ satisfies the chain condition, then by $F_*$ we'11 denote the geometric realization of the semi-simplicial set, whose non-degenerate p-simplices are the functions $f \varepsilon F$ with $\mid$ dom $f \mid=p+1$, and whose faces are defined by the formulas $d_j(f)=\left.f\right|_{\left\{i_0, \ldots, \hat{i}_j, \ldots, i_p\right\}}$ where $\left\{i_0, \ldots, i_p\right\}=\operatorname{dom} f,\left(i_0<\ldots<i_p\right)$. If $f \varepsilon F,|\operatorname{dom} f|=p+1$, then by $|f|$ we'll denote the corresponding p-simplex of $F_*$. It is clear that $|f| \cap|g|$ is either empty or else equals $|\inf (f, g)|$. In particular, $F_*$ is a simplicial space [7].

Let $R$ be a ring (associative with identity), $R^{\infty}$ the free left $R$-module on the basis $e_1, \ldots, e_n, \ldots$, and $R^n$ its submodule generated by $e_1, \ldots, e_n$. If $X$ is any subset of $R^{\infty}$, then by $U_m(X)$ we' 11 denote the subset of $F_m(X)$ consisting of those functions $f$ for which $f\left(i_0\right), \ldots, f\left(i_p\right)$ is a unimodular frame (i.e., a basis of a free direct summand of $R^{\infty}$ ), where $\left\{i_0, \ldots, i_p\right\}=\operatorname{dom}(f)$.

\begin{theo}
Suppose $R$ is a ring, $r=$ s.r.R and $m, n$ are natural numbers. Then $U_m\left(R^n\right)$ is $\min (m-2, n-r-1)$-acyclic.
\end{theo}


\begin{coro}
    $U_n\left(R^n\right) \quad \text { is }(n-r-1)-\operatorname{acyc} 1 \text { ic. }$
\end{coro}

\begin{coro}
Consider in $\mathrm{St}_{n+1}(\Lambda)$ the following subgroups: $A^i=\left\{\alpha: e_i \cdot \pi(\alpha)=e_i\right\}(i=1, \ldots, n+1)$ and consider the simplicial set $Z^{\prime}\left(S t_{n+1}(R), A^i\right)$ constructed as in (2.5), but using left cosets instead of right cosets. This simplicial set is $(n-r)$-acyclic.  
\end{coro}














\section{Whitehead's K-theory}

\section{Quillen's K-theory}


\part{Homological stability}
\chapter{Motivation}


\cite{kupersHomstabPdf2021}

The symmetric group $\Sigma_n$ is the group of bijections of the finite set $\underline{n}=\{1, \ldots, n\}$, under composition. The classifying space $B G$ of a discrete group $G$, such as $\Sigma_n$, is the connected space determined uniquely up to weak homotopy equivalence by the property
$$
\pi_*(B G)= \begin{cases}G & \text { if } *=1, \\ 0 & \text { otherwise }\end{cases}
$$

It can be constructed by extracting from $G$ the groupoid $* / / G$ given by:
- a single object *,
- morphisms given by $* \xrightarrow{g} *$ for $g \in G$, and
- composition given by multiplication.

We then take its nerve to obtain a simplicial set, and take the geometric realisation to get a topological space $|N(* / / G)|$; this is a model for $B G$. Exercise 1.3.1 proves it indeed has the desired property.


\begin{prop}
    $H_*\left(B \Sigma_n ; \mathbb{Z}\right)$  is the same as computing the group homology of $\Sigma_n$ with coefficients in $\mathbb{Z}$.
\end{prop}
Let us compute these groups and the homology of their classifying spaces for the first few values of $n$.

\begin{example}
    \begin{enumerate}
        \item For $n=0,1$, the group $\Sigma_n$ is trivial so its classifying space is weakly contractible and hence has trivial homology.
        \item Example 1.1.4. For $n=2, \Sigma_2$ is isomorphic to the cyclic abelian group $\mathbb{Z} / 2$. Then $B \mathbb{Z} / 2$, as constructed above, is homotopy equivalent to $\mathbb{R} P^{\infty}$. We conclude that
        $$
        H_*(B \mathbb{Z} / 2 ; \mathbb{Z})=H_*\left(\mathbb{R} P^{\infty} ; \mathbb{Z}\right)= \begin{cases}\mathbb{Z} & \text { if } *=0 \\ \mathbb{Z} / 2 & \text { if } *>0 \text { is odd, } \\ 0 & \text { if } *>0 \text { is even. }\end{cases}
        $$
    \item
        Example 1.1.5. For $n=3$, the group $\Sigma_3$ is the dihedral group $D_3$ with 6 elements (i.e. the symmetries of a triangle). A more complicated computation given in Exercise 1.3.5 yields the homology of $D_3$ :
$$
H_*\left(B D_3 ; \mathbb{Z}\right)=\left\{\begin{array}{lll}
\mathbb{Z} & \text { if } *=0 \\
\mathbb{Z} / 2 & \text { if } *>0 \text { and } * \equiv 1 \quad(\bmod 4) \\
\mathbb{Z} / 6 & \text { if } *>0 \text { and } * \equiv 3 \quad(\bmod 4), \\
0 & \text { otherwise }
\end{array}\right.
$$
    \end{enumerate}
\end{example}

\paragraph*{Conjectures}

\begin{enumerate}
    \item $\text { Each reduced homology group } \widetilde{H}_d\left(B \Sigma_n ; \mathbb{Z}\right) \text { is finite and has small exponent. }$
    \item The homology in fixed degree $*=d$ becomes independent of $n$ as $n \rightarrow \infty$.
    \item Before becoming independent of $n$, the homology only increases in size.
    \item The $p$-power torsion only changes when $p \mid n$.
\end{enumerate}


If we want to attempt to prove (2)-(4), we need a better way to compare the homology groups for different $n$ than just as abstract abelian groups. This is done by observing that the inclusion $\underline{n} \hookrightarrow \underline{n+1}$ of finite sets gives a homomorphism
$$
\sigma: \Sigma_n \longrightarrow \Sigma_{n+1},
$$
by extending a permutation of $\underline{n}$ by the identity on $n+1 \in \underline{n+1}$ to a permutation of $n+1$. Our construction of $B G$ is natural in groups and homomorphisms, so this homomorphism induces a map
$$
\sigma: B \Sigma_n \longrightarrow B \Sigma_{n+1},
$$
which in turn induces a map $\sigma_*: H_*\left(B \Sigma_n ; \mathbb{Z}\right) \rightarrow H_*\left(B \Sigma_{n+1} ; \mathbb{Z}\right)$ on homology. We can then give sharper formulations of (2)-(4) in terms of these stabilisation maps:
(2') The maps $\sigma_*$ are isomorphisms in a range increasing with $n$.\\
(3') The maps $\sigma_*$ are injective.\\
(4') The maps $\sigma_*$ are isomorphisms on $p$-power torsion unless $p \mid n+1$.\\
Property (1) holds for all finite groups, and the result which proves it also implies (4'):

\begin{prop}
For a finite group $G, \widetilde{H}_*(B G ; \mathbb{Z}[1 /|G|])=0$. More generally, for $H \subset G$ the map $\iota_*: H_*(B H ; \mathbb{Z}[1 /[G: H]]) \rightarrow H_*(B G ; \mathbb{Z}[1 /[G: H]])$ admits a right inverse $\tau$ (i.e. $\iota_* \circ \tau=\mathrm{id}$ ).
\end{prop}

To deduce (4') from Proposition 1.1.6, note that $\left[\Sigma_{n+1}: \Sigma_n\right]=n+1$ so by the long exact sequence on homology groups so that $H_*\left(B \Sigma_n ; \mathbb{Z}\right) \rightarrow H_*\left(B \Sigma_{n+1} ; \mathbb{Z}\right)$ is surjective after inverting $n+1$. Now set $n+1$ equal to $p$ and invoke (3').
It is phenomenon indicated by (2') that is the subject of this minicourse:

A sequence $X_0 \xrightarrow{\sigma} X_1 \xrightarrow{\sigma} X_2 \xrightarrow{\sigma} \cdots$ exhibits \textbf{homological stability} if the maps $\sigma_*: H_*\left(X_n ; \mathbb{Z}\right) \rightarrow H_*\left(X_{n+1} ; \mathbb{Z}\right)$ are isomorphisms in a range of degrees * increasing with $n$.

In the next two lectures we will prove the following result, due to Nakaoka [Nak60] (though he proved much more):
\begin{theo}
The sequence $B \Sigma_0 \xrightarrow{\sigma} B \Sigma_1 \xrightarrow{\sigma} B \Sigma_2 \xrightarrow{\sigma} \cdots$ exhibits homological stability. More precisely, the induced map
    $$
    \sigma_*: H_*\left(B \Sigma_n ; \mathbb{Z}\right) \longrightarrow H_*\left(B \Sigma_{n+1} ; \mathbb{Z}\right)
    $$
    is surjective if $* \leq \frac{n}{2}$ and an isomorphism if $* \leq \frac{n-1}{2}$.
\end{theo}

Remark 1.1.9. Of course, if we know property (3') holds then the range in the previous theorem in which $\sigma_*$ is an isomorphism improves to $* \leq \frac{n}{2}$. However, property ( 3 ') is rather special—related to the existence of transfer maps-and you should not expect it to hold for general sequences of classifying spaces of groups. We will not comment on it again, but see Exercise 1.3.6.\\
Remark 1.1.10. The ranges in the previous remark are optimal among those of the form $* \leq a n+b$ with $a, b \in \mathbb{Q}$.

\section{Applications}

Homological stability is a structural property of a sequence of groups, or more
generally topological spaces, but it is also useful tool. In fact, many homological stability theorems are proven in service of obtaining other mathematical results. To illustrate this, I now want to explain some straightforward applications of Theorem 1.1.8. These concern the transfer of information from low n to high n and vice-versa. They can be
obtained by other methods as well, but their generalisations to other sequences of groups
often can not.

\subsection{Altenating groups}

Recall that for path-connected $X$, the Hurewicz map $\pi_1(X) \rightarrow H_1(X ; \mathbb{Z})$ coincides with abelianisation (we are suppressing the basepoint). In particular, the map $G \rightarrow$ $H_1(B G ; \mathbb{Z})$ induces an isomorphism $G^{\text {ab }} \rightarrow H_1(B G ; \mathbb{Z})$ naturally in $G$. Thus we can understand the abelianisation of $\Sigma_n$ by computing its first homology group.
The sign homomorphism sign: $\Sigma_n \rightarrow \mathbb{Z} / 2$ yields a map
$$
\text { sign: } B \Sigma_n \longrightarrow B \mathbb{Z} / 2,
$$
which induces a map on homology. This is compatible with stabilisation, in the sense that sign $\circ \sigma=$ sign, so we get a commutative squares
$$
\begin{array}{cc}
H_1\left(B \Sigma_{n-1} ; \mathbb{Z}\right) \xrightarrow{\sigma_*} H_1\left(B \Sigma_n ; \mathbb{Z}\right) \\
\downarrow_{\text {sign }} & \mid \text { sign } \\

\mathbb{Z} / 2 \xlongequal{Z} / 2 .
\end{array}
$$

The map $H_1\left(B \Sigma_2 ; \mathbb{Z}\right) \rightarrow \mathbb{Z} / 2$ is an isomorphism because sign: $\Sigma_2 \rightarrow \mathbb{Z} / 2$ is. By Theorem 1.1.8, in the commutative diagram
the right-most top horizontal map is surjective and the other top horizontal maps are isomorphisms. A single diagram chase then deduces from the fact that the left-most vertical map is an isomorphism that all other vertical maps are.

Thus we have used homological stability to prove that
$$
\text { sign: } \Sigma_n \longrightarrow \mathbb{Z} / 2
$$
is the abelianisation for $n \geq 2$, or equivalently that the kernel of the sign homomorphism is exactly the subgroup $\left[\Sigma_n, \Sigma_n\right]$ generated by commutators. Recalling that this kernel is exactly the alternating group $A_n$, we conclude that:

\begin{theo}
$\left[\Sigma_n, \Sigma_n\right]=A_n$.
\end{theo} 

Remark 1.2.2. This is a fact you likely knew already, and elementary group-theoretic arguments exist. We could have used this fact instead to give an elementary proof of Theorem 1.1.8 in degree $*=1$.

\section{Group Completion}

Homological stability implies that for in fixed degree $*$, for $n$ sufficienty large the canonical map
$$
H_*\left(B \Sigma_n ; \mathbb{Z}\right) \longrightarrow \underset{n \rightarrow \infty}{\operatorname{colim}} H_*\left(B \Sigma_n ; \mathbb{Z}\right)
$$
is an isomorphism; the right hand side is known as the stable homology. This has two somewhat tautological consequences:
1. We can compute the right side from the left side.
2. We can compute the left side from the right side.

This is particularly interesting because the stable homology on the right side has a more familiar description.

When we constructed the stabilisation map, we used that inclusion $\underline{n} \rightarrow \underline{n+1}$ yields a homomorphism $\Sigma_n \rightarrow \Sigma_{n+1}$. More generally, disjoint union induces a homomorphism $\Sigma_n \times \Sigma_m \rightarrow \Sigma_{n+m}$, which yields "multiplication" maps
$$
B \Sigma_n \times B \Sigma_m \longrightarrow B \Sigma_{n+m},
$$
making the space $\bigsqcup_{n \geq 0} B \Sigma_n$ into a unital topological monoid (these are associative but not commutative, and it is probably better to say $E_1$-space since that is a homotopy-invariant notion).

\begin{theo}[McDuff-Segal]
     If $M$ is a homotopy-commutative unital associative topological monoid, then $H_*(M ; \mathbb{Z})\left[\pi_0^{-1}\right] \cong H_*(\Omega B M ; \mathbb{Z})$.    
\end{theo}

%entender mejor esta parte

\section{Serre's finiteness theorem and variations}

Let us now use Corollary 1.2.6. By (1) the groups $H_*\left(B \Sigma_n ; \mathbb{Z}\right)$ are finite for $*>0$. By Theorem 1.1.8 the same is true for the stable homology as long as restrict to degrees $* \leq \frac{n}{2}$. Since $n$ is arbitrary, the stable homology is finite in all positive degrees. This has the following consequence:

\begin{theo}
$\pi_*(\mathbb{S})$ is finite for all $*>0$.
\end{theo}

%we can say something smimilar about torsion....

Exercise 1.3.8 (Using Serre's finiteness theorem). Serre proved that $\pi_*(\mathbb{S})$ is finite for $*>0$. Combine this with Corollary 1.2.6 and Exercise 1.3.6 to prove that the sequence $B \Sigma_0 \xrightarrow{\sigma} B \Sigma_1 \xrightarrow{\sigma} B \Sigma_2 \xrightarrow{\sigma} \cdots$ exhibits homological stability. (Hint: you will not be able to give an explicit range.)

Remark 1.3.9. See [McD75] for a similar qualitative argument for configuration spaces of manifolds.


\chapter{Homological stability for symmetric groups}











\printbibliography % Add this line to print the bibliography section



\end{document}