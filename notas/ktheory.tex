
\part{K-theory}


The subject can be said to begin with Alexander Grothendieck (1957), who used it to formulate his Grothendieck-Riemann-Roch theorem. It takes its name from the German Klasse, meaning "class". [4]
Grothendieck needed to work with coherent sheaves on an algebraic variety $X$. Rather than working directly with the sheaves, he defined a group using isomorphism classes of sheaves as generators of the group, subject to a relation that identifies any extension of two sheaves with their sum. The resulting group is called $K(X)$ when only locally free sheaves are used, or $G(X)$ when all are coherent sheaves. Either of these two constructions is referred to as the Grothendieck group; $K(X)$ has cohomological behavior and $G(X)$ has homological behavior.

If $X$ is a smooth variety, the two groups are the same. If it is a smooth affine variety, then all extensions of locally free sheaves split, so the group has an alternative definition.

In topology, by applying the same construction to vector bundles, Michael Atiyah and Friedrich Hirzebruch defined $K(X)$ for a topological space $X$ in 1959, and using the Bott periodicity theorem they made it the basis of an extraordinary cohomology theory. It played a major role in the second proof of the Atiyah-Singer index theorem (circa 1962). Furthermore, this approach led to a noncommutative K-theory for $\mathrm{C}^*$-algebras.

Already in 1955, Jean-Pierre Serre had used the analogy of vector bundles with projective modules to formulate Serre's conjecture, which states that every finitely generated projective module over a polynomial ring is free; this assertion is correct, but was not settled until 20 years later. (Swan's theorem is another aspect of this analogy.)


The other historical origin of algebraic K-theory was the work of J. H. C. Whitehead and others on what later became known as Whitehead torsion.

There followed a period in which there were various partial definitions of higher K-theory functors. Finally, two useful and equivalent definitions were given by Daniel Quillen using homotopy theory in 1969 and 1972. A variant was also given by Friedhelm Waldhausen in order to study the algebraic K-theory of spaces, which is related to the study of pseudo-isotopies. Much modern research on higher K-theory is related to algebraic geometry and the study of motivic cohomology.

The corresponding constructions involving an auxiliary quadratic form received the general name L-theory. It is a major tool of surgery theory.

In string theory, the K-theory classification of Ramond-Ramond field strengths and the charges of stable Dbranes was first proposed in 1997.[5]


\chapter{Milnor's K-theory}
References \cite{srinivasAlgebraicKTheory1996}

Let $R$ be an associative ring (with 1 ), and let $\mathcal{P}(R)$ denote the category of finitely generated projective $R$-modules. We define the Grothendieck group $K_0(R)$ to be the quotient
$$
K_0(R)=\mathcal{F} / \mathcal{R},
$$
$\mathcal{F}=$ free Abelian group on the isomorphism classes of projective modules in $\mathcal{P}(R)$,
$\mathcal{R}=$ subgroup generated by elements
$$
[P \oplus Q]-[P]-[Q], \text { for all } P, Q \in \mathcal{P}(R) \text {. }
$$

Thus, for any $P, Q \in \mathcal{P}(R),[P]=[Q]$ in $K_0(R) \Longleftrightarrow P \oplus P^{\prime} \cong Q \oplus P^{\prime}$ for some $P^{\prime} \in \mathcal{P}(R) \Longleftrightarrow P \oplus R^n \cong Q \oplus R^n$ for some $n \geq 0$ Further, we can find $Q^{\prime} \in \mathcal{P}(R)$ such that $P^{\prime} \oplus Q^{\prime} \cong R^n$ for some $n$, since $P^{\prime}$ is a quotient of some $R^n\left(P^{\prime}\right.$ is finitely generated) and $P^{\prime}$ is projective. Hence $P \oplus P^{\prime} \cong Q \oplus P^{\prime} \Longrightarrow P \oplus R^n \cong Q \oplus R^n$.\\
If $f: R \rightarrow S$ is a homomorphism of rings, $f$ induces a functor $\mathcal{P}(R) \rightarrow$ $\mathcal{P}(S)$ given by $P \longmapsto S \otimes_R P$. This preserves direct sums, and hence induces a homomorphism $f_*: K_0(R) \rightarrow K_0(S)$.

\begin{prop}
    \begin{enumerate}
        \item Let $(R, \mathcal{M})$ be a local ring, i.e., $\mathcal{M}$ is a 2 -sided maximal ideal, and $R-\mathcal{M}=$ $R^*$. Then $K_0(R)=\mathbb{Z}$, with a generator given by the class of the free $R$-module of $\operatorname{rank} 1$.
        \item Let $R$ be a Dedekind domain, i.e., a commutative Noetherian integrally closed domain such that every non-zero prime ideal of $R$ is maximal. Then $K_0(R) \cong \mathbb{Z} \oplus C \ell(R)$ where $C \ell(R)$ is the ideal class group of $R$, the group of isomorphism classes of invertible ideals (with tensor product as the group operation).
    \end{enumerate}
\end{prop}

\section{$K_1$}

Let $E_n(R)$ be the subgroup of \textbf{elementary matrices}, defined to be the group generated by the matrices $e_{i j}^{(n)}(\lambda), 1 \leq i \neq j \leq n, \lambda \in R$, where $e_{i j}^{(n)}(\lambda)$ is the unipotent matrix whose only non-trivial off-diagonal entry is $\lambda$ in the $(i, j)$ th position. Thus, if $i<j$, then $e_{i j}^{(n)}(\lambda)$ has the form 

\[
e_{ij}^{(n)}(\lambda) =
\begin{pmatrix}
1 & 0 & \cdots & 0 & \cdots & 0 \\
0 & 1 & \cdots & 0 & \cdots & 0 \\
\vdots & \vdots & \ddots & \vdots & \ddots & \vdots \\
0 & 0 & \cdots & 1 & \lambda & \cdots & 0 \\
\vdots & \vdots & \ddots & 0 & 1 & \ddots & \vdots \\
0 & 0 & \cdots & 0 & \cdots & 0 & 1
\end{pmatrix}
\]

Let $G L_n(R) \hookrightarrow G L_{n+1}(R)$ by
$$
A \longmapsto\left[\begin{array}{ll}
A & 0 \\
0 & 1
\end{array}\right]
$$
and let $G L(R)=\lim _{\rightarrow} G L_n(R)$. Similarly, let $E(R)=\lim _{\rightarrow} E_n(R)$. Since $e_{i j}^{(n)}(\lambda) \longmapsto e_{i j}^{(n+1)}(\lambda)$ under $E_n(R) \hookrightarrow E_{n+1}(R)$, we obtain matrices $e_{i j}(\lambda)$ $\in E(R)$ as the common image of all the $e_{i j}^{(n)}(\lambda)$ for $n \geq i, j$, and $E(R)$ is the subgroup of $G L(R)$ generated by the $e_{i j}(\lambda)$. The $e_{i j}(\lambda)$ satisfy the following identities:

\begin{enumerate}
    \item $e_{i j}(\lambda) \cdot e_{i j}(\mu)=e_{i j}(\lambda+\mu), \forall \lambda, \mu \in R$
    \item $\left[e_{i j}(\lambda), e_{k \ell}(\mu)\right]=1$ for $j \neq k, i \neq \ell$
    \item $\left[e_{i j}(\lambda), e_{j k}(\mu)\right]=e_{i k}(\lambda \mu)$ for $i \neq k, \forall \lambda, \mu \in R$.
\end{enumerate}

\begin{lemm}
    \begin{enumerate}
        \item $E_n(R)$, $n/geq 3$, and $E(R)$ are perfect groups.
        \item (Whitehead) $E(R)=[E(R), E(R)]=[G L(R), G L(R)]$.
    \end{enumerate}
\end{lemm}

$$\begin{aligned} K_1(R) & =G L(R) /[G L(R), G L(R)] \\ & =G L(R) / E(R)=H_1(G L(R), \mathbb{Z})\end{aligned}$$


\begin{prop}
Let $(R, \mathcal{M})$ be a (possibly non-commutative) local ring. Then the natural $\operatorname{map} G L_1(R) \rightarrow K_1(R)$ induces an isomorphism
    $$
    R^* /\left[R^*, R^*\right] \cong K_1(R)
    $$    
\end{prop}

\begin{coro}
[Dieudonné]. If $D$ is a division ring, then $K_1(D) \cong$ $D^* /\left[D^*, D^*\right]$, induced by the Dieudonné determinant $G L(D) \rightarrow\left(D^*\right)^{a b}$.    
\end{coro}

\section{$K_2$}

Let $R$ be a ring with identity. The \textbf{$n$th Steinberg group} $S t_n(R)$ is defined to be the quotient of the free group on symbols $x_{i j}^{(n)}(\lambda)$ for $1 \leq$ $i, j \leq n, i \neq j$, and for all $\lambda \in R$, modulo the normal subgroup generated by the words:
\begin{enumerate}
    \item $x_{i j}^{(n)}(\lambda) \cdot x_{i j}^{(n)}(\mu) \cdot x_{i j}^{(n)}(\lambda+\mu)^{-1}$ for all $i, j$, for all $\lambda, \mu \in R$
    \item $\left[x_{i j}^{(n)}(\lambda), x_{k \ell}^{(n)}(\mu)\right]$ for $i \neq \ell, k \neq j$, for all $\lambda, \mu \in R$
    \item $\left[x_{i j}^{(n)}(\lambda), x_{j k}^{(n)}(\mu)\right] \cdot x_{i k}^{(n)}(\lambda \mu)^{-1}$ for $i \neq k$, for all $\lambda, \mu \in R$.
\end{enumerate}

By properties of elementary matrices, we have a natural surjection $\phi_n: S t_n(R) \rightarrow E_n(R)$, given by $\phi_n\left(x_{i j}^{(n)}(\lambda)\right)=e_{i j}^{(n)}(\lambda)$. We also have natural homomorphisms $\operatorname{St}_n(R) \rightarrow$ $S t_{n+1}(R)$ (which need not be injective), and so we obtain the infinite Steinberg group
$
S t(R)=\lim _{\rightarrow} S t_n(R),
$
and the surjection $\phi: \operatorname{St}(R) \longrightarrow E(R)$. Then $$K_2(R):=\ker \phi.$$






% \newpage
% For $n \geq 3$ the \textbf{Steinberg group} $\operatorname{St}_n(R)$ of a ring $R$ is the group defined by generators $x_{i j}(r)$, with $i, j$ a pair of distinct integers between 1 and $n$ and $r \in R$, subject to the following "Steinberg relations":
% $$
% \begin{gathered}
% x_{i j}(r) x_{i j}(s)=x_{i j}(r+s), \\
% {\left[x_{i j}(r), x_{k \ell}(s)\right]= \begin{cases}1 & \text { if } j \neq k \text { and } i \neq \ell, \\
% x_{i \ell}(r s) & \text { if } j=k \text { and } i \neq \ell, \\
% x_{k j}(-s r) & \text { if } j \neq k \text { and } i=\ell .\end{cases} }
% \end{gathered}
% $$

% As observed in (1.3.1), the Steinberg relations are also satisfied by the elementary matrices $e_{i j}(r)$ which generate the subgroup $E_n(R)$ of $G L_n(R)$. Hence there is a canonical group surjection $\phi_n: S t_n(R) \rightarrow E_n(R)$ sending $x_{i j}(r)$ to $e_{i j}(r)$.

% The Steinberg relations for $n+1$ include the Steinberg relations for $n$, so there is an obvious map $S t_n(R) \rightarrow S t_{n+1}(R)$. We write $S t(R)$ for $\lim _{\longrightarrow} S t_n(R)$ and observe that by stabilizing, the $\phi_n$ induce a surjection $\phi: S t(R) \rightarrow E(R)$.


% The group $K_2(R)$ is the kernel of $\phi: S t(R) \rightarrow E(R)$. Thus there is an exact sequence of groups
% $$
% 1 \rightarrow K_2(R) \rightarrow S t(R) \xrightarrow{\phi} G L(R) \rightarrow K_1(R) \rightarrow 1 .
% $$

% It will follow from Theorem 5.2.1 below that $K_2(R)$ is an abelian group. Moreover, it is clear that $S t$ and $K_2$ are both covariant functors from rings to groups, just as $G L$ and $K_1$ are.

% \begin{theo}
% $K_2(R)$ is an abelian group. In fact it is precisely the center of $\operatorname{St}(R)$.
% \end{theo}

% We'll define right actions of the symmetric group $S_n$ on ${ }_{G L}(R)$ and on $S t_n(R)$ by setting
% $$
% \left(\alpha^s\right)_{k, \ell}=\alpha_s(k), s(\ell) ; \quad x_{k \ell}(a)^s=x_s^{-1}(k), s^{-1}(\ell)(a) .
% $$

% These actions are compatible with the projections $S t_n(R) \rightarrow E_n(R)$ and with the homomorphisms $S t_n(R)+S t_{n+1}(R)$ and $G L_n(R)+G L_{n+1}(R)$. In particular, they induce an action on $\overline{S t}_n(R)$.

% \begin{lemm}
%     For any $s \in S_{n+1}$ the embeddings $u_n$ and $u_n^s$ are homotopic.
% \end{lemm} 




\chapter{Grothendieck's K-theory}





\chapter{Whitehead's K-theory}

\chapter{Quillen's K-theory}

\section{The Plus Construction}
(check properties of classifying space)

\begin{theo}[Quillen]
Let $(X, x)$ be a path connected space, $N \triangleleft$ $\pi_1(X, x)$ a perfect normal subgroup. Then there exists a continuous map of pairs $f:(X, x) \longrightarrow\left(X^{+}, x^{+}\right)$such that
\begin{enumerate}
    \item There is an exact sequence
    $$
    0 \longrightarrow N \longrightarrow \pi_1(X, x) \xrightarrow{f_*} \pi_1\left(X^{+}, x^{+}\right) \longrightarrow 0
    $$
    \item For any local coefficient system $L$ on $X^{+}$,
    $$
    f_*: H_n\left(X, f^* L\right) \longrightarrow H_n\left(X^{+}, L\right)
    $$
    is an isomorphism for any $n \geq 0$.
    \item If $g:(X, x) \longrightarrow(Y, y)$ is a continuous map such that
    $$
    N \subset \operatorname{ker}\left(g_*: \pi_1(X, x) \longrightarrow \pi_1(Y, y)\right)
    $$
    then there exists a continuous map $h:\left(X^{+}, x^{+}\right) \longrightarrow(Y, y)$, unique up to homotopy, making the diagram commute.
\end{enumerate}

\end{theo}

\begin{prop}
    \begin{itemize}
        \item Let $(\hat{X}, \hat{x}) \longrightarrow(X, x)$ be the covering of $X$ corresponding to the subgroup $N \triangleleft \pi_1(X, x)$, and let $\left(\tilde{X}^{+}, \tilde{x}^{+}\right)$be the universal covering of $\left(\mathrm{X}^{+}, x^{+}\right)$. Then $\left(\tilde{X}^{+}, \tilde{x}^{+}\right)$is the result of applying the plus construction to $(\hat{X}, \hat{x})$.
        \item Let $f_i:\left(X_i, x_i\right) \longrightarrow\left(X_i^{+}, x_i^{+}\right), i=1,2$ be obtained from the plus construction, for given perfect normal subgroups $N_i \triangleleft$ $\pi_1\left(X_i, x_i\right)$. Then $\left(f_1, f_2\right):\left(X_1 \times X_2,\left(x_1, x_2\right)\right) \longrightarrow\left(X_1^{+} \times X_2^{+},\left(x_1^{+}, x_2^{+}\right)\right)$ is homotopy equivalent to the result of applying the plus construction to $N_1 \times N_2 \triangleleft \pi_1\left(X_1 \times X_2,\left(x_1, x_2\right)\right)$.
    \end{itemize}
\end{prop}

\begin{prop}
    Let $F(R)$ be the homotopy fiber of $B G L(R) \longrightarrow B G L(R)^{+}$. Then:
    \begin{enumerate}
        \item $F(R)$ is acyclic, i.e., $\tilde{H}_n(F(R), \mathbb{Z})=0$ for all $n \geq 0$.
        \item $\pi_1(F(R)) \cong S t(R)$, the Steinberg group.
        \item $\pi_1(F(R))$ acts trivially on $\pi_i(F(R)), i \geq 2$ i.e., $F(R)$ is simple in dimensions $\geq 2$.   
    \end{enumerate} 
\end{prop}

\begin{coro}
$\pi_i\left(B G L(R)^{+}\right) \cong K_i(R), i=1,2$, and
$
\pi_3\left(B G L(R)^{+}\right) \cong H_3(S t(R), \mathbb{Z}) .
$
\end{coro}
Quillen's K-theory is defined as
$$K_i(R)=\pi_i\left(B G L(R)^{+}\right), i \geq 1$$





\chapter{Volodin's K-theory}

Let $G$ be a group and $\left\{G_i\right\}{ }_{i \varepsilon I}$ a family of subgroups. Define $V\left(G,\left\{G_i\right\}\right)$, or just $V(G)$ to be the simplicial complex, whose vertices are the elements of $G$, where $g_0, \ldots, g_p\left(g_i \neq g_j\right)$ form a $p$-simplex if for some $G_i$ all the elements $g_j g_k^{-1}$ lie in $G_i$. If $H$ is another group with a family of subgroups $\left\{H_j\right\}$ and $\phi: G \rightarrow H$ is a homomorphism sending each $G_i$ into some $H_j$, then $\phi$ induces a simplicial map $V(\phi): V(G) \rightarrow V(H)$.

In many situations it is more convenient to use simplicial sets instead of simplicial complexes: Denote by $W\left(G,\left\{G_i\right\}\right)$ the geometric realization of the simplicial set whose p-simplices are the sequences $\left(g_0, \ldots, g_p\right)$ of elements of $G$ (not necessarily distinct) such that for some $G_i$ al1 $g_j g_k^{-1}$ lie in $G_i$, the $r$-th face (resp. degeneracy) of this simplex being obtained by omitting $g_r$ (resp., repeating $g_r$ ). Associating with any p-simplex $\left(g_0, \ldots, g_p\right)$ the linear singular simplex of the space $V(G)$ which sends the $i$-th vertex of the standard simplex to $g_j$, we obtain a map of simplicial sets from $W(G)$ to the simplicial set of singular simplices of $V(G)$ and hence a cellular map (linear on any simplex) from $W(G)$ to $V(G)$. This map is a homotopy equivalence .... %put in simplicial sets

Suppose that $R$ is a ring, $n$ a natural number and $\sigma$ a partial ordering of $\{1, \ldots, n\}$. Define $T_n^\sigma(R)$ to be the subgroup of $G L_n(R)$ consisting of the $\alpha$ with $\alpha_{i j}=1$ and $\alpha_{i j}=0$ if $i \& j$. Subgroups of this form will be called triangular subgroups of $G L_n(R)$. The space $V\left(G L_n(R),\left\{T_n^\sigma(R)\right\}\right)$ will be denoted by $V_n(R)$. Since any partial ordering may be extended to a linear ordering, it suffices to consider linear orderings when defining $V_n(R)$. The natural embedding $G L_n \hookrightarrow G L_{n+1}(R)$ defines an embedding $V_n(R) \longleftrightarrow V_{n+1}(R)$ and we'l1 define $V_{\infty}(R)$ as $\underset{\rightarrow}{\lim _n} V_n(R)$. \\
Finally for $i \geq 1$, put $$k_{i, n}(R)=\pi_{i-1}\left(V_n(R)\right)$$ and $k_i(R)=k_{i, \infty}(R)=\lim _{\rightarrow} k_{i, n}(R)$ (compare [26], [27]). Evidently $K_{1, n}(R)=G L_n(R) / E_n(R)$ and $K_{i, n}(R)$ is a group if $i \geq 2$, and this group is abelian if $i \geq 3$. Moreover the $K_i(R)$ are abelian groups for all $i \geq 1$ (see [26], [27]). The connected component of $V_n(R)$ passing through $T_n$ equals $V\left(E_n(R),\left\{T_n^\sigma(R)\right\}\right)$. It is easy to show that the universal covering space of $V_n\left(E_n(R),\left\{T_n^\sigma(R)\right\}\right)$ equals $V\left(S t(R),\left\{T_n^\sigma(R)\right\}\right)$, where $T_n^\sigma$ is identified with the subgroup of $S t_n(R)$ generated by the $x_{i j}(a)$ with a $\varepsilon R, i \stackrel{\sigma}{<} j(n \geq 3)$. Hence

\begin{lemm}
    $K_{2, n}(R)=\operatorname{ker}\left(S t_n(R)+E_n(R)\right)$, and $K_{i, n}(R)=\pi_{i-1}\left(V\left(S t_n(R)\right)\right)=\pi_{i-1}\left(W\left(S t_n(R)\right)\right) \quad$ if $\quad i \geq 3 \quad(n \geq 3)$.
\end{lemm}    

Let's define $\overline{S t}_n(R)$ to be the inverse image of $G L_n(R)$ under the projection $S t(R) \rightarrow E(R)$. There is a canonical homomorphism $S t_n(R) \rightarrow \overline{s t}_n(R)$ and stability for $K_1, k_2$ ([10], [20], [22]) shows that this homomorphism is surjective if $n \geq s . r . R+1$ and bijective if $n \geq s . r . R+2$. The spaces $W\left(S t_n(R)\right)$ and $W\left(\overline{S t}_n(R)\right)$ will play an essential role in the sequel. We'll denote them by $W_n(R), \bar{W}_n(R)$, resp. (So $W_n(R)=\bar{W}_n(R)$ if $n \geq$ s.r. $R+2$. )

\begin{lemm}
Denote the canonical embedding $\bar{W}_n(R) \longleftrightarrow \bar{W}_{n+1}(R)$ by $u_n$. If $n \geq s \cdot r . R$ and $x \in \overline{S t}_{n+1}(R)$, then $u_n$ and $u_n \cdot x$ are homotopic. (Here $\left.\left(u_n \cdot x\right)(g)=\left(u_n(g)\right) \cdot x \cdot\right)$)
\end{lemm}

\begin{lemm}
For any $s \in S_{n+1}$ the embeddings $u_n$ and $u_n^s$ are homotopic.   
\end{lemm}


For any simplicial set $X$ we'll denote by $C_*(X)$ its chain complex, i.e., the complex of abelian groups with $C_p(x)$ equal to the free abelian group generated by the p-simplices of $X$ and each differential equal to an alternating sum of homomorphisms induced by taking faces. It is well known that $C_*(X)$ is homotopy equivalent to the singular complex of the geometric realization of $X$. In view of (1.5) the maps of complexes $C_*\left(u_n\right), C_*\left(u_n(n, n+1)\right): C_*\left(\bar{W}_n(R)\right)+C_*\left(\bar{W}_{n+1}(R)\right)$ are homotopic. Looking through the proof of (1.5) one sees that the corresponding homotopy operator $\phi_{n+1}^k: C_p\left(\bar{W}_n(R)\right)+C_{p+1}\left(\bar{W}_{n+1}(R)\right)$ may be taken in the following form: (We denote $x_{k, n+1}(1)$ by $x_k$ and
$$
\begin{aligned}
& \left.x_{n+1, k}(-1) \text { by } y_k\right) \\
& \phi_{n+1}^k\left(\alpha_0, \ldots, \alpha_p\right)=\sum_{i=0}^p(-1)^{i+1}\left[\left(\alpha_0^{x_k y_k}, \ldots, \alpha_i x_k y_k, \alpha_i^{(k, n+1)}, \ldots, \alpha_p^{(k, n+1)}\right)\right. \\
& \quad-\left(\alpha_0^{x_k y_k}, \ldots, \alpha_i{ }^x y_k, \alpha_i x_k y_k, \ldots, \alpha_p^{x_k y_k}\right) \\
& \quad+\left(\alpha_0^{x_k} \cdot y_k, \ldots, \alpha_i^{x_k} \cdot y_k, \alpha_i{ }^{x_k y_k}, \ldots, \alpha_p{ }_k y_k\right)-\left(\alpha_0 y_k, \ldots, \alpha_i y_k, \alpha_i, \ldots, \alpha_p\right) \\
& \left.\quad+\left(\alpha_0 y_k, \ldots, \alpha_i y_k, \alpha_i^{x_k} \cdot y_k, \ldots, \alpha_p^{x_k} \cdot y_k\right)-\left(\alpha_0 y_k, \ldots, \alpha_i y_k, \alpha_i y_k, \ldots, \alpha_p y_k\right)\right]
\end{aligned}
$$


\begin{lemm}
The homotopy operators $\phi_{n+1}^k$ have the following properties:
    \begin{enumerate}
        \item $(\partial - \alpha(k, n+1))=d \phi_{n+1}^k(\alpha)+\phi_{n+1}^k(d \alpha)$, where $\alpha=\left(\alpha_0, \ldots, \alpha_p\right)$ is a $p$-simplex of $\bar{W}_n(R)$.
        \item $\phi_{n+1}^n \mid C_*\left(\bar{W}_{n-1}(R)\right)=0$.
        \item For any $s \in S_n$ the following formula is valid:
        $$
        \phi_{n+1}^k\left(\alpha^s\right)=\left[\phi_{n+1}^s(k)(\alpha)\right]^s
        $$
        \item $\phi_{n+1}^k \mid C_*\left(\bar{W}_{n-1}(R)\right)=\left(\phi_n^k\right)(n+1, n)$   
    \end{enumerate}
\end{lemm}

\begin{lemm}
Suppose $c \in C_p\left(\bar{W}_{n-q}(R)\right)$, dc $\& C_{p-1}\left(\bar{W}_{n-q-1}(R)\right)$. Set
    $$
    \begin{aligned}
    & c_0=c, c_1=\phi_{n-q+1}^{n-q}\left(c_0\right) \& c_{p+1}\left(\bar{W}_{n-q+1}(R)\right), \ldots, c_k \\
    & =\phi_{n-q+k}^{n-q+k-1}\left(c_{k-1}\right) \varepsilon c_{p+k}\left(\bar{w}_{n-q+k}(R)\right) \text {. Then, if } k \geq 1 \text {, we have: } \\
    & d c_k=c_{k-1}-c_{k-1}^{(n-q+k, n-q+k-1)}+\ldots+(-1)^k c_{k-1}^{(n-q+k, \ldots, n-q)} . \\
    &
    \end{aligned}$$
\end{lemm}


\subsection{The Aciclicity Theorem}

If $X$ is an arbitrary set, we'll denote by $F_m(X)$ the partially ordered set of functions defined on non-empty subsets of $\{1, \ldots, m\}$ and taking values in $X$. The partial ordering is defined as follows:
$$
f \leq g \Leftrightarrow \operatorname{dom} f \subset \operatorname{dom} g,\left.g\right|_{\text {dom }} f=f .
$$
(Here dom $f$ is the subset of $\{1, \ldots, m\}$ where $f$ is defined). Following van der Kallen [11] we'll say that $F \subset F_m(X)$ satisfies the chain condition if $F$ contains with any function all its restrictions (to non-empty subsets of its domain). It is clear that $f$ and $g$ have a common restriction if and only if there exists i $\varepsilon\{1, \ldots, m\}$ such that $f$ and $g$ are defined at $i$ and equal at $i$. In this case there obviously exists a maximal common restriction inf( $f, g)$.

If $F \subset F_m(X)$ satisfies the chain condition, then by $F_*$ we'11 denote the geometric realization of the semi-simplicial set, whose non-degenerate p-simplices are the functions $f \varepsilon F$ with $\mid$ dom $f \mid=p+1$, and whose faces are defined by the formulas $d_j(f)=\left.f\right|_{\left\{i_0, \ldots, \hat{i}_j, \ldots, i_p\right\}}$ where $\left\{i_0, \ldots, i_p\right\}=\operatorname{dom} f,\left(i_0<\ldots<i_p\right)$. If $f \varepsilon F,|\operatorname{dom} f|=p+1$, then by $|f|$ we'll denote the corresponding p-simplex of $F_*$. It is clear that $|f| \cap|g|$ is either empty or else equals $|\inf (f, g)|$. In particular, $F_*$ is a simplicial space [7].

Let $R$ be a ring (associative with identity), $R^{\infty}$ the free left $R$-module on the basis $e_1, \ldots, e_n, \ldots$, and $R^n$ its submodule generated by $e_1, \ldots, e_n$. If $X$ is any subset of $R^{\infty}$, then by $U_m(X)$ we' 11 denote the subset of $F_m(X)$ consisting of those functions $f$ for which $f\left(i_0\right), \ldots, f\left(i_p\right)$ is a unimodular frame (i.e., a basis of a free direct summand of $R^{\infty}$ ), where $\left\{i_0, \ldots, i_p\right\}=\operatorname{dom}(f)$.

\begin{theo}
Suppose $R$ is a ring, $r=$ s.r.R and $m, n$ are natural numbers. Then $U_m\left(R^n\right)$ is $\min (m-2, n-r-1)$-acyclic.
\end{theo}


\begin{coro}
    $U_n\left(R^n\right) \quad \text { is }(n-r-1)-\operatorname{acyc} 1 \text { ic. }$
\end{coro}

\begin{coro}
Consider in $\mathrm{St}_{n+1}(\Lambda)$ the following subgroups: $A^i=\left\{\alpha: e_i \cdot \pi(\alpha)=e_i\right\}(i=1, \ldots, n+1)$ and consider the simplicial set $Z^{\prime}\left(S t_{n+1}(R), A^i\right)$ constructed as in (2.5), but using left cosets instead of right cosets. This simplicial set is $(n-r)$-acyclic.  
\end{coro}









